% mainfile: ../../main.tex
\chapter{Introduction}\label{ch:speck:introduction}
Noise is ubiquitous in condensed matter physics experiments, and in mesoscopic systems in particular it can easily drown out the sought-after signal.
Hence, characterizing (and subsequently mitigating) noise is an essential task for the experimentalist.
But noise comes in as many different forms as there are types of signal sources and detectors, whether it be a voltage source or a photodetector, and while some instruments have built-in solutions for noise analysis, they vary in functionality and capability.
Moreover, the measured signal often does not directly correspond to the noisy physical quantity of interest, making it desirable to be able to manipulate the raw data before processing.

