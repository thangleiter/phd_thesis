% mainfile: ../../main.tex
\chapter{Conclusion and outlook}\label{ch:speck:conclusion}
\AutoLettrine{In} \thispart, I introduced the \pyspeck \python package for interactive, backend-agnostic noise spectroscopy.
I first presented the theory behind spectral noise estimation based on time series analysis in \cref{ch:speck:theory}.
There, I showed how, by the Wiener-Khinchin theorem, the \gls{psd} $S(\omega)$ is related to a time-dependent signal by the Fourier transform of its autocorrelation function $C(\tau)$.
After discretizing the problem, I discussed how the arising spectral windowing introduces distortions in the estimated \gls{psd} through effects such as spectral leakage and scalloping loss.
The introduction of windowing naturally led to the need for a more efficient method of spectrum estimation to avoid data wastefulness.
This is addressed by Welch's method, which trades spectral resolution for estimation accuracy for a given input size by dividing the input into segments and letting them overlap.
Finally, I discussed the various parameters that influence the properties of the spectrum estimate produced by Welch's method.

In \cref{ch:speck:software}, I then introduced the \pyspeck package that facilitates noise spectroscopy based on the theory developed in the previous chapter.
While in principle agnostic to the specific technique (simple periodogram, Bartlett's method, \etc), it employs Welch's method by default to perform efficient estimation of noise \acrlongpl{psd}.
To make noise spectroscopy as approachable as possible, the package provides a unified interface to various hardware and software backends to manage \acrlong{daq}.
Furthermore, it incorporates rich plotting features to ease data analysis and thereby expedite the feedback loop with the lab.
I first outlined the design choices underlying the source code and explained the mechanism employed to infer \gls{daq} settings from, for example, the physical parameters of the resulting spectrum estimate.
I described the interface to drivers for different hardware instruments that makes it possible to use the same user code for spectrum acquisition independent of the specific instrument used for \acrlong{daq} while at the same time keeping the amount of required driver code at a minimum.
Finally, I showcased the interactive features by means of a typical workflow.
Here, the serial approach of recording single spectra that are saved to disk at a time stands in contrast to the non-persistent live mode, where data is continuously acquired and displayed in a separate thread or process.

%\paragraph{Outlook.\label{par:speck:outlook}}
There are several possible avenues for future development of the \pyspeck package.
An obvious case is adding support for more \gls{daq} hardware instruments by implementing \code{DAQ} interfaces.
Modular devices such as those offered by \sidehref{https://www.qblox.com/research}{QBLOX} and \sidehref{https://www.quantum-machines.co/}{Quantum Machines} are on track to become the new standard in quantum technology labs.
Implementing drivers for these instruments would benefit the adoption of both the instruments and the \pyspeck package.
Another valuable addition would be supporting generic instruments abstracted by \qumada~\cite{Huckemann2025a}.
\qumada is a \qcodes-based measurement framework that provides a unified interface to instruments in order to -- in a similar spirit to \pyspeck's \code{DAQ} interface -- abstract away internals of individual instruments and provide users with a standardized way to interact with them.
This approach should naturally lend itself to a single implementation of the \code{DAQ} class supporting various instruments through the unified \qumada interface.

Next, incorporating noise spectroscopy into the standard measurement workflow of quantum device experiments would allow experimentalists to quickly gauge noise levels as they are performing measurements.
If for some reason the noise changed\sidenote{
    As it happens often, unfortunately.
}
the experimentalist could quickly obtain insight into the noise by analyzing the spectrum.
In a client-server architecture, which is inherently asynchronous, such as Zurich Instrument's \sidehref{https://www.zhinst.com/ch/en/instruments/labone/labone-instrument-control-software}{LabOne} software, this is already possible using the web interface.
But of course the strength of the \pyspeck package stems from its capacity to be utilized in conjunction with any hardware instrument.

\begin{marginlisting}
    \begin{py}[
        fontsize=\footnotesize,%
        breaklines,%
        breakafter=.,%
    ]
        # daq/tee.py
        import dataclasses
        import threading
        import multiprocessing as mp

        from .base import DAQ

        @dataclasses.dataclass
        class TeeDAQ(DAQ):
            settings: mp.managers.DictProxy
            queue: mp.JoinableQueue
            event: threading.Event

            def setup(self, **_):
                settings = self.DAQSettings(self.settings)
                return settings.to_consistent_dict()

            def acquire(self, **_):
                while not self.event.is_set():
                    yield self.queue.get(block=True)
    \end{py}
    \caption[\code{TeeDAQ} template]{
        Template design for a proxy \code{DAQ} implementation to stream noise spectra from an external measurement framework.
        The \code{settings} attribute is a dictionary proxy shared between processes and used to pass acquisition parameters from the measurement framework to \pyspeck.
    }
    \label{lst:speck:conclusion:tee}
\end{marginlisting}

One way to implement such functionality would be to introduce a proxy \code{DAQ} subclass to be used together with the live mode presented in \cref{subsec:speck:software:features:live_view}.
This proxy class would serve as an interface to external measurement software and expose two attributes; first, a data queue, into which the external code could place arbitrary time series data that was obtained during some measurement, and second, a shared dictionary to hold acquisition parameters as these might change between measurements.
Because the live view mode runs in the background, the external measurement framework could push data to the queue whenever new data was taken without obstructing the measurement workflow.

\Cref{lst:speck:conclusion:tee} shows a template design for such a \code{TeeDAQ} class.\sidenote{
    The UNIX shell command \texttt{data = \$(measure(**settings) | tee daq)} should give an indication as to the naming.
}
The \code{setup()} method ignores the input parameters and instead obtains the current settings from the shared \code{settings} proxy.
Similarly, instead of fetching data from an instrument itself, the \code{acquire()} method attempts to fetch data from the shared \code{queue} and blocks the thread if no data is present, thereby efficiently idling and consuming no resources unless triggered by the external caller.
A measurement framework would then interact with the \code{TeeDAQ} object as exemplarized by the following code:
\begin{py}
    daq = TeeDAQ(...)
    spect = Spectrometer(daq)
    view = spect.live_view()
    ...
    data = measure(fs, n_pts)
    daq.settings.update(fs=fs, n_pts=n_pts)
    daq.queue.put(data)
\end{py}
Measurement frameworks integrating with this interface could thus provide experimentalists live feedback on current noise levels with negligible overhead and minimal code adaptation.

Finally, it might be useful to not only allow estimating \glspl{psd} but also \glspl{csd} or \emph{cross-spectra}.
The cross-spectrum\sidenote{
    Again, we use the two terms interchangably unless otherwise indicated, see \cref{sidenote:speck:theory_vocabulary} in \cref{ch:speck:theory}.
}
is the Fourier transform of not the autocorrelation but the cross-correlation function $C(\tau)$ (\cref{eq:speck:autocorrelation}) between two random processes.
Take a set of processes $\{x_1(t), \allowbreak x_2(t), \allowbreak \dotsc, \allowbreak x_n(t)\}$ that correspond to noise measured at different locations in a sample.
The cross-correlation function between variables $x_i$ and $x_j$ is then given by\sidenote{
    Again assuming wide-sense stationary processes.
}
\begin{align}\label{eq:speck:conclusion:crosscorrelation}
    C_{ij}(\tau) = \expval{x_i(t)\conjugate x_j(t + \tau)}.
\end{align}
This function (and its Fourier pair the cross-spectrum $S_{ij}(\omega)$) quantifies the degree of correlation between noise at site $i$ and noise at site $j$.
Unlike the \emph{auto}-spectrum (or self-spectrum), the cross-spectrum is always a complex quantity, even for real $x_i(t)$.
It is not hard to see that for quantum processors, for example, these kinds of correlations could have significant impact on operation, and on error correction in particular~\cite{Aharonov2006,Nickerson2019,Clader2021}.
To incorporate cross-spectra in the \pyspeck package, only small changes should be necessary.

First, the data acquisition logic would need to be adapted.
Two possible routes suggest themselves here; first, specialized \code{DAQ} classes could be implemented that, in place of yielding one batch of time series data, yield two batches each time they are queried.
This approach first of all requires instruments with multiple channels,\sidenote{
    If one sticks to single \code{DAQ} instances managing single instruments.
}
which is not necessarily given.
Furthermore, it would incur additional coding efforts by having to re-implement each \code{DAQ} class for cross-spectra.
On the other hand, it would arguably make synchronization between channels easier to achieve.

A less involved path would adapt the \code{Spectrometer} class to work with multiple \code{DAQ}s.
This would not involve additional driver work\sidenote{
    Except possibly ensuring thread safety and timing synchronization if multiple \code{DAQ} objects communicate with the same physical instrument.
}
and allow the \code{Spectrometer} object to ensure synchronization between the different \code{DAQ}s.
The internal workflow shown in \cref{lst:speck:daq:workflow} would then need to be slightly adapted to the code shown in \cref{lst:speck:conclusion:cross_workflow}.
The downside of this approach is that synchronization of different instruments or channels would need to be taken care of externally.

\begin{listing}[htpb]
    \begin{py}
        daq_1 = MyDAQ(driver_handle, channel=1)
        # Or MyOtherDAQ(driver_handle_2) if another instrument
        daq_2 = MyDAQ(driver_handle, channel=2)
        daqs = (daq_1, daq_2)

        parsed_settings = [daq.setup(**user_settings) for daq in daqs]
        assert all_equal(parsed_settings), "DAQ settings do not match"

        acquisition_generators = [daq.acquire(**parsed_settings[0])
                                  for daq in daqs]
        for data_buffers in zip(*acquisition_generators):
            estimate_csd(*data_buffers)
    \end{py}
    \caption[Proposed \code{DAQ} workflow for cross-spectra]{
        Proposed \code{DAQ} workflow for estimating cross-spectra.
        Each hardware channel (same or different instruments) is assigned to a \code{DAQ} object.
        After instrument configuration, it is asserted that the parameters match.
        Finally, data is fetched from both channels and fed into a \gls{csd} estimator.
        Note that triggering would need to be implemented externally.
    }
    \label{lst:speck:conclusion:cross_workflow}
\end{listing}

Further code adaptations would involve minor changes such as replacing the spectrum estimator with \code{scipy.signal.csd()}~\sidecite{CSDScipy} for \gls{csd} estimation\sidenote{
    In practice, working with the normalized \gls{csd}, or correlation coefficient~\cite{Rojas-Arias2023,Yoneda2023}
    \begin{align}
        r_{ij}(\omega) = \frac{S_{ij}(\omega)}{\sqrt{S_{ii}(\omega)S_{jj}(\omega)}}
    \end{align}
    with $S_{ii}(\omega)$ the \gls{psd} of process $x_i$ would likely be more favorable.
}
and make the plotting conform to complex data.\sidenote{
    It could be worthwile to add a subplot to display both the magnitude and phase of the complex quantity $S_{ij}(\omega)$.
}
% TODO:
%  - improve this part?
%  - spectrogram?
%  - rms plot in live view?
