% mainfile: ../../main.tex
\chapter{Vibration performance}\label{ch:setup:vibrations}
\AutoLettrine{A} microscope's performance is limited chiefly by two factors; first and foremost the resolution and imaging fidelity are limited by the systematic aberrations introduced by the optics.\sidenote{
    Besides the limit set by the wavelength-dependent diffraction, of course.
}
Various types of aberrations exist, and modern microscopes usually include a complex assembly of optics to compensate for these errors.
The second factor is vibration noise.
This becomes more significant the higher the resolution of the microscope simply because ambient, environmental vibrations within the range of human civilization is typically on the order of $\qty{100}{\micro\meter\per\second}$ \gls{rms}~\cite{Gordon1999}.
Comparing that to \glspl{tem} with atomic resolution, it is clear that these instruments require purpose-built rooms to reduce the vibration level to acceptable levels.

%Due to space constraints, the complexity of optics in our microscope is limited.
%Since we do not require the microscope to cover a wide range of wavelengths, chromatic aberrations do not play a significant role.\sidenote{
%    At least when not considering polarization.
%    The excitation extinction ratio is highly dependent on wavelength, \cf \cref{ch:setup:optical}
%% TODO: ref
%}
The demands on the microscope discussed in \thethesis are fortunately much more relaxed as the features we need to resolve are on the micrometer scale.
However, we face the additional challenge of ultra-low temperatures, or rather the manner in which they are achieved.
The microscope is integrated into a \emph{dry} \gls{dr}.
In contrast to a \emph{wet} \gls{dr}, which uses a liquid Helium bath, these systems achieve the pre-cooling necessary for the \ch{^3He}/\ch{^4He} dilution refrigeration cycle to work by adding a secondary refrigeration mechanism, a \gls{ptr}.
These are closed-cycle systems that work with \ch{^4He} compressed to \textasciitilde\qty{21}{\bar} on the high-pressure and \textasciitilde\qty{7}{\bar} on the low-pressure side.
A rotating valve connecting high and low pressure lines to the cryostat in turn produces alternating gas flow inside a regenerator, where the gas absorbs heat at the low-temperature and and deposits heat at the high-temperature end~\cite{Radebaugh2009,DeWaele2011}.
In commercial \glspl{ptr} the frequency of the pulses of Helium gas, determined by the rotary valve motor, is usually fixed at values around \qty{1.5}{\hertz}.

Naturally, the compressor, the rotary valve motor, and the Helium pulses themselves introduce vibrations into the cryostat.
While the cold foot of the \gls{ptr} is not rigidly connected to the cryostat interior,\sidenote{
    In the \odin copper braids connect the cold head to the \gls{pt1} and \gls{pt2} plates.
    There exist commercial systems that use gas exchange instead, for example the \mbox{CryoConcept} HEXA-DRY series~\cite{CryoConceptHexaDry}.
}
the entire cold head assembly rests with rubber feet on the cryostat top plate in the system's delivery status.
Thus, our microscope does not only encounter passive environmental vibrations but also the active disturbance from the \gls{ptr}.
To characterize and improve upon the vibration isolation, I performed vibration noise spectroscopy using the techniques and tools presented in \cref{part:speck}.\sidenote{
    The endeavour was triggered by a sudden increase in visually observed vibrations in the microscope image.
    As it turned out, the cause was a damaged nanopositioner bearing rather than environmental.
}
I employed two different approaches that I lay out in the following; first, using a commercial piezoelectric accelerometer (\cref{sec:setup:vibrations:accel}) and second, using the optical response of a spatial reflectance gradient (\cref{sec:setup:vibrations:optic}).
As will become clear, the two approaches complement each other because they are sensitive to slightly different quantities.

\section{Vibration isolation}\label{sec:setup:vibrations:isolation}
A simple yet effective method of vibration isolation is to suspend the system on passive air springs.
These are typically constructed with two separate air chambers, a spring and a damping chamber, connected by pneumatic tubing.
The load is rigidly mounted to a plunger that rests on a diaphragm sealing the spring chamber.
Excitations of the load induce oscillations in the variable spring chamber volume.
The connection to the fixed-volume damping chamber provides a flow impedance\sidenote{
    The speed of a fluid in laminar flow through a round pipe is proportional to the pressure gradient along the flow direction and to the square of the distance from the wall.
}
that manifests as a damping force to the spring chamber oscillations.

\begin{marginfigure}
    \centering
    \includegraphics{img/pdf/setup/spring_tf}
    \caption[\imgsource{img/pdf/setup/springs.py}]{
        Force transmission function of a damped harmonic oscillator (solid black line).
        Below $\omega=\sqrt{2}\omega_0$ (dotted vertical line), external excitations are amplified (shaded red area).
        For larger damping $\gamma$, the amplification at resonance becomes smaller (not shown).
        Above $\omega=\sqrt{2}\omega_0$, excitations are attenuated (shaded green area).
        The dashed line shows a more realistic model for an air spring~\cite{FabreekaAirSprings}, whose attenuation above the break frequency becomes smaller as the damping constant $\gamma$ is increased, whereas the harmonic oscillator's high-frequency behavior is independent of $\gamma$.
    }
    \label{fig:setup:vibrations:spring:tf}
\end{marginfigure}

Let us adopt a simple toy model to gain an intuition for the behavior of a mass suspended on air springs as function of vibration frequency by modelling it as a damped harmonic oscillator.
We can compute the transfer function $H(s)$ for external perturbations $u(t)$ from the Laplace transform of the Newtonian equation of motion
\begin{equation}\label{eq:setup:spring:eom}
    \ddot{x}(t) + 2\gamma\dot{x}(t) + \omega_0^2 x(t) = m\inverse u(t)
\end{equation}
for the displacement from equilibrium $x(t)$.
Here, $\gamma$ is the damping constant, $\omega_0$ the resonance frequency of the undamped system, and $m$ the oscillator mass.
The amplitude transfer function given by
\begin{equation}\label{eq:setup:spring:tf}
    H(s) = \frac{x(s)}{u(s)} = \frac{1}{m}\frac{1}{s^2 + 2\gamma s + \omega_0^2}
\end{equation}
is drawn in \cref{fig:setup:vibrations:spring:tf} normalized to the DC response for $s=\i\omega$ (solid black line).
Below $\omega=\sqrt{2}\omega_0$, the suspension in fact leads to amplification of external impulses.
This is independent of the damping $\gamma$ and the reason why resonance frequencies as small as possible are desirable in vibration isolation.
Above this frequency, the system attenuates with $\qty{40}{\decibel}$ per decade.
A more accurate model results in the transfer function drawn as a dashed line~\sidecite{FabreekaAirSprings}.
Here, the attenuation at high frequencies is only $\qty{20}{\decibel}$ per decade.

From \cref{fig:setup:vibrations:spring:tf}, we can infer two possible approaches to isolating a mass from vibrations.
The first is to make the system's resonance frequency $\omega_0$ as possible by resting it on a spring damping system.
This maximizes the region in which external influences are attenuated.
The second is to do the opposite, \ie, make the entire system as stiff and thereby $\omega_0$ as large as possible.\sidenote{
    The stiffness $k$ is related to the resonance frequency by $k = m\omega_0^2$.
}
While this minimizes the attenuation region, it also moves the amplification region close to the resonance to higher frequencies, and possibly further away from the external excitation.

What does this mean for our case of a dry \gls{dr}?
The rotary valve motor of the \gls{ptr} generates pulses with frequency \qty{1.4}{\hertz}.
Commercial damping systems that the space constraints in our lab allow to be accommodated, for example the CFM Schiller MAS 25~\cite{CFMSchiller}, have resonance frequencies around $f_0 = \qty{2.5}{\hertz}$, implying the first two harmonics of the \gls{ptr} excitation fall into the amplification area as discussed above.
Hence, the initial isolation concept for the cryostat envisaged mounting the rotary valve motor rigidly to the stiff aluminium item profile frame, which was additionally filled with sand.
% TODO: lay out suspension setup here

\section{Accelerometric vibration spectroscopy}\label{sec:setup:vibrations:accel}
The most straightforward method of measuring vibration noise is an accelerometer.
These are devices that convert translational forces, for example by means of a loaded spring, into electrical signals.
They are mounted rigidly to the \gls{dut} and typically connected to some sort of signal conditioner providing a constant current bias to the sensor and putting out a voltage proportional to the acceleration.
The most sensitive and low-frequency designs use piezoelectric materials like Quartz crystals for sensitivities in the range of \qty{10}{\volt\per\gaccel} with a broadband noise floor of \qty{2}{\micro\gaccel}~\sidecite{WilcoxonAccel}.

To assess the vibration level in the sample position, I designed a small angle bracket onto which the accelerometer\sidenote{
    Wilcoxon 731-207 kindly lent by Marcus Eßer~\cite{WilcoxonAccel}.
}
can be screwed either in vertical or horizontal direction in the sample puck of the \gls{dr}, allowing measurements of the displacement noise along the direction of gravity as well as perpendicular to it and the optical axis.
The accelerometer is connected to the coaxial cables installed in the cryostat via an adapter cable from 10-32 to \gls{sma} connector.
Outside of the cryostat, the signal is routed to a signal conditioner that provides the necessary current bias and outputs a voltage which is digitized by a \dmm connected to the measurement computer.
% TODO tense

\begin{marginlisting}
    \begin{py}[
        fontsize=\footnotesize,
        breaklines,%
        breakafter=.,%
    ]
        from qutil.signal_processing import fourier_space
        from qutil.functools import chain, scaled
        from qutil import const

        sensitivity = scaled(1 / 9.9 / const.g)
        fourier_procfn = chain(
            sensitivity,
            fourier_space.derivative
        )
    \end{py}
    \caption{Functionality to transform the conditioned voltage to displacement in Fourier space.}
    \label{lst:setup:vibrations:accelerometer}
\end{marginlisting}
\begin{figure}
    \centering
    \includegraphics{img/pdf/setup/spect_accel}
    \caption[\imgsource{img/py/setup/vibration_spectroscopy.py}]{
        Top: displacement noise spectra acquired at room temperature when the \gls{ptr} is switched on (blue, magenta) or off (green and orange), and when the air suspension is switched enabled (magenta, orange) or disabled (blue, green).
        Bottom: band-limited \gls{rms} computed from the \glspl{psd} in the upper panel (\cf \cref{eq:speck:psd:bandpower}).
    }
    \label{fig:setup:vibrations:accelerometer}
\end{figure}

Since the sensor's (conditioned) output is a voltage directly proportional to the acceleration, it is straightforward to compute the displacement \gls{psd} from time series data measured with the \gls{dmm} using the \pyspeck package presented in \cref{ch:speck:software}~\cite{Hangleiter_pyspeck}.
Leveraging the \code{fourier_procfn} argument, I transformed the voltage data first to acceleration and then, by integration, to displacement in frequency space as indicated in \cref{lst:setup:vibrations:accelerometer}.
To assess the impact of the \gls{ptr} and the suspension, I measured the displacement noise \gls{psd} for each combination of the two being switched on and off.
The cryostat was closed, its vacuum chamber evacuated, and the magnet, a significant seismic mass, mounted as usual.
The measurements are shown in \cref{fig:setup:vibrations:accelerometer} together with the band-limited \gls{rms} (\cf \cref{eq:speck:psd:bandpower}),
\begin{equation}\label{eq:setup:vibrations:rms}
    \rms_S(f) = \sqrt{\int_{f_\mr{min}}^{f}\dd{f^\prime}S^2(f^\prime)}.
\end{equation}
When the \gls{ptr} is switched off, the spectra with and without suspension are dominated by broadband vibration noise, although quite some structure around \qtylist{15;33;60}{\hertz} can be observed.\sidenote{
    Note the curious peaks slightly offset from the second and third harmonic of the \gls{ptr} frequency in the spectrum with suspension enabled and \gls{ptr} disabled.
    We may speculate that these are due to the \glspl{ptr} of other cryostates in other labs in the vicinity that are transmitted through the floor.
    Two were running two rooms over at the time the data was acquired.
}
When it is switched on, the \gls{ptr} pulses at \qty{1.4}{\hertz} and a large number of its higher harmonics visually dominate the spectra.
Clearly, the suspension has a larger impact in this case, matching qualitatively the behavior discussed in \cref{sec:setup:vibrations:isolation}.
At high frequencies, it manages to almost completely suppress the broadband excitation observed without the suspension.
At low frequencies, on the other hand, the \gls{ptr} harmonics are amplified to the degree that the band-limited \gls{rms} is dominated by their contribution.
Only at around \qty{10}{\hertz}, the attenuation starts to take effect.
Overall, the \gls{ptr} is found to raise the displacement noise amplitude from \qty{0.5}{\micro\meter} \gls{rms} to \qty{10}{\micro\meter} \gls{rms}, while the suspension, over the entire frequency range, has at best no positive influence.

This result is less than encouraging.
At that level of \gls{rms}-fluctuations, we'd have a slim chance of resolving micrometer-scale features using the microscope.
But is the \emph{absolute} magnitude of displacement noise at the sample position really the correct measure for the microscope performance?
Indeed, if the sample oscillates in phase with the objective and ocular lenses as well as the \gls{smf}, we will still obtain a perfect imaging fidelity.
So actually only the \emph{relative} displacements of sample, lenses, and detection fiber affect the microscope's performance.
To characterize these, I developed an optical \emph{in-situ} technique to measure the displacement noise based on knife-edge reflectance fluctuations that I will present in the following section.
% TODO: gaussian beam diameter to first order in z and x
%  Spot size: sqrt(1 + (z/z_r)^2) = 1 + .5(z/z_r)^2
%  Radial: exp(-r^2/2) = 1 - .5r^2

\section{Optical vibration spectroscopy}\label{sec:setup:vibrations:optic}
The gate electrodes on our samples are fabricated using two separate lithography processes; first, the smallest structures are written using \gls{ebl} in two steps.
Then, larger structures on the order of \qty{1}{\micro\meter} and above are written using optical lithography.
In the region where the two overlap on the mesa to establish electrical contact, the highly reflective \ch{Ti/Au} optical gates have a width of \qty{14}{\micro\meter} and a height of \qty{160}{\nano\meter} and lie on top of the poorly reflecting \ch{GaAs} surface, resulting in a step-like reflectance profile.
Scanning perpendicularly across such a straight edge between a poorly and a highly reflecting material is known as a knife-edge measurement and is typically used to measure the spatial extent of a laser spot~\cite{Arnaud1971,Skinner1972,Khosrofian1983}.
We can use the same setup to measure the displacement noise; instead of manually shifting our knife edge across the beam spot, though, we measure the reflectance fluctuations induced by the knife edge fluctuating relative to the spot.

\begin{marginfigure}
    \centering
    \begin{tikzpicture}[
    round/.style={rounded corners=10pt},%
    font=\footnotesize,%
]

    % Mesa edge
    \draw[thick]
        (-2,0)
        -- ++(4.5,0)
    ;
    % Gate
    \path[draw=black, thick, fill=RWTHblack10, round]
        (-0.75,1)
        -- ++(0,-2.5)
        -- ++(1.5,0)
        -- ++(0,2.5)
    ;
    \path[draw=black, thick, dashed]
        (-0.75,1)
        -- ++(1.5,0)
    ;
    % Scan direction
    \draw[->, thick]
        (0.25,-0.75)
        -- ++(1,0) node[right] [align=center]{Scan \\ direction}
    ;
    % Laser spot
    \draw[dashed]
        (0.75,-0.75) circle (0.25)
    ;
    % Coordinate axes
    \draw[->]
        (1.75,0.3)
        -- ++(0.5,0) node[above] {$y$}
    ;
    \draw[->]
        (1.75,0.3)
        -- ++(0,0.5) node[right] {$z$}
    ;
    \draw
        (1.75,0.3) circle (0.1) node[left] {$x$}
    ;

    % Labels
    \node (mesa) at (-1.5,-0.25) {Mesa};
    \node (gate) at (0,0) [align=center]{Optical \\ gate};

\end{tikzpicture}
    \caption[\imgsource{img/tikz/setup/knife_edge.tex}]{
        Sketch of the region of the sample used for optical vibration spectroscopy.
        The coordinate system follows the magnet's; $z$ is parallel to gravity, and $x$ is perpendicular to the \gls{qw} plane.
        The optical gate extends further north as indicated by the dashed line.
    }
    \label{fig:setup:vibrations:knife_edge:sketch}
\end{marginfigure}

The scenario is sketched in \cref{fig:setup:vibrations:knife_edge:sketch} in the coordinate system defined by the magnet such that $z$ is along gravity's axis and $x$ is the out-of-plane axis.
Focusing the laser (indicated by a dashed circle) onto the edge of the optical gate, we can move the sample using the $y$-axis nanopositioner and observe a decrease in reflected intensity if the gate is moved away from the laser and an increase if it is moved towards the laser.
This gradient in reflected intensity can be inverted to obtain the vibration noise along $y$ by monitoring the intensity as a function of time.

Let us take a closer look at the reflected intensity when the laser spot has a finite overlap with the edge of the gate.
Under the simplifying assumption of a perfectly sharp drop-off and taking the reflectance of the gold gate to be unity, we can write the reflectance as function of the coordinate perpendicular to the gate edge at $y=0$ as
\begin{equation}\label{eq:setup:reflectance_step}
    R(y) = \begin{dcases}
        1, & y \leq 0 \\
        r, & y > 0,
    \end{dcases}
\end{equation}
where $r$ is the reflectance of the bare \ch{GaAs} surface.
Assuming a perfect Gaussian (\gls{temmode}$_{00}$ mode) beam characterized by its waist radius $w_0$ at which the intensity drops to $1/e^2$ of its maximum value, the laser intensity profile in 1D is given by
\begin{equation}\label{eq:setup:gaussian}
    I(y) = I_0\exp(-\frac{2y^2}{w_0^2}),
\end{equation}
where $I_0 = \flatfrac{P_0}{w_0}$ with $P_0$ the total beam power.
The power reflected when the spot partially overlaps with the reflectance step can then be expressed as the convolution
\begin{align}\label{eq:setup:knife_edge}
    P_R(y) &= R(y) \ast I(y) \\
           &= \frac{I_0 w_0}{2}\sqrt{\frac{\pi}{2}}\left[ 1 - (1 - r)\erf\left(\frac{y\sqrt{2}}{w_0}\right) \right]
\end{align}
in the $yz$ focal plane, where $\erf(y)$ is the error function.

\begin{marginfigure}
    \centering
    \includegraphics{img/pdf/setup/knife_edge_theory}
    \caption[\imgsource{img/py/setup/vibration_spectroscopy.py}]{
        Theoretical reflected power for a Gaussian beam of width $w_0$ and a reflectance contrast of $1-r$ according to \cref{eq:setup:knife_edge}.
        The dashed line indicates the leading order approximation at $y=0$.
    }
    \label{fig:setup:vibrations:knife_edge:theory}
\end{marginfigure}

The function is plotted in \Cref{fig:setup:vibrations:knife_edge:theory}.
The contrast that can be achieved is given by $1-r$.
Furthermore, for $y\in[-w_0/2, w_0/2]$ the function is well-approximated by
\begin{equation}\label{eq:setup:knife_edge:approx}
    P_R(y)\approx -I_0(1-r)y + \frac{I_0 w_0}{2}\sqrt{\frac{\pi}{2}}
\end{equation}
drawn as a dashed line.
Since we measure the photon count rate rather than the power, $\Phi = \flatfrac{P}{h\nu}$ with $\nu$ the laser wavelength, we rewrite this as
\begin{equation}\label{eq:setup:knife_edge:linearized}
    \Phi_R(y) = -sy + \frac{\Phi_0}{2}\sqrt{\frac{\pi}{2}}.%\Phi_{\mr{avg}},
\end{equation}
where we defined the \emph{sensitivity}
\begin{equation}\label{eq:setup:knife_edge:sensitivity}
    s = \frac{\Phi_0}{w_0}(1 - r).
\end{equation}
Hence, to obtain a more sensitive probe for vibrations, one could either improve the reflectance contrast $1-r$, decrease the spot size $w_0$, or increase the incident photon flux $\Phi_0$.\sidenote{
    Note that the smaller $w_0$, the smaller also the maximum displacement amplitude that can be resolved as the derivative goes to zero as $y\to w_0$.
}
In our case, the former two are fixed by the setup and sample, whereas the latter is limited by the maximum data transfer rate of the \tagger counting card, \qty{9}{\mega\sample\per\second}.

Starting from \cref{eq:setup:knife_edge:linearized}, it is straightforward to obtain the displacement in the vicinity of $y=0$ as function of photon flux,
\begin{equation}\label{eq:setup:knife_edge:procfn}
     y(\Phi_R) = \frac{w_0}{1-r}\left[\frac{1}{2}\sqrt{\frac{\pi}{2}} - \frac{\Phi_R}{\Phi_0}\right].
\end{equation}
To summarize, we can position the laser spot on the edge of an optical gate and record a time trace of the photon flux by using the \tagger to count the photons detected by the \glspl{apd} mounted on the side exit of the spectrometer.
Using \cref{eq:setup:knife_edge:procfn} we can then convert the flux into a displacement and proceed with the usual spectral noise estimation as explained in \cref{part:speck}.

\clearpage
\begin{marginfigure}
    \centering
    \includegraphics{img/pdf/setup/knife_edge}
    \caption[\imgsource{img/py/setup/vibration_spectroscopy.py}]{
        Calibration of the length reference scale.
        The top shows a \acrshort{cmos} camera image in (black corresponding to high intensity) of the white light spot on the edge of the optical gate as indicated in \cref{fig:setup:vibrations:knife_edge:sketch}.
        Several diffraction lines can be seen parallel to the edge.
        The vertical dashed lines indicate the region in which the intensity slope was fitted.
        The horizontal dashed lines indicate the extent of rows averaged over.
        The lower plots show a line cut along the central row of the considered region (top) and its derivative (bottom).
    }
    \label{fig:setup:vibrations:calibration:length_scale}
\end{marginfigure}

I will now lay out the experimental procedure of calibrating the system to (implicitly) obtain the parameters $w_0$, $r$, and $\Phi_0$.
The first challenge is obtaining a proper length reference scale.
While the nanopositioners on which the sample is mounted do in principle have a resistive position readout, it is extremely unreliable at small displacements.
Therefore, I calibrated the relative position using the imaging arm of the optical head.
\Cref{fig:setup:vibrations:calibration:length_scale} depicts the procedure.
I illuminated the sample with the white light, positioned the spot on the edge of the optical gate, and imaged the sample with the \gls{cmos} camera.\sidenote{
    Thorlabs DCC1545M with \qty{5.2}{\micro\meter} square pixels.
    % TODO: check Model no
}
I then extracted the position of the edge, in pixels, for several rows to obtain some statistics by fitting a linear function to the edge profile in a small region between two refraction maxima.
Repeating this step for different DC voltages applied to the nanopositioner,
%\sidenote{
%    The positioners operate in so-called slip-stick mode.
%    In physics terms, the scheme corresponds to alternating adiabatic and diabatic ramps.
%    Applying a DC voltage elongates the piezoelectric element by a small amount and moves the positioner table.
%    When quickly ramping the voltage back down, the table slips and stays in place while the piezo contracts back to its equilibrium elongation.
%    This way, distances much larger than the piezoelectric elongation can be travelled.
%    Staying in the adiabatic regime results in reproducible displacements but also limits the travel.
%}
this yields the proportionality factor between the nanopositioner DC voltage, $V_{\mr{DC}}$, and the position of the gate edge on the camera.
By measuring the total width of the gate on the camera image, I obtained the magnification by referencing it to the design width,
\begin{equation}\label{eq:setup:knife_edge:magnification}
    M = \frac{W[\unit{\pixel}]}{W[\unit{\micro\meter}]}.
\end{equation}
Again performing a linear fit (shown in \cref{fig:setup:vibrations:calibration:pos_vs_vdc}) to the data for different voltages then results in the linear transformation from DC voltage to position.

\begin{marginfigure}
    \centering
    \includegraphics{img/pdf/setup/knife_edge_fits}
    \caption[\imgsource{img/py/setup/vibration_spectroscopy.py}]{
        Linear fit of the edge positions extracted from the analysis in \cref{fig:setup:vibrations:calibration:length_scale} for different DC voltages applied to the nanopositioner.
        Error bars are propagated standard errors of the weighted average of edge positions extracted from different rows.
    }
    \label{fig:setup:vibrations:calibration:pos_vs_vdc}
\end{marginfigure}
\begin{marginfigure}
    \centering
    \includegraphics{img/pdf/setup/knife_edge_slope}
    \caption[\imgsource{img/py/setup/vibration_spectroscopy.py}]{
        Photon count rate as function of DC voltage applied to the nanopositioner.
        Fitting the region $V_{\mr{DC}}\in[0.5, 7]\,\unit{\volt}$ yields a slope of \qty{2.36+-0.02}{\mega\cps\per\micro\meter}.
        Error bars on the count rate show the standard error on the mean over multiple observations and error bars on the position show the fit error from \cref{fig:setup:vibrations:calibration:pos_vs_vdc}.
    }
    \label{fig:setup:vibrations:calibration:pos_vs_cps}
\end{marginfigure}

Finally, I switched from white light illumination to the laser, focused it onto the edge of a gate, and measured the photon count rate reflected off the sample as a function of $V_{\mr{DC}}$, from which we can finally extract the desired sensitivity (slope) of count rate over displacement.
The data and fit are shown in \cref{fig:setup:vibrations:calibration:pos_vs_cps}.
Clearly, the count rate is linear in the displacement over a large range, indicating that for fluctuations with amplitude on the order of \qty{100}{\nano\meter} \gls{rms}, the measurement sensitivity should be sufficiently robust.

\begin{figure}
    \centering
    \includegraphics{img/pdf/setup/spect_optic}
    \caption[\imgsource{img/py/setup/vibration_spectroscopy.py}]{}
    \label{fig:setup:vibrations:optical}
\end{figure}

\section{Routes for improvement}\label{sec:setup:vibrations:outlook}
Several improvements could be made to the system if the external conditions would allow it.
First, the rotary valve motor should be moved further away from the cold head.\sidenote{
    Clearly, this will impact the performance of the \gls{ptr} to some extent and should therefore be considered carefully.
}
It is currently\sidenote{
    As per the initial installation status.
}
connected to the cold head with a flexible hose at a right angle and a distance of roughly \qty{50}{\cm}, which is below the minimum bend radius recommended by \OI.\sidenote{
    Note that the orientation of the motor, which is horizontal with the axis, is also not the recommended configuration.
}
Additionally, the term \enquote{flexible} is relative here given the pressure of \qty{20}{\bar}.
Increasing the length of the hose should reduce its relative rigidity and thereby its ability to transmit vibrations from the motor to the cold head.

Next, the cold head should be mounted firmly to a secondary reference frame, for instance the ceiling or the lower cryostat frame on which the springs rest.
An intuitively obvious step, it has also been shown in the literature that decoupling the \gls{ptr} from the cryostat in this fashion leads to significant improvements in vibration isolation~\cite{Olivieri2017}.\sidenote{
    The former option was attempted, but showed no clear improvements in the measurements for reasons unclear, see \cref{ch:app:setup:vibrations} for additional data.
    It did emphatically deteriorate the inter-departmental atmosphere.
    Apologies to the institute on the floor above.
}
Acoustic insulation of the room and \gls{ptr} flex hoses could further improve the low-frequency response of the system~\cite{Schmoranzer2019,Oh2021}.
Lastly, let me note that there also exist cryocoolers with variable operating frequency that can thus be tuned away from problematic resonances in the system~\sidecite{TransMitPTR}.

%On the methodology side, a simple way to improve the sensitivity of the optical vibration spectroscopy method detailed in \cref{sec:setup:vibrations:optic} is to increase the laser power and therefore the count rate to raise the true vibration noise floor above the photon shot noise floor of \qty[power-half-as-sqrt]{1}{\cts\per\hertz\tothe{0.5}}.
% TODO: think about this, simulate spectrum

In \cref{ch:app:setup:vibrations}, I show additional spectroscopy data, including data measured along the gravitational axis in the puck and on the floor of different rooms, which suggests moving to a different laboratory could also benefit the vibration stability.

\cite{Caparrelli2006,Pelliccione2013}
