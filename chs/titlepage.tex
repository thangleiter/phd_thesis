% mainfile: ../main.tex

\newcommand{\thesisTitle}{\selectlanguage{english} Millikelvin Confocal Microscopy of Electrostatic Exciton Traps and Filter-Function Description of Noisy Quantum Dynamics}
\newcommand{\thesisDate}{\today}
\newcommand{\authorName}{Tobias Hangleiter}
\newcommand{\firstExaminer}{Prof\@. Dr\@. Hendrik Bluhm}
\newcommand{\secondExaminer}{Prof\@. Dr\@. Christoph Stampfer}

% For PDF
\author{\authorName}
\title{\thesisTitle}
\date{\thesisDate}

% We override the kaobook title page with RWTH's demands
\makeatletter
\renewcommand*{\maketitle}{%
    \begin{titlepage}
        \pdfbookmark[0]{Cover}{Cover}
        \selectlanguage{german}
        \sffamily
        \flushright
        \hfill
        \vfill
        \centering
        %\rule[5pt]{\textwidth}{.4pt} \par
        {\huge\thesisTitle\par}
        %\rule[5pt]{\textwidth}{.4pt} \par
        \vfill
        %Von der Fakultät für Mathematik, Informatik und Naturwissenschaften der RWTH Aachen University zur Erlangung des akademischen Grades eines Doktors der Naturwissenschaften genehmigte Dissertation \\ % not yet
        Der Fakultät für Mathematik, Informatik und Naturwissenschaften der RWTH Aachen University vorgelegte Dissertation zur Erlangung des akademischen Grades eines Doktors der Naturwissenschaften \\
        %\bigskip vorgelegt von \\
        \bigskip von \\
        \bigskip \authorName, M\@.Sc\@. \\
        \smallskip aus \\
        \smallskip Filderstadt

        %\begin{flushleft}
        %    \vspace{2\bigskipamount}
        %    {
        %        \small
        %        Berichter:~\firstExaminer \\
        %        \hphantom{Berichter:}~\secondExaminer    \\
        %        \medskip Tag der mündlichen Prüfung: TBD \\
        %        \medskip Diese Dissertation ist auf den Internetseiten der Universitätsbibliothek
        %        verfügbar.
        %    }
        %\end{flushleft}

    \end{titlepage}
    \ifx
        \@lowertitleback\@empty\else
        \clearpage
        \thispagestyle{empty}%
        \begingroup
        \setlength{\parindent}{0pt}%
        \ifx\@uppertitleback\@empty\else \@uppertitleback \par\bigskip\fi
        \vspace*{\fill}%
        \@lowertitleback
        \endgroup
        \clearpage
    \fi
}
\makeatother

%\uppertitleback{} % Header
\lowertitleback{
    {
        \small
        \begin{center}
            \noindent\includegraphics[width=0.2\textwidth]{img/pdf/logos/IQI_logo}
            \bigskip\\
            \textbf{JARA-Institute for Quantum Information}\\
            \bigskip
            an initiative of\\
            \bigskip
            \begin{tabular}{ m{.45\textwidth} m{.45\textwidth}}
                \textbf{RWTH Aachen University}                                                 & \textbf{Forschungszentrum Jülich GmbH} \\
                \textit{Quantum Technology Group}                                               & \textit{JARA-Institute for Quantum Information} \\
                Department of Physics                                                           & Peter-Grünberg-Institute (PGI-11) \\
                Otto-Blumenthal-Straße                                                          & Wilhelm-Johnen-Straße \\
                52070~Aachen                                                                    & 52428~Jülich \\
                \bigskip\noindent\includegraphics[height=1cm]{img/pdf/logos/rwth_qutech_en_rgb} & \bigskip\noindent\includegraphics[height=1cm]{img/pdf/logos/Logo_FZ_Juelich_cmyk}
            \end{tabular}
        \end{center}
    }

    \bigskip

    \textbf{Colophon} \\
    This document was typeset with the help of \href{https://sourceforge.net/projects/koma-script/}{\KOMAScript} and \href{https://www.latex-project.org/}{\LaTeX} using the \href{https://github.com/fmarotta/kaobook/}{kaobook} class. \\
    The source code of this thesis is available at: \url{https://github.com/thangleiter/phd_thesis}

    \medskip

    \textbf{\authorName} \\
    \textit{\thesisTitle} \\
    \thesisDate \\

    \smallskip
}

%----------------------------------------------------------------------------------------
%	OUTPUT TITLE PAGE AND PREVIOUS
%----------------------------------------------------------------------------------------

% Note that \maketitle outputs the pages before here

\frontmatter % Denotes the start of the pre-document content, uses roman numerals
\maketitle
