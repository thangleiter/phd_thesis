% mainfile: ../../main.tex
\chapter{The \mjolnir measurement framework}\label{ch:exp:mjolnir}
\AutoLettrine{Mjölnir}

\begin{marginfigure}[]
    \forestset{
    dir tree/.style={
        for tree={
            parent anchor=south west,
            child anchor=west,
            anchor=mid west,
            inner ysep=0pt,
            grow'=0,
            align=left,
            s sep=1ex,
            edge path={
                \noexpand\path [draw, \forestoption{edge}] (!u.parent anchor) ++(0.75em,0) |- (.child anchor)\forestoption{edge label};
            },
            font=\footnotesize\ttfamily,
            if n children=0{}{
                delay={
                    prepend={[,phantom, calign with current]}
                }
            },
            fit=band,
            before computing xy={
                l=1.25em
            }
        },
    }
}
\begin{forest}
    dir tree
    [
        [{\faIcon[regular]{folder} doc}]
        [{\faIcon[regular]{folder-open} src}
            [{\faIcon[regular]{folder-open} mjolnir}
                [{\faIcon[regular]{folder-open} config}
                    [{\faIcon[regular]{file-code} physical\_[\ldots].yaml}]
                    [{\faIcon[regular]{file-code} optical\_path.yaml}]
                    [{\faIcon[regular]{file-code} fig\_F10.yaml}]
                    [{\faIcon[regular]{file-code} \ldots}]
                ]
                [{\faIcon[regular]{folder-open} instruments}
                    [{\faIcon[regular]{file-code} \_\_init\_\_.py}]
                    [{\faIcon[regular]{file-code} logical\_[\ldots].py}]
                    [{\faIcon[regular]{file-code} physical\_[\ldots].py}]
                ]
                [{\faIcon[regular]{folder-open} measurements}
                    [{\faIcon[regular]{file-code} \_\_init\_\_.py}]
                    [{\faIcon[regular]{file-code} handler.py}]
                    [{\faIcon[regular]{file-code} measures.py}]
                    [{\faIcon[regular]{file-code} sweeps.py}]
                ]
                [{\faIcon[regular]{folder} parameters}
                    %[{\faIcon[regular]{file-code} \_\_init\_\_.py}]
                    %[{\faIcon[regular]{file-code} customized.py}]
                ]
                [{\faIcon[regular]{folder-open} plotting}
                    [{\faIcon[regular]{file-code} \_\_init\_\_.py}]
                    [{\faIcon[regular]{file-code} live\_view.py}]
                    [{\faIcon[regular]{file-code} plot\_nd.py}]
                ]
                [{\faIcon[regular]{file-code} \_\_init\_\_.py}]
                [{\faIcon[regular]{file-code} calibration.py}]
                %[{\faIcon[regular]{file-code} \_version.py}]
                %[{\faIcon[regular]{file-code} helpers.py}]
                [{\faIcon[regular]{file-code} \ldots}]
            ]
        ]
        [{\faIcon[regular]{file-code} main.py}]
        [{\faIcon[regular]{file-code} pyproject.toml}]
        [{\faIcon[regular]{file-code} \ldots}]
    ]
\end{forest}

    \caption[\imgsource{img/tikz/experiment/mjolnir_tree.tex}]{
        Source tree structure of the \mjolnir package.
        Logical \qcodes instruments and parameters are defined in the \code{instruments} and \code{parameters} modules, respectively.
        Instruments are configured using \code{yaml} files located in the \code{config} directory.
        The \code{measurements} module provides classes for the abstraction of measurements using \qcodes underneath.
        Live plots of instrument data as well as a plot function for multidimensional measurement data are defined in the \code{plotting} module.
        \code{calibration.py} contains routines for power, \acrshort{ccd}, and excitation rejection calibrations.
        The \code{main.py} file is a code cell-based script that serves as the entrypoint for measurements.
    }
    \label{fig:exp:mjolnir:tree}
\end{marginfigure}
