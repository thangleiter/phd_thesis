% mainfile: ../../main.tex
\chapter{Observations}\label{ch:exp:observations}
% Membrane measurements?
% PL measurements of doped membranes?
In this chapter, I present and discuss optical measurements of gated semiconductor membranes.
All of the samples under investigation here had the same nominal heterostructure layout; a \qty{20}{\nano\meter} \ch{GaAs} \gls{qw} sandwiched between two modulation-doped \qty{90}{\nano\meter} \ch{Al_{0.33}Ga_{0.67}As} barriers of which \qty{50}{\nano\meter} are an undoped spacer layer.
Together with a \qty{5}{\nano\meter} or \qty{10}{\nano\meter} \ch{GaAs} cap to on both sides to protect the Aluminium from oxidation, the membranes had a thickness of \qty{210}{\nano\meter} or \qty{220}{\nano\meter} after thinning.

Gate electrodes are patterned on the top and bottom side of the membrane using \gls{ebl}, where here and elsewhere in \thethesis I refer by \enquote{top} to gates on the etched and by \enquote{bottom} to those on the grown surface.
That is, \enquote{top} gates are on the air side when the sample is mounted in the fridge, whereas \enquote{bottom} gates are in contact with the epoxy glueing them to the Silicon host chip.
Typical designs have gates on the top and bottom sides of the membrane start at different positions on the mesa and converge towards the exciton trap site, where they overlap laterally.
For details about the fabrication process, refer to \citerr{Descamps2021}{Kindel2025}.

I installed the samples in the \gls{dr} introduced in \cref{part:setup} and cooled them to Millikelvin temperatures.
As detailed there, \gls{pl} measurements can be performed by illuminating the sample with a \gls{cw} laser above the band gap through an objective lens in front of the sample.
The emitted \gls{pl} radiation is picked up by the same lens and coupled into a \gls{smf} outside the fridge, from where it is sent to a diffraction grating spectrometer with a \gls{ccd} for spectral analysis.

In total, I show measurements from three different samples.
Unfortunately, none of the samples tested had a fully functional exciton trap with four working gates.
Several old samples fabricated during the work of \citet{Descamps2021} appeared to have aged, resulting in broken gates or otherwise unconventional behavior.
Newly fabricated samples on old heterostructure pieces frequently showed contact problems, either of optical gates to the mesa, between different \gls{ebl} resolution steps, or Ohmics, while other gates had leakage to ground or the \gls{qw}.

I pursued two different approaches to positioning the laser on the samples.
The first was using the white-light imaging arm of the confocal microscope (see \cref{part:setup}).
The larger gate structures on the samples typically show good contrast in the \gls{cmos} camera image, allowing orientating oneself following the sample design.
However, the simplicity of the optics\sidenote{
    The objective is just a singlet lens, compared to sophisticated commerical objectives containing a large number of optics inside to correct optical abberations.
    See for example the \href{https://www.attocube.com/en/products/microscopes/features/cryogenic-compatible-achromatic-high-na-objectives/lt-apo-lwd}{Attocube LT-APO/LWD}.
}
results in a comparably poor contrast for the smallest features on the scale below \qty{1}{\micro\meter}.
The magnification factor of the microscope is \num{30}, resulting in a feature size of \qty{160}{\nano\meter} per pixel on the camera.
In the best-case scenario, the cryostat vibration noise is on the order of \qty{100}{\nano\meter} \gls{rms} (\cref{sec:setup:vibrations:optic}), or roughly one pixel.
Resolving features only a few pixels in size is thus clearly on the edge of the microscope's capabilities.
Positioning the sample in this way is usually fairly reliable with some experience if one takes visual identification of the exciton trap gates as a target.
In practice, it is still necessary to fine-tune the position once the light source is switched to the laser because first, the focal distance of the objective lens shifts slightly when switching from a broadband to a monochromatic light source, and second, the focal spot of the laser is much smaller than that of the white light.\sidenote{
    The white light is launched from a \qty{400}{\micro\meter} diameter \gls{mmf} and collimated by a \num{0.13} \gls{na} lens, compared to a \qty{5}{\micro\meter} diameter \gls{smf} collimated by a \num{0.18} \gls{na} lens for the laser.
    % TODO: check mmf lens NA
}

The second approach is to roughly align the position on a large gate feature using the camera image, switch to laser illumination, and monitor the \gls{pl} signal when biasing the gate.
One can then move along the gate towards the exciton trap by following the \gls{pl} features expected for a gate: a reduced \gls{qw} emission due to absorption (see \cref{sec:exp:tmm}) and enhanced reflection of the laser line, as well as a Stark-shifted \gls{pl}.
This has the advantage that the functionality of the gate can be monitored.
Biasing also the other gates of the trap under investigation one can then look for additional Stark shifting of the \gls{pl}.
If the effect of each gate voltage can be observed, the position of the laser will be close to the trap center, and can then be optimized further.
While this is in a sense flying in the dark, it is a quite reliable method for an experienced experimentalist \emph{if} the gates are fully functional.

\section{Photoluminescence spectroscopy}\label{sec:exp:observations:pl}

\begin{marginfigure}
    \centering
    \includegraphics{img/pdf/experiment/2deg_pl}
    \caption[
        \sampleid{Doped M1_05_49-2}
        \thewavelength{795}
        \thepower{0.92}{\micro}
        \protect\newline
        \imgsource{img/py/experiment/pl.py}
    ]{
        \Gls{pl} of the bare \gls{2deg}.
        Magenta line is a smoothing spline fit to the data.
        Indicated by dotted gray lines are the Fermi edge at high and the band edge at low energy.
        The Fermi edge has a Fermi distribution (exponential indicated by a dashed gray line) whose temperature is typically much higher than the lattice temperature ($\sim\qty{1}{\kelvin}$).
        Below the band edge there is an exponential tail (dashed gray line) due to impurities that permeates far into the gap.
    }
    \label{fig:exp:pl:2deg}
\end{marginfigure}

\Cref{fig:exp:pl:2deg} shows a typical \gls{pl} spectrum obtained on the bare, unbiased \gls{qw} of a doped membrane sample.
This measurement corresponds to the configuration already discussed in \cref{sec:exp:theory:pl}.
Due to the Pauli exclusion principle, electrons require an energy of at least $E_\mr{F}\left(1 + m_{\mr{c}}^\ast/m_{\mr{hh}}^\ast\right)$ above the band gap and, because of the vanishingly small photon momentum, a momentum of at least $k_\mr{F}$ to be excited into a free state above the Fermi level $\mu$ (dotted gray line).\sidenote{
    This is known as the Burstein-Moss shift~\cite{Burstein1954,Moss1954}.
}
Once excited, they quickly relax down to the Fermi edge at $\mu$ from where they can recombine emitting a photon.
As the Fermi sea is at a finite temperature, the high-energy shoulder of the \gls{pl} spectrum is hence thermally broadened according to the Fermi distribution function of the electron gas (dashed gray line).
The associated temperature is around \qty{1.5}{\kelvin} and hence orders of magnitude higher than the lattice temperature of \qty{10}{\milli\kelvin}.
This effect has already been observed by \citet{Pinczuk1984}.
Like in those experiments, the temperature of the Fermi edge does not vary significantly with excitation power, making carrier heating due to high excitation power an unlikely cause~\cite{Ulbrich1973}.

\Gls{pl} emission is possible also at lower energies as electrons inside the Fermi sea recombine with the free photo-hole that scatters towards the valence band ege.
The band gap then defines the low-energy shoulder of the \gls{pl} spectrum, below which there are -- ideally -- no states available (dotted gray line).
However, the \gls{pl} reveals there are indeed free states decaying exponentially into the gap, originating most likely from impurities (dashed gray line).
Compared to the results of \citet{Kamburov2017}, the \gls{pl} spectra obtained here are much flatter over energy, with the \gls{pl} peak typically close to the middle between gap and Fermi edge.
Conversely, \citet{Kamburov2017} observed a strong peak at the band gap,\sidenote{
    \Cf also \citet{Gabbay2008}, who observe all but no \gls{fes} in samples nominally comparable to ours.
}
indicating that in our samples holes are more strongly localized and therefore have a wider spread in $k$-space, enabling transitions in a wider range of wave vectors~\cite{Skolnick1987}.
This would in turn imply increased alloy disorder or interface roughness~\cite{Gabbay2008}, an observation we shall come back to in \cref{sec:exp:observations:ple}.

From the width of the \gls{2deg} emission, we can calculate the charge carrier density by relating it to the Fermi energy in two dimensions~\cite{Pinczuk1984,Ihn2009},
\begin{equation}\label{eq:exp:pl:n}
    n = \frac{m^{\ast}_{\mr{c}}E_{\mr{F}}}{\pi\hbar^2} = \frac{\mu\Delta E}{\pi\hbar^2},
\end{equation}
where $\Delta E$ is the bandwidth of the emission (dashed gray lines in \cref{fig:exp:pl:2deg}) and $\mu = m^{\ast}_{\mr{c}}m^{\ast}_{\mr{hh}}/(m^{\ast}_{\mr{c}} + m^{\ast}_{\mr{hh}})$ is the reduced mass of conduction and valence band.
For this particular sample, \cref{eq:exp:pl:n} yields $n = \qty{5e11}{\per\square\centi\meter}$ or, equivalently, $E_{\mr{F}} = \qty{18}{\milli\electronvolt}$ and $k_{\mr{F}} = \qty{1.8e8}{\per\meter}$.
Comparing this value to that obtained from a simulation of the heterostructure with nominal doping concentration $N_{\mr{d}} = \qty{6.5e17}{\per\cubic\centi\meter}$ using a self-consistent Poisson-Schrödinger solver~\cite{PoissonSchroedinger}, $n = \qty{1.9e11}{\per\square\centi\meter}$, shows a significant discrepancy indicating a severe mismatch between nominal and actual doping concentrations.\sidenote{
    Note that the carrier density obtained thus does not vary significantly with excitation power, ruling out photo-doping as a source of the discrepancy.
    See \cref{ch:app:exp:observations} for a measurement of the \gls{2deg} \gls{pl} as function of power.
}
Finally, we observe that the gap according to the preceding analysis is redshifted from the undoped bulk gap of \qty{1.519}{\electronvolt}~\cite{Vurgaftman2001} by \qty{13}{\milli\electronvolt}.\sidenote{
    The redshift is in fact larger still due to the confinement energy of the \gls{qw}, estimated to be $\qty{17}{\milli\electronvolt}$ in \cref{sec:exp:theory:qcse}.
}
\Citet{Descamps2021} hypothesized that the removal of the \ch{GaAs} substrate and the associated change in strain leads to this lowering of the band gap.
However, this effect was also already observed by \citet{Pinczuk1984} in \enquote{bulk} modulation-doped \ch{GaAs} \glspl{qw}.
There, the authors put forth a renormalization of the band gap due to many-body interactions as an explanation.
Indeed, it is likely just bandgap narrowing due to doping~\cite{Jain1992}. % TODO: awkward

\subsection{Quantum-confined Stark shift}\label{subsec:exp:observations:pl:qcse}

\begin{figure}
    \centering
    \includegraphics{img/pdf/experiment/honey_H13_stark_shift_vs_gate}
    \caption[
        \sampleid{Honey H13}
        \thewavelength{795}
        \thepower{1}{\micro}
        \protect\newline
        \imgsource{img/py/experiment/pl.py}
    ]{
        \Gls{pl} as function of gate voltage on a single fan-out gate on the bottom (left) and top (right) side of the membrane.
        The behavior is qualitatively similar but the overall quantum efficiency lower by an order of magnitude for gates on the bottom (as-grown buried) side.
        Dotted gray lines are a guide to the eye demonstrating that the changes to the \gls{pl} spectrum set in at the same voltage for both types of gates (around \qty{-0.7}{\volt}).
    }
    \label{fig:exp:pl:honey_H13_stark_shift_vs_gate}
\end{figure}

Let us now address the behavior of the \gls{pl} under electric fields.
To this end, the laser is positioned on an \gls{ebl}-written gate and a negative voltage is applied to the gate.
\cref{fig:exp:pl:honey_H13_stark_shift_vs_gate} depicts measurements on a well-behaved sample.
The left (right) panel shows the \gls{pl} as function of voltage with the laser positioned on a bottom (top) gate.
In contrast to the intuition obtained in \cref{sec:exp:theory:qcse}, there is no immediate effect to be observed once the voltage is switched on.
This is most likely due to the presence of the \gls{2deg} screening the external electric field.
At $V_{\mr{gate}} = \qty{-0.7}{\volt}$, the high-energy shoulder of the emission starts to shift towards lower energies while the low-energy shoulder stays invariant.
Per the previous section, we can interpret this as the Fermi energy and thereby the charge carrier density being lowered as the \gls{2deg} is depleted.
As the electric field tilts the bands, the band edges of both conduction and valence band move in sync and therefore the band gap is not modified in this regime.\sidenote{
    We would in fact expect a slight increase in confimenent energy as the carrier density is lowered because the band bending due to the surplus electric charge of the \gls{2deg} is lifted.
}
As the \gls{2deg} is gradually depleted, a broad exciton peak emerges that shifts approximately quadratically with the applied voltage as ionization from the interaction with the Fermi sea of electrons becomes less (\cf \cref{sec:exp:theory:pl}).
Beyond \qty{-1.5}{\volt}, the \gls{2deg} is completely depleted.
The voltage difference between onset and completion of the depletion matches roughly the value expected from a simulation without screening for the nominal device parameters (see \cref{tab:app:exp:samples,tab:app:exp:samples:ps}), \qty{-0.7}{\volt}.

Besides the voltage dependence, another feature stands out from \cref{fig:exp:pl:honey_H13_stark_shift_vs_gate}: the \gls{pl} intensity is lower by an order of magnitude when on top of a back gate compared to a top gate.
This is at first puzzling, as in the latter case there is an additional semi-transparent gate\sidenote{
    \qty{7}{\nano\meter} \ch{Au} with a \qty{2}{\nano\meter} \ch{Ti} adhesion layer.
}
absorbing and reflecting both laser and \gls{pl} radiation, whereas in the former there is only the bare heterostructure, so we would expect the exact opposite!
I elucidate this issue in \cref{sec:exp:tmm}.

\begin{marginfigure}
    \centering
    \includegraphics{img/pdf/experiment/doped_M1_05_49-2_difference_mode}
    \caption[
        \sampleid{Doped M1_05_49-2}
        \thevoltage{-1.3}{CM}
        \thewavelength{795}
        \thepower{10}{\micro}
        \protect\newline
        \imgsource{img/py/experiment/pl.py}
    ]{
        \Gls{pl} as function of difference-mode voltage on a large exciton trap.
        The observed Stark shift follows roughly the expected quadratic dispersion, but is offset by \qty{0.75}{\volt} with respect to zero bias.
        Dashed gray line is a guide to the eye of a parabola with curvature \qty{-3.5}{\milli\electronvolt\per\volt\squared}.
        Line cuts in the upper panel are taken at the voltages indicated by dash-dotted lines in the lower.
    }
    \label{fig:exp:pl:doped_M1_05_49-2_difference_mode}
\end{marginfigure}

Moving to a large exciton trap with a single set of top and bottom gates,\sidenote{
    \qty{5}{\micro\meter} diameter.
}
we can measure the behavior of the Stark shift in the intended setting of local confining gates on either side of the membrane.
We define the virtual gates
\begin{align}\label{eq:exp:observations:virtual_gates}
    \VDM &= \VT - \VB, \\
    \VCM &= \VT + \VB,
\end{align}
where DM (CM) stands for difference (common) mode and T (B) for top and bottom.
Clearly, \VDM should ideally be proportional to the out-of-plane electric field across the membrane, $\VDM = Ft$ with $t$ the membrane thickness, whereas \VCM should tune the band edge offset from the Fermi level $\mu$.
As we observed previously, the presence of the \gls{2deg} screens the electric field generated by \VDM.
A good operating point is therefore at a negative \VCM which should deplete the \gls{2deg} (or at least reduce the charge carrier density).
\Cref{fig:exp:pl:doped_M1_05_49-2_difference_mode} shows a \gls{pl} map as function of the difference mode voltage \VDM at $\VCM = \qty{-1.3}{\volt}$.
From \cref{fig:exp:pl:honey_H13_stark_shift_vs_gate}, where the optical measurement of the charge carrier density results in a similar value as for the sample in \cref{fig:exp:pl:doped_M1_05_49-2_difference_mode}, $n\sim\qty{5e11}{\per\centi\meter\squared}$, we would expect this common mode voltage to suffice in at least overcoming the screening and reducing the carrier density in the \gls{qw}.
However, there is clearly still a \gls{2deg} emission present for a large range of \VDM, and the Fermi edge is at the same energy as without a gate bias.
Overall, the Stark shift pattern is symmetric but offset by $\VDM = \qty{0.75}{\volt}$.
This behavior is also observed in the response of the \gls{pl} to a single gate in this and several other samples, where the onset of an effect by bottom gate is significantly later than that for the top gate, suggesting that the voltage is screened by some mechanism.
Perhaps surprisingly, the gates on the \emph{bottom} side of the membrane display this behavior, \ie, the gates on the as-grown surface.\sidenote{
    I note that the sample of \cref{fig:exp:pl:honey_H13_stark_shift_vs_gate} was fabricated on the same heterostructure as the device in Figure 4(d) of \citer{Descamps2023}, which showed little to no electrical hysteresis unlike most other samples investigated by \citet{Descamps2021}.
}
This makes surface states an unlikely candidate for the screening as the quality of the grown surface should be better than that of the etched surface~\cite{Descamps2021}.
A possible cause might be oxygen segregation during growth\footnote{A.~Ludwig, private communication.} or other impurities~\cite{Nguyen2020}.

The fact that the \gls{2deg} is depleted by \VDM in the range \qtyrange{-0.2}{-1.4}{\volt} and above \qty{1.5}{\volt} at all is unexpected.
Even with an offset as just discussed, the symmetric dependence of the \gls{pl} emission on \VDM implies that the lever arms of both gates are comparable and hence $\VDM - \qty{0.75}{\volt}$ \emph{should} correspond to the out-of-plane electric field such that the energy is unchanged in the middle of the quantum well, $\VDM - \qty{0.75}{\volt} \propto F(z-t/2)$.
In and of itself, this parameter should not reduce the charge carrier density in an isolated \gls{qw} to first order.\sidenote{
    We might instead expect a small change in the apparent gap energy as the well is tilted.
}
One possible explanation for the observed behavior is the depletion of the \gls{2deg} by electrons tunneling out of the well and recombining with the donor ions, rendering one of the doped barriers electrically neutral.
With the doping pulling down the conduction band between \qty{50}{\nano\meter} and \qty{90}{\nano\meter} away from the \gls{qw}, the tunneling rate will be more pronounced than estimated for the undoped case in \cref{sec:exp:theory:qcse}.
Taking the large trap diameter in this case into account, there are $\sim\num[print-unity-mantissa=false]{1e5}$ electrons in the unbiased \gls{qw} on the area of the trap gates.
Thus, at a tunneling rate of \qty{1}{\mega\hertz}, all electrons would tunnel out of the \gls{qw} within \qty{100}{\milli\second} on average, putting a response of the system into the steady state within the fairly long time scales of a \gls{pl} measurement\sidenote{
    Say \qtyrange{0.1}{10}{\second}.
}
well within reasonable bounds. % TODO: phrasing
In the literature, this is known as carrier sweep-out and has been studied in the context of solar cells and other electroabsorptive devices~\cite{Larsson1988,Schneider1988,Fox1991}.

The dashed gray line in \cref{fig:exp:pl:doped_M1_05_49-2_difference_mode} shows a parabola as a guide to the eye.
For difference-mode voltages below \qty{-1.8}{\volt}, the exciton Stark shift follows this quite closely.
The upper panel depicts three line cuts at the positions indicated by the dash-dotted lines in the main panel.
As can be seen from the cut at \qty{-2.3}{\volt} (orange), the line shape appears to consist of distinct emission lines assigned by \citet{Descamps2021} as the neutral, singly, and doubly negatively charged excitons. % TODO: exciton/trion mixing ref
I return to the question of line assignment in \cref{sec:exp:observations:ple}.

\subsection{Power dependence}\label{subsec:exp:observations:pl:power}
\begin{figure}
    \centering
    \includegraphics{img/pdf/experiment/doped_M1_05_49-2_power}
    \caption[
        \sampleid{Doped M1_05_49-2}
        \thevoltage{-2.7}{DM}
        \thevoltage{-1.3}{CM}
        \thewavelength{795}
        \protect\newline
        \imgsource{img/py/experiment/pl.py}
    ]{
        \Gls{pl} as function of excitation power $P$.
        In the left panel two qualitatively different regimes are indicated by dashed gray lines as guides to the eye; below \qty{10}{\nano\watt}, the main peak displays a blueshift logarithmic in excitation power, $E = \qty{1.485}{\electronvolt} + \qty{5}{\milli\electronvolt}\log_{10} P$.
        Above, the blueshift diminishes significantly.
        Three additional lines, indicated by arrows, with varying power dispersion are visible.
        Right panel shows a fit to data at $P=\qty{1}{\micro\watt}$.
        A sum of seven individual lines (dashed, gray) is required to fit the data.
        The dashed gray lines are the individual contributions.
    }
    \label{fig:exp:pl:doped_M1_05_49-2_power}
\end{figure}

As outlined in \cref{sec:exp:theory:complexes}, the dependence of emitted \gls{pl} power on excitation power of individual emission lines can help inferring the excitonic species responsible.
Moreover, as the density of excitons depends on the excitation power, interaction effects between them influence the energy of the emission.
Such a measurement is shown in the left panel of \cref{fig:exp:pl:doped_M1_05_49-2_power} for the same common-mode voltage as in \cref{fig:exp:pl:doped_M1_05_49-2_difference_mode} and $\VDM = \qty{-2.7}{\volt}$ (\ie, corresponding to the lowest line in that plot) where the \gls{2deg} is completely depleted..
Two qualitatively different regimes can be observed as indicated by the dashed gray lines as guides to the eye.
Below \qty{10}{\nano\watt} excitation power, corresponding to approximately \qty{0.75}{\watt\per\square\centi\meter} at the beam diameter measured in \cref{part:setup}, the main line displays a blueshift logarithmic in excitation power.
Above this value, the blueshift is much less pronounced.
At powers above \qty{10}{\nano\watt}, two additional lines on the low-energy side of the main peak become visible that appear to converge towards higher powers, while above \qty{100}{\nano\watt} a line appears on the high-energy side that has a similar blueshift as the main peak at low powers.
In principle, the blueshift as function of power is to be expected.
Increasing the excitation power increases the density of excitons, either because they are spatially localized or confined or because of a finite diffusion speed.
The increased density corresponds to an increased overlap of single exciton's constituents and thus in turn leads to an increasingly repulsive Coulomb interaction that raises their energy.\sidenote{
    Of course, the opposite -- an attractive interaction -- is also possible for small numbers of excitons, resulting in biexcitons.
    The two cases can be thought of analagously as bonding and anti-bonding orbital configurations.
}
This mechanism underpins the use of exciton traps such as \glspl{saqd} as single-photon sources; upon spectrally filtering on the emission wavelength of a single exciton, only a single photon can be emitted from the trap at a time since the presence of another exciton in the trap would shift the emission energy of both. % TODO: move this somewhere else
It is not understood though why the blueshift abruptly changes in quality at \qty{10}{\nano\watt} excitation power.

The right panel shows a line cut taken at \qty{1}{\micro\watt}.
A weighted sum of seven individual Voigt profile line shapes~\cite{VoigtProfileWiki},
\begin{subequations}\label{eq:exp:voigt}
    \begin{equation}\tag{\ref{eq:exp:voigt}}
        V(E; \sigma, \gamma) = G(E; \sigma) \ast L(E; \gamma),
    \end{equation}
    that is, a convolution of Gaussian and Lorentzian line shapes given by
    \begin{align}
        G(x; \sigma) &= \frac{1}{\sigma\sqrt{2\pi}}\exp(-\frac{x^2}{2\sigma^2}), \label{eq:exp:gaussian} \\
        L(x; \gamma) &= \frac{1}{\pi}\frac{\gamma}{\gamma^2 + x^2}, \label{eq:exp:lorentzian}
    \end{align}
\end{subequations}
is required to obtain an adequate fit.
The Voigt profile arises from a combination of two separate line broadening mechanisms that manifest as a Gaussian (\cref{eq:exp:gaussian}) and Lorentzian (\cref{eq:exp:lorentzian}) line shape.
The former describes inhomogeneous broadening due to noise faster than the data acquisition time, whereas the latter is the homogeneous broadening due to for example the finite lifetime of the emitting state.\sidenote{
    This is also known as the linewidth's transform limit; energy and (life)time are Fourier pairs by Heisenberg's uncertainty principle.
}
All but the two outermost lines in the best fit are dominated by the inhomogeneous, Gaussian contribution to \cref{eq:exp:voigt} with widths $\sigma$ in the range of \qtyrange{0.1}{1}{\milli\electronvolt} and a peak separation on the order of \qty{2}{\milli\electronvolt}.
According to \citer{Descamps2021}, the lifetime of a Stark-shifted exciton is on the order of \qty{1}{\nano\second} due to the reduced wave function overlap.
This corresponds to a homogeneous linewidth of $2\gamma = \flatfrac{\hbar}{\tau} \sim \qty{660}{\nano\electronvolt}$, several orders of magnitude below the observed linewidth, and it is thus consistent with the fact that most peaks are best fit by inhomogeneously broadened line shapes.
The large a number of lines is certainly unexpected and cannot be explained by different excitonic species.
Where it is possible to track individual peaks as function of excitation power, their power dependence is linear, $\int\dd{E} V(E)\propto P$, suggesting neutral excitons or band-to-band recombination as origins of the emission.

\begin{figure*}
    \centering
    \includegraphics{img/pdf/experiment/doped_M1_05_49-2_multiplets}
    \caption[
        \sampleid{Doped M1_05_49-2}
        \thevoltage{0}{B}
        \protect\newline
        \imgsource{img/py/experiment/pl.py}
    ]{
        Wide-range \gls{pl} parameter sweep on a large exciton trap plotted as function of excitation power and detection energy.
        Rows are data for three different excitation wavelengths, columns for four different top gate voltages \VT and share color and line cut scales.
        Line cuts are taken at the indicated positions of $P_{\mathrm{det}} = \qtylist{7;16;35}{\nano\watt}$ and drawn scaled by the fraction of excitation power with respect to \qty{35}{\nano\watt}.
    % Atomic step fluctuations would shift by ~900 μeV (a = 5.6 Α)
    }
    \label{fig:exp:pl:doped_M1_05_49-2_multiplets}
\end{figure*}

A possible explanation for the multitude of lines is thus that they originate from different spatial locations, a hypothesis we return to in \cref{subsec:exp:observations:pl:spatial}.
First, let let us conclude this section with a large parameter sweep of this trap, shown in \cref{fig:exp:pl:doped_M1_05_49-2_multiplets}.
Each of the three rows (with separate panels for line cuts each) show data for different excitation wavelengths, $\lambda_{\mr{exc}}$, each of the columns show data for different top gate voltage, \VT, and the \gls{pl} is plotted as function of detection energy, $E_{\mr{det}}$, and excitation power, $P_{\mr{exc}}$.
The line cuts taken at lower powers (blue and green) are scaled to match the one at the highest power (orange) assuming a linear power dependence.
For all data $\VB = \qty{0}{\volt}$ so that $\VT = \VDM = \VCM$.
Despite the comparatively small changes in voltage, the behavior of the sample changes significantly.
Whereas for $\VT = \qty{-2.08}{\volt}$ the observed \gls{pl} features are similar to those in \cref{fig:app:exp:observations:2deg_pl_power_dependence},
at $\VT = \qty{-1.92}{\volt}$ there is a very large number of lines in the spectrum, most but not all of which share the same blueshift as function of excitation power.
The effect of a different excitation wavelength appears to mostly be a shift of the features along the excitation power axis and thus simply a change in absorption efficiency, although some features also change qualitatively.
I discuss the wavelength dependence in more detail in \cref{sec:exp:observations:ple}.

So what is the origin of the substructure of the \gls{pl} emission?
The peak distance is on the order of \qty{1}{\milli\electronvolt}.
This value matches fairly closely the order of magnitude of change in ground state energy, \qty{900}{\micro\electronvolt}, we would expect from fluctuations in the width of the \gls{qw} by one atomic layer of \ch{GaAs} with lattice constant $a = \qty{5.65}{\angstrom}$ for the design well width $L = \qty{20}{\nano\meter}$ (\cref{sec:exp:theory:pl}).

\subsection{Spatial dependence}\label{subsec:exp:observations:pl:spatial}
As mentioned several times already, the nanopositioners on which the sample is mounted show hysteresis and are therefore not suited for reproducible spatial maps of the sample.
The hysteresis is due to the non-adiabaticity of the method of movement in the so-called slip-stick mode (\cf \cref{ch:setup:cooling}).
What is more, the resistive position readout is also fairly unreliable below, say, \qty{5}{\micro\meter} resolution.
Nonetheless, we can at the very least perform simple one-dimensional sweeps after manually positioning the sample at a given starting position.
A more sophisticated algorithmic approach using feature detection may allow also two-dimensional maps with a reasonable accuracy.

\begin{marginfigure}
    \centering
    \includegraphics{img/pdf/experiment/fig_F10_positioning}
    \caption[
        \sampleid{Fig F10}
        \thewavelength{795}
        $V_{y}=\qty{30}{\volt}$
        \protect\newline
        \imgsource{img/py/experiment/pl.py}
    ]{
        \Gls{pl} of the unbiased \gls{qw} as the laser is stepped across a bottom gate.
        The line traces in the upper panel are taken at the positions indicated by dash-dotted lines.
        Positioner steps are converted to distance using a linear fit of the positioner readout after the initial hysteresis has worn off (about \num{10} steps).
        The Fermi edge shows a slight redshift when on top of the gate in this sample.
    }
    \label{fig:exp:pl:fig_F10_positioning}
\end{marginfigure}

\Cref{fig:exp:pl:fig_F10_positioning} shows the \gls{pl} collected from the sample as the positioner is stepped perpendicularly across an unbiased gate.
The vertical axis also gives a position coordinate which is computed from the steps by fitting the position readout once hysteresis has worn off.
The blue and green dash-dotted line correspond to the center of and beside the gate, respectively, with line cuts drawn in the upper panel.
Clearly, the \gls{pl} intensity is quenched significantly by the gate, beyond what one could expect from simple reflection and absorption of laser and \gls{pl} radiation.
I perform \gls{tmm} simulations to explain this behavior in \cref{sec:exp:tmm} (see also \cref{fig:exp:pl:honey_H13_stark_shift_vs_gate}).
Curiously, the map also shows a shifting of the Fermi edge close to the gate.
This behavior was consistently observed on this sample next to an unusual \gls{pl} line shape (\cf \cref{fig:exp:pl:2deg}).
The former effect could potentially be caused by band-deformation due to strain from the metallic gates.

\begin{figure}
    \centering
    \includegraphics{img/pdf/experiment/doped_M1_05_49-2_positioning}
    \caption[
        \sampleid{Doped M1_05_49-2}
        \thevoltage{-0.43}{DM}
        \thevoltage{-3.75}{CM}
        $V_{y}=V_{z}=\qty{30}{\volt}$
        \thepower{1}{\micro}
        \thewavelength{795}
        \protect\newline
        \imgsource{img/py/experiment/pl.py}
    ]{
        \Gls{pl} of a large exciton trap as function of position perpendicular (left) and parallel (right) to gravity.
        The trap is biased so that the emission is Stark-shifted towards lower energy.
        Upper panels show line cuts taken along the dash-dotted lines in the lower panels, demonstrating fairly identical behavior along both axes.
    }
    \label{fig:exp:pl:doped_M1_05_49-2_positioning}
\end{figure}

Next, I show two sweeps along orthogonal axes across the biased exciton trap discussed before in \cref{fig:exp:pl:doped_M1_05_49-2_difference_mode,fig:exp:pl:doped_M1_05_49-2_power,fig:exp:pl:doped_M1_05_49-2_multiplets,fig:exp:pl:doped_M1_05_49-2_ple}.
While this is not unequivocal proof of zero-dimensional confinement, it does suggest that the effective exciton potential is lowered in a laterally localized fashion. % TODO: me, pleeaase
The left panel of \cref{fig:exp:pl:doped_M1_05_49-2_positioning} shows a scan along the in-plane axis perpendicular to gravity, while the right shows the in-plane axis parallel and against gravity.
Regrettably, the resistive position readout did not yield anything but noise in these measurements, prohibiting a conversion of positioner steps into relative position as before.
Both scans were acquired with the sample voltage applied to the positioners (\qty{30}{\volt}), which is the same used in \cref{fig:exp:pl:fig_F10_positioning} as well.
We can hence roughly expect \num{10} steps to correspond to \qty{1}{\micro\meter} in $y$ direction.
Naturally, a single step against gravity displaces the sample by a smaller amount compared to the perpendicular direction, but given the circular shape of the trap and the similarity of the \gls{pl} features the displayed ranges should be roughly the same.
The upper panels show line cuts taken at the positions indicated by dash-dotted lines in the lower panels as usual.
Both sweeps display very similar features.
In the center of the trap, the Stark-shifted emission line consists of a single peak.
Towards the edges, two surprising effects take place: first, the Stark shift of the dominant peak increases in magnitude, and second, a large number of additional, faint lines appear.
The increase in effective electric field towards the edge of an exciton trap is reminiscent of the simulation of the same type of experiment for an undoped \gls{qw} by \citet[Figure~6.4]{Descamps2021}, although it was not explained there.
Indeed, it also appears to be in conflict with Figure~2.16 \ibid, where a monotonic increase in effective exciton potential as function of distance from the trap center closer to our intuition was predicted.
In the device under study here, there is of course a \gls{2deg} whose presence towards the edge of the trap screens the voltages, in theory contributing to an attenuation of the Stark shift.
Why we observe to opposite is thus quite puzzling.

The second surprising feature seen in \cref{fig:exp:pl:doped_M1_05_49-2_positioning} is the addition of faint lines that appear to branch out from the main exciton line as the laser spot is moved away from the center of the trap.
These could be related to the lines observed in \cref{fig:exp:pl:doped_M1_05_49-2_multiplets}.

% TODO: ...

\section{Photoluminescence excitation spectroscopy}\label{sec:exp:observations:ple}
\cite{Huard2000,Yusa2000}
\cite{Esser2000,Esser2001}

\begin{figure}
    \centering
    \includegraphics{img/pdf/experiment/doped_M1_05_49-2_ple}
    \caption[
        \sampleid{Doped M1_05_49-2}
        \thevoltage{-1.3}{CM}
        \thepower{1}{\micro}
        \protect\newline
        \imgsource{img/py/experiment/ple.py}
    ]{
        \Gls{pl} (solid lines) and \gls{ple} (dashed lines) for different voltages \VDM (\cf \cref{fig:exp:pl:doped_M1_05_49-2_difference_mode}).
        The \gls{ple} data points correspond to \gls{pl} spectra integrated up to the laser line.
        Arrows indicate the Stokes shift $\Delta E_{\mr{S}}$, which is approximately constant (although assignment of the gap peak is difficult for \qtylist{1.57;0}{\volt}) until the \gls{2deg} is fully depleted and the exciton resonance shifts quadratically with the electric field.
        The features at \qty{1.555}{\electronvolt} do not shift with the voltage and are thus likely unrelated to the trap.
        For the largest voltage, there is another peak at \qty{1.51}{\electronvolt} whose origin is unclear.
    }
    \label{fig:exp:pl:doped_M1_05_49-2_ple}
\end{figure}

\clearpage

\section{Transfer-matrix method simulations of the membrane structure}\label{sec:exp:tmm}
The \gls{tmm} is a computationally efficient method of obtaining the electric field in layered structures.
In this section, I perform simulations of the heterostructure membranes investigated in \thispart using the \pymoosh package~\cite{Langevin2024} to elucidate the observed quenching of \gls{pl} when illuminating gate electrodes as well as the overall optical efficiency.\sidenote{
    Strictly speaking, the term \acrshort{tmm} only refers to one of the several formalisms implemented in the \pymoosh package.
    While fast, it is not the most numerically stable, and other methods may be preferred if wall time is not a limiting issue.
}
I will first briefly recap the simulation method following \citer{Langevin2024}.
For more details, refer to \ibid and references therein.

Consider a layered structure along $z$ with interfaces at $z_i, i\in\lbrace 0, 1, \dotsc, N+1\rbrace$ that is translationally invariant along $x$ and $y$.
Each layer $i$ may consist of a different dielectric material characterized by a (complex) relative permittivity $\epsilon_{r,i}$.\sidenote{
    We disregard magnetic materials with relative permeability $\mu_r\neq 1$ for simplicity.
}
The electric field component along $y$ of an electromagnetic wave \gls{te} mode originating in some far away point satisfies the Helmholtz equation
\begin{equation}\label{eq:exp:tmm:helmholtz}
    \pdv[2]{E_y}{z} + \gamma_i^2 E_y = 0,
\end{equation}
where $\gamma_i = \sqrt{\epsilon_{r,i}k_0^2 - k_x^2}$ with $k_0=\flatfrac{\omega}{c}$ the wave vector in vacuum and $k_x$ the component along $x$.
In layer $i$ of the structure, the solution to \cref{eq:exp:tmm:helmholtz} may be written as a superposition of plane waves incident and reflected on the lower and upper interfaces~\cite{Langevin2024},
\begin{equation}\label{eq:exp:tmm:fields}
    \begin{dcases}
        E_{y,i}(z) = A_i^{+}\exp{\i\gamma_i[z-z_{i}]} + B_i^{+}\exp{-\i\gamma_i[z-z_{i}]}, \\
        E_{y,i}(z) = A_i^{-}\exp{\i\gamma_i[z-z_{i+1}]} + B_i^{-}\exp{-\i\gamma_i[z-z_{i+1}]},
    \end{dcases}
\end{equation}
where the coefficients with superscript $+$ ($-$) are referenced to the phase at the upper (lower) interface, respectively.
Matching these solutions at $z=z_i$ for all $i$ to satisfy the interface conditions imposed by Maxwell's equations gives rise to a linear system of equations, the solution to which can be obtained through several different methods.

A particularly simple method is the \acrlong{tmm} ($T$-matrix formalism), which corresponds to writing the interface conditions at $z=z_i$ as the matrix equation
\begin{equation}\label{eq:exp:tmm:interface}
    \pmqty{A_{i+1}^{+}\\B_{i+1}^{+}} = T_{i,i+1}\pmqty{A_{i}^{-}\\B_{i}^{-}}
\end{equation}
with
\begin{equation}\label{eq:exp:tmm:T}
    T_{i,i+1} = \frac{1}{2\gamma_{i+1}}\begin{pmatrix}
                                           \gamma_{i} + \gamma_{i+1} & \gamma_{i} - \gamma_{i+1} \\
                                           \gamma_{i} - \gamma_{i+1} & \gamma_{i} + \gamma_{i+1}
    \end{pmatrix}
\end{equation}
the transfer matrix for interface $i$.
Connecting the coefficients for adjacent interfaces within a layer of height $h_i = z_{i+1} - z_{i}$ requires propagating the phase,
\begin{equation}\label{eq:exp:tmm:propagation}
    \pmqty{A_{i}^{-}\\B_{i}^{-}} = C_{i}\pmqty{A_{i}^{+}\\B_{i}^{+}},
\end{equation}
with
\begin{equation}\label{eq:exp:tmm:C}
    C_{i} = \exp\left\lbrace\diag(-\i\gamma_i h_i, \i\gamma_i h_i)\right\rbrace.
\end{equation}
Iterating \cref{eq:exp:tmm:C,eq:exp:tmm:T}, the total transfer matrix $T = T_{0,N+1}$ then reduces to the matrix product
\begin{equation}\label{eq:exp:tmm:T:total}
    T = T_{N,N+1}\prod_{i=0}^{N-1} T_{i,i+1} C_i.
\end{equation}
From $T$, the reflection and transmission coefficients can be obtained as $r=A_0^{-}=-\flatfrac{T_{01}}{T_{00}}$ and $t=B_{N+1}^{+}=rT_{10} + T_{11}$.
Taking the absolute value square of reflection and transmission coefficients then yields the reflectance \reflectance and the transmittance \transmittance, which correspond to the fraction of total incident power being reflected and transmitted, respectively.
To obtain the absorptance \absorptance, the fraction of power being absorbed, in layer $i$, one can compute the difference of the $z$-components of the Poynting vectors (\cf \cref{eq:setup:optics:coupling:poynting}) at the top of layers $i$ and $i+1$.
In the \gls{te} case considered here, \cref{eq:setup:optics:coupling:poynting} reduces to~\cite{Langevin2024}
\begin{equation}\label{eq:exp:tmm:poynting}
    \bvec{S}_i = \re\left[\frac{\gamma_i^{\ast}}{\gamma_0}\left(A_i^{+} - B_i^{+}\right)^{\ast}\left(A_i^{+} + B_i^{+}\right)\right]
\end{equation}
and is hence straightforward to extract from the calculation of either the $S$ or $T$ matrices.

\Cref{eq:exp:tmm:T:total} is simple to evaluate on a computer, making this method attractive for numerical applications.
However, the opposite signs in the argument of the exponentials in \cref{eq:exp:tmm:C} can lead to instabilities for evanescent waves ($\gamma_i\in\mathbb{C}$) due to finite-precision floating point arithmetic~\cite{Duetz}.
Rewriting \cref{eq:exp:tmm:T} to have incoming and outgoing fields on opposite sides of the equality alleviates this issue while sacrificing the simple matrix-multiplication composition rule in what is known as the scattering matrix ($S$-matrix) formalism.
Finally, note that for a thorough accounting of in- and out-going field amplitudes, excitonic effects should be included, for example using the approach by \citet{DAndrea1990}.

Beyond the calculation of the aforementioned coefficients, the \gls{tmm} formalism also allows to compute the full spatial dependence of the fields.
Two cases are implemented in \pymoosh: irradiation of the layered structured with a Gaussian beam rather than plane waves of infinite extent, and a current line source inside the structure.
In the first case, the previously assumed translational invariance along $x$ leading to a plane-wave spatial dependence is replaced by a superposition of plane waves weighted with a normally distributed amplitude,\sidenote{
    \Ie, the inverse Fourier transform of $\mc{E}_0(k_x) E_{y,i}(k_x, z)$.
}
\begin{equation}\label{eq:exp:tmm:gauss:x}
    E_{y,i}(x,z) = \exp(\i k_x x)\rightarrow \int\ddf{k_x}\mc{E}_0(k_x) E_{y,i}(k_x, z)\exp(\i k_x x),
\end{equation}
with (\cf \cref{eq:setup:gaussian})
\begin{equation}\label{eq:exp:tmm:gauss:ampl}
    \mc{E}_0(k_x) = \frac{w_0}{2\sqrt{\pi}}\exp\left\lbrace - \i k_x x_0 -\left[\frac{w_0 k_x}{2}\right]^2\right\rbrace
\end{equation}
and
\begin{equation}\label{eq:exp:tmm:gauss:z}
    E_{y,i}(k_x, z) = A_{i}^{-}\exp\lbrace\i\gamma_i(k_x)[z-z_{i+1}]\rbrace + B_{i}^{+}\exp\lbrace -\i\gamma_i(k_x)[z-z_{i}]\rbrace,
\end{equation}
and where we considered only normal incidence for simplicity.

In the second case, \citet{Langevin2024} consider an AC current $I$ flowing through a translationally invariant, one-dimensional wire along $y$ at $x=x_{\mr{s}}$.
This introduces a source term into the Helmholtz equation, \cref{eq:exp:tmm:helmholtz}, which, upon Fourier transforming in $x$ direction, leads to
\begin{equation}\label{eq:exp:tmm:helmholtz:green}
    \pdv[2]{\hat{E}_y}{z} + \gamma_i^2\hat{E}_y = -\i\omega\mu_0 I\delta(z)\exp(\i k_x x_{\mr{s}}).
\end{equation}
The electric field $\hat{E}_{y,i}(k_x, z)$ is thus proportional to the Green's function of \cref{eq:exp:tmm:helmholtz:green} and can be obtained using a similar procedure as in the case of a distant source incident on the structure by matching the interface conditions.
Performing the inverse Fourier transform by means of \cref{eq:exp:tmm:gauss:x} with constant weights, $\mc{E}_0(k_x)\equiv 1$, then yields the two-dimensional spatial distribution of the electric field, $E_{y,i}(x, z)$.

\begin{figure}
    \centering
    \includegraphics{img/pdf/experiment/tmm_field}
    \caption[\imgsource{img/py/experiment/tmm.py}]{
        Absolute value of the electric field inside the double-gated heterostructure under illumination with a Gaussian beam at $\lambda=\qty{825}{\nano\meter}$ from the top.
        Top (bottom) panels show the structure with the default (optimized) barrier thickness of \qty{90}{\nano\meter} (\qty{122}{\nano\meter}), respectively.
        Dotted horizontal lines indicate interfaces between different materials while the vertical dash-dotted line indicates the position of the line cuts shown in the left column.
        Increasing the thickness of the barrier has two beneficial effects; first, the overall field intensity inside the structure is higher by a factor of two, and second, there is a peak rather than a knot in the \gls{qw} at a depth of $\sim\qty{120}{\nano\meter}$ ($\sim\qty{150}{\nano\meter}$), leading to enhanced absorption.
    }
    \label{fig:exp:tmm:field}
\end{figure}

\begin{margintable}
    \centering
    \footnotesize
    \caption{
        Absorptance $\mathscr{A}$ and reflectance $\mathscr{R}$ in the \gls{qw} for different configurations of the heterostructure.
        \enquote{Bare} is the standard structure without gate electrodes.
        \enquote{TG} and \enquote{BG} are with a gate on either the top or bottom side.
        \enquote{TG+BG} is with gates on both sides as on a trap site.
    }
    \label{tab:exp:tmm:absorptance_reflectance}
    % This table is generated by img/py/experiment/tmm.py
\begin{tabular}{lrr}
\toprule
 & $\mathcal{A}$ & $\mathcal{R}$ \\
\midrule
Bare & 0.029 & 0.22 \\
TG & 0.018 & 0.42 \\
BG & 0.005 & 0.83 \\
TGBG & 0.0041 & 0.85 \\
\bottomrule
\end{tabular}

\end{margintable}

\begin{marginfigure}
    \centering
    \includegraphics{img/pdf/experiment/tmm_absorptance}
    \caption[\imgsource{img/py/experiment/tmm.py}]{
        \Gls{qw} absorptance \absorptance in a heterostructure with default (blue) and optimized (magenta) barrier thickness and top and bottom gates as function of wavelength.
        Optimization was performed at \qty{825}{\nano\meter} using the differential evolution algorithm implemented in \pymoosh, resulting in a barrier thickness of \qty{122}{\nano\meter} and an absorptance better by a factor of \num{16} at \qty{6.3}{\percent}.
    }
    \label{fig:exp:tmm:wavelengths}
\end{marginfigure}
\clearpage
\begin{marginfigure}
    \centering
    \includegraphics{img/pdf/experiment/tmm_green}
    \caption[\imgsource{img/py/experiment/tmm.py}]{
        Real part of the electric field emitted by a current line located in the \gls{qw} (black point) for different cases of the unoptimized structure.
        From top to bottom: bare heterostructure, top gate, bottom gate, top and bottom gate.
        The half space $z<0$ is the air above the membrane in the direction of the objective lens and the dotted lines indicate interfaces between materials.
        Evidently, the bottom gate reduces the amplitude in the upper half of the membrane and thereby the outcoupling efficiency compared to the structures with just a top gate, consistent with what is observed in the experiment.
    }
    \label{fig:exp:tmm:green}
\end{marginfigure}

\begin{marginfigure}
    \centering
    \includegraphics{img/pdf/experiment/tmm_green_opt_tgbg}
    \caption[\imgsource{img/py/experiment/tmm.py}]{
        Real part of the electric field emitted by a current line located in the \gls{qw} (black point) for the default (top) and optimized (bottom) structures with top and bottom gates.
        Optimizing the barrier thickness for absorption in the \gls{qw} evidently also drastically improves the outcoupling efficiency into the halfspace $z<0$.
    }
    \label{fig:exp:tmm:green:opt:tgbg}
\end{marginfigure}
