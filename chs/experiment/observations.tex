% mainfile: ../../main.tex
\chapter{Observations}\label{ch:exp:observations}
\AutoLettrine{Pidgeons}
\section{Transfer-matrix method simulations of the membrane structure}\label{sec:exp:tmm}
The \gls{tmm} is a computationally efficient method of obtaining the electric field in layered structures.
In this section, I perform simulations of the heterostructure membranes investigated in \thispart using the \pymoosh package~\cite{Langevin2024} to elucidate the observed quenching of \gls{pl} when illuminating gate electrodes as well as the overall optical efficiency.\sidenote{
    Strictly speaking, the term \acrshort{tmm} only refers to one of the several formalisms implemented in the \pymoosh package.
    While fast, it not the most numerically stable, and other methods may be preferred if wall time is not a limiting issue.
}
I will first briefly recap the simulation method following \citer{Langevin2024}.
For more details, refer to \ibid and references therein.

Consider a layered structure along $z$ with interfaces at $z_i, i\in\lbrace 0, 1, \dotsc, N+1\rbrace$ that is translationally invariant along $x$ and $y$.
Each layer $i$ may consist of a different dielectric material characterized by a (complex) relative permittivity $\epsilon_{r,i}$.\sidenote{
    We disregard magnetic materials with relative permeability $\mu_r\neq 1$ for simplicity.
}
The electric field component along $y$ of an electromagnetic wave \gls{te} mode originating in some far away point satisfies the Helmholtz equation
\begin{equation}\label{eq:exp:tmm:helmholtz}
    \pdv[2]{E_y}{z} + \gamma_i^2 E_y = 0,
\end{equation}
where $\gamma_i = \sqrt{\epsilon_{r,i}k_0^2 - k_x^2}$ with $k_0=\flatfrac{\omega}{c}$ the wave vector in vacuum and $k_x$ the component along $x$.
In layer $i$ of the structure, the solution to \cref{eq:exp:tmm:helmholtz} may be written as a superposition of plane waves incident and reflected on the lower and upper interfaces~\cite{Langevin2024},
\begin{equation}\label{eq:exp:tmm:fields}
    \begin{dcases}
        E_{y,i}(z) = A_i^{+}\exp{\i\gamma_i[z-z_{i}]} + B_i^{+}\exp{-\i\gamma_i[z-z_{i}]}, \\
        E_{y,i}(z) = A_i^{-}\exp{\i\gamma_i[z-z_{i+1}]} + B_i^{-}\exp{-\i\gamma_i[z-z_{i+1}]},
    \end{dcases}
\end{equation}
where the coefficients with superscript $+$ ($-$) are referenced to the phase at the upper (lower) interface, respectively.
Matching these solutions at $z=z_i$ for all $i$ to satisfy the interface conditions imposed by Maxwell's equations gives rise to a linear system of equations, the solution to which can be obtained through several different methods.

A particularly simple method is the \acrlong{tmm} ($T$-matrix formalism), which corresponds to writing the interface conditions at $z=z_i$ as the matrix equation
\begin{equation}\label{eq:exp:tmm:interface}
    \pmqty{A_{i+1}^{+}\\B_{i+1}^{+}} = T_{i,i+1}\pmqty{A_{i}^{-}\\B_{i}^{-}}
\end{equation}
with
\begin{equation}\label{eq:exp:tmm:T}
    T_{i,i+1} = \frac{1}{2\gamma_{i+1}}\begin{pmatrix}
        \gamma_{i} + \gamma_{i+1} & \gamma_{i} - \gamma_{i+1} \\
        \gamma_{i} - \gamma_{i+1} & \gamma_{i} + \gamma_{i+1}
    \end{pmatrix}
\end{equation}
the transfer matrix for interface $i$.
Connecting the coefficients for adjacent interfaces within a layer of height $h_i = z_{i+1} - z_{i}$ requires propagating the phase,
\begin{equation}\label{eq:exp:tmm:propagation}
    \pmqty{A_{i}^{-}\\B_{i}^{-}} = C_{i}\pmqty{A_{i}^{+}\\B_{i}^{+}},
\end{equation}
with
\begin{equation}\label{eq:exp:tmm:C}
    C_{i} = \exp\left\lbrace\diag(-\i\gamma_i h_i, \i\gamma_i h_i)\right\rbrace.
\end{equation}
Iterating \cref{eq:exp:tmm:C,eq:exp:tmm:T}, the total transfer matrix $T = T_{0,N+1}$ then reduces to the matrix product
\begin{equation}\label{eq:exp:tmm:T:total}
    T = T_{N,N+1}\prod_{i=0}^{N-1} T_{i,i+1} C_i.
\end{equation}
From $T$, the reflection and transmission coefficients can be obtained as $r=A_0^{-}=-\flatfrac{T_{01}}{T_{00}}$ and $t=B_{N+1}^{+}=rT_{10} + T_{11}$.
Taking the absolute value square of reflection and transmission coefficients then yields the reflectance \reflectance and the transmittance \transmittance, which correspond to the fraction of total incident power being reflected and transmitted, respectively.
To obtain the absorptance \absorptance, the fraction of power being absorbed, in layer $i$, one can compute the difference of the $z$-components of the Poynting vectors (\cf \cref{eq:setup:optics:coupling:poynting}) at the top of layers $i$ and $i+1$.
In the \gls{te} case considered here, \cref{eq:setup:optics:coupling:poynting} reduces to~\cite{Langevin2024}
\begin{equation}\label{eq:exp:tmm:poynting}
    \bvec{S}_i = \re\left[\frac{\gamma_i^{\ast}}{\gamma_0}\left(A_i^{+} - B_i^{+}\right)^{\ast}\left(A_i^{+} + B_i^{+}\right)\right]
\end{equation}
and is hence straightforward to extract from the calculation of either the $S$ or $T$ matrices.

\Cref{eq:exp:tmm:T:total} is simple to evaluate on a computer, making this method attractive for numerical applications.
However, the opposite signs in the argument of the exponentials in \cref{eq:exp:tmm:C} can lead to instabilities for evanescent waves ($\gamma_i\in\mathbb{C}$) due to finite-precision floating point arithmetic~\cite{Duetz}.
Rewriting \cref{eq:exp:tmm:T} to have incoming and outgoing fields on opposite sides of the equality alleviates this issue while sacrificing the simple matrix-multiplication composition rule in what is known as the scattering matrix ($S$-matrix) formalism.

Beyond the calculation of the aforementioned coefficients, the \gls{tmm} formalism also allows to compute the full spatial dependence of the fields.
Two cases are implemented in \pymoosh: irradiation of the layered structured with a Gaussian beam rather than plane waves of infinite extent, and a current line source inside the structure.
In the first case, the previously assumed translational invariance along $x$ leading to a plane-wave spatial dependence is replaced by a superposition of plane waves weighted with a normally distributed amplitude,\sidenote{
    \Ie, the inverse Fourier transform of $\mc{E}_0(k_x) E_{y,i}(k_x, z)$.
}
\begin{equation}\label{eq:exp:tmm:gauss:x}
    E_{y,i}(x,z) = \exp(\i k_x x)\rightarrow \int\ddf{k_x}\mc{E}_0(k_x) E_{y,i}(k_x, z)\exp(\i k_x x),
\end{equation}
with (\cf \cref{eq:setup:gaussian})
\begin{equation}\label{eq:exp:tmm:gauss:ampl}
    \mc{E}_0(k_x) = \frac{w_0}{2\sqrt{\pi}}\exp\left\lbrace - \i k_x x_0 -\left[\frac{w_0 k_x}{2}\right]^2\right\rbrace
\end{equation}
and
\begin{equation}\label{eq:exp:tmm:gauss:z}
    E_{y,i}(k_x, z) = A_{i}^{-}\exp\lbrace\i\gamma_i(k_x)[z-z_{i+1}]\rbrace + B_{i}^{+}\exp\lbrace -\i\gamma_i(k_x)[z-z_{i}]\rbrace,
\end{equation}
and where we considered only normal incidence for simplicity.

In the second case, \citet{Langevin2024} consider an AC current $I$ flowing through a translationally invariant, one-dimensional wire along $y$ at $x=x_{\mr{s}}$.
This introduces a source term into the Helmholtz equation \cref{eq:exp:tmm:helmholtz} which, upon Fourier transforming in $x$ direction, leads to
\begin{equation}\label{eq:exp:tmm:helmholtz:green}
    \pdv[2]{\hat{E}_y}{z} + \gamma_i^2\hat{E}_y = -\i\omega\mu_0 I\delta(z)\exp(\i k_x x_{\mr{s}}).
\end{equation}
The electric field $\hat{E}_{y,i}(k_x, z)$ is thus proportional to the Green's function of \cref{eq:exp:tmm:helmholtz:green} and can be obtained using a similar procedure as in the case of a distant source incident on the structure by matching the interface conditions.
Performing the inverse Fourier transform by means of \cref{eq:exp:tmm:gauss:x} with constant weights, $\mc{E}_0(k_x)\equiv 1$, then yields the two-dimensional spatial distribution of the electric field, $E_{y,i}(x, z)$.

\begin{figure}
    \centering
    \includegraphics{img/pdf/experiment/tmm_field}
    \caption[\imgsource{img/py/experiment/tmm.py}]{
        Absolute value of the electric field inside the double-gated heterostructure under illumination with a Gaussian beam at $\lambda=\qty{825}{\nano\meter}$ from the top.
        Top (bottom) panels show the structure with the default (optimized) barrier thickness of \qty{90}{\nano\meter} (\qty{122}{\nano\meter}), respectively.
        Dotted horizontal lines indicate interfaces between different materials while the vertical dash-dotted line indicates the position of the line cuts shown in the left column.
        Increasing the thickness of the barrier has two beneficial effects; first, the overall field intensity inside the structure is higher by a factor of two, and second, there is a peak rather than a knot in the \gls{qw} at a depth of $\sim\qty{120}{\nano\meter}$ ($\sim\qty{150}{\nano\meter}$), leading to enhanced absorption.
    }
    \label{fig:exp:tmm:field}
\end{figure}

\begin{margintable}
    \centering
    \footnotesize
    \caption{}
    \label{tab:}
    % This table is generated by img/py/experiment/tmm.py
\begin{tabular}{lrr}
\toprule
 & $\mathcal{A}$ & $\mathcal{R}$ \\
\midrule
Bare & 0.029 & 0.22 \\
TG & 0.018 & 0.42 \\
BG & 0.005 & 0.83 \\
TGBG & 0.0041 & 0.85 \\
\bottomrule
\end{tabular}

\end{margintable}

\begin{marginfigure}
    \centering
    \includegraphics{img/pdf/experiment/tmm_absorptance}
    \caption[\imgsource{img/py/experiment/tmm.py}]{
        \Gls{qw} absorptance \absorptance in a heterostructure with default (blue) and optimized (magenta) barrier thickness and top and bottom gates as function of wavelength.
        Optimization was performed at \qty{825}{\nano\meter} using the differential evolution algorithm implemented in \pymoosh, resulting in a barrier thickness of \qty{122}{\nano\meter} and an absorptance better by a factor of \num{16} at \qty{6.3}{\percent}.
    }
    \label{fig:exp:tmm:wavelengths}
\end{marginfigure}
\clearpage
\begin{marginfigure}
    \centering
    \includegraphics{img/pdf/experiment/tmm_green}
    \caption[\imgsource{img/py/experiment/tmm.py}]{
        Real part of the electric field emitted by a current line located in the \gls{qw} (black point) for different cases of the unoptimized structure.
        From top to bottom: bare heterostructure, top gate, bottom gate, top and bottom gate.
        The half space $z<0$ is the air above the membrane in the direction of the objective lens and the dotted lines indicate interfaces between materials.
        Evidently, the bottom gate reduces the amplitude in the upper half of the membrane and thereby the outcoupling efficiency compared to the structures with just a top gate, consistent with what is observed in the experiment.
    }
    \label{fig:exp:tmm:green}
\end{marginfigure}

\begin{marginfigure}
    \centering
    \includegraphics{img/pdf/experiment/tmm_green_opt_tgbg}
    \caption[\imgsource{img/py/experiment/tmm.py}]{
        Real part of the electric field emitted by a current line located in the \gls{qw} (black point) for the default (top) and optimized (bottom) structures with top and bottom gates.
        Optimizing the barrier thickness for absorption in the \gls{qw} evidently also drastically improves the outcoupling efficiency into the halfspace $z<0$.
    }
    \label{fig:exp:tmm:green:opt:tgbg}
\end{marginfigure}
\clearpage

\begin{figure}
    \centering
    \includegraphics{img/pdf/experiment/honey_H13_stark_shift_vs_gate}
    \caption[
        \sampleid{Honey H13}
        \thewavelength{795},
        \thepower{1}{\micro}.
        \protect\newline
        \imgsource{img/py/experiment/pl.py}
    ]{
        \Gls{pl} as function of gate voltage on a single fan-out gate on the bottom (left) and top (right) side of the membrane.
        The behavior is qualitatively similar but the overall quantum efficiency lower by an order of magnitude for gates on the bottom (as-grown buried) side.
    }
    \label{fig:exp:pl:honey_H13_stark_shift_vs_gate}
\end{figure}

\begin{marginfigure}
    \centering
    \includegraphics{img/pdf/experiment/doped_M1_05_49-2_difference_mode}
    \caption[
        \sampleid{Doped M1_05_49-2}
        \thevoltage{-1.3}{CM},
        \thewavelength{795}.
        \protect\newline
        \imgsource{img/py/experiment/pl.py}
    ]{
        \Gls{pl} as function of difference-mode voltage on a large exciton trap.
        The observed Stark shift follows the expected quadratic dispersion, but is offset by \qty{0.9}{\volt} with respect to zero bias (dash-dotted gray line).
        Remnant \gls{pl} of the \gls{2deg} from outside the trap region is faintly visible below \qty{-1}{\volt}.
    }
    \label{fig:exp:pl:doped_M1_05_49-2_difference_mode}
\end{marginfigure}

\begin{marginfigure}
    \centering
    \includegraphics{img/pdf/experiment/fig_F10_positioning}
    \caption[
        \sampleid{Fig F10}
        \thewavelength{795}.
        \protect\newline
        \imgsource{img/py/experiment/pl.py}
    ]{}
    \label{fig:exp:pl:fig_F10_positioning}
\end{marginfigure}

\begin{figure}
    \centering
    \includegraphics{img/pdf/experiment/doped_M1_05_49-2_power}
    \caption[
        \sampleid{Doped M1_05_49-2}
        \thevoltage{-2.7}{DM},
        \thevoltage{-1.3}{CM},
        \thewavelength{795}.
        \protect\newline
        \imgsource{img/py/experiment/pl.py}
    ]{
    }
    \label{fig:}
\end{figure}

\begin{figure}
    \centering
    \includegraphics{img/pdf/experiment/doped_M1_05_49-2_multiplets}
    \caption[
        \sampleid{Doped M1_05_49-2}
        \thevoltage{0}{B},
        \protect\newline
        \imgsource{img/py/experiment/pl.py}
    ]{}
    \label{fig:}
\end{figure}
