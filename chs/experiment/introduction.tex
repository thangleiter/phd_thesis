% mainfile: ../../main.tex
\chapter{Introduction}\label{ch:exp:introduction}
\AutoLettrine{Just} as quantum computers are conceived as the quantum analogon of classical computers with bits and logic operations switched out by quantum counterparts, so can one devise \emph{networks} of such objects, where quantum information generated or processed at a quantum \emph{node} is distributed across long distances through the quantum counterparts of classical information channels~\cite{Nielsen2011,Simon2017}.
Famously envisioned by \citet{Kimble2008}, the concept can be extended to the idea of a \emph{quantum internet}.
A wide array of ideas has been put forth that make use of the theoretical capabilities of such quantum networks.

Initially, quantum networks were studied in the context of quantum cryptography~\cite{Bennett1984,Ekert1991,Deutsch1996,Gisin2002}.
There, the no-cloning theorem and clever use of entanglement ensure quantum-secured communication between distant parties that cannot be eavesdropped upon or tampered with by adversarial parties without detection.\sidenote{
    As ever in cryptography, new protocols keep getting hacked and loop-holes are discovered~\cite{Huang2018,Pang2020}.
    It will be interesting to see, therefore, if the security will faithfully transfer from theory to experiment.
}
Considerable attention has also been paid to the notion of distributed quantum computation~\cite{Cirac1999}.
As quantum computers do not appear to be on the same course of miniaturization as classical computers have been, there might turn out to be a limit to the physical size of quantum computers and, hence, a limit to the processing power of a monolithic node.
Distributed quantum computation resolves this bottleneck by allowing computations to be executed across separate nodes much like classical supercomputer clusters.
Although a comprehensive resource assessment of the feasibility of such approaches is still outstanding, initial results are promising~\cite{Jacinto2025}, and experimental demonstrations of small computations have recently been shown~\cite{Main2025}.
A concept combining distributed quantum computation with quantum cryptography is blind quantum computation~\cite{Childs2005,Giovannetti2013}, which promises a form of cloud-based quantum computation.
Also here first experimental demonstrations have been achieved~\cite{Wei2025}.

Next, quantum networks have garnered interest in the field of quantum sensing.
This term refers to the branch of quantum technology in which individual quantum systems are employed as highly sensitive sensors, for example of magnetic fields, or, more generally, to perform measurements of physical quantities~\cite{Giovannetti2004,Degen2017}.
There exist proposals to employ quantum networks for long-baseline telescopes that use optical interferometry to enhance the resolution of astronomical imaging~\cite{Gottesman2012,Khabiboulline2019}, akin to the techniques used to produce the first image of a supermassive black hole~\cite{TheEventHorizonTelescopeCollaboration2019}.
Going beyond technological applications, the capability to coherently transmit quantum states across large distances opens the pathway to tests of quantum theory itself, and where it might fail~\cite{Weinberg1989}.\sidenote{
    This area of physics is termed \emph{foundations of physics}.
}
At least since the publication of the \gls{epr}--paradox~\cite{Einstein1935}, tests of the non-locality of quantum mechanics have been proposed~\cite{Bell1964,Clauser1969} and performed~\cite{Hensen2015,Storz2023}.
More recently, for example, a small quantum network was used to rule out a description of quantum theory by real numbers~\cite{Li2022} and we may expect more such experiments to come~\cite{Shadbolt2014}.
Indeed, research into complex quantum networks and their properties and possible applications is still in its beginnings~\cite{Nokkala2024}.

%\cite{Hardy2009}\cite{Goswami2020}\cite{Rozema2024} % causal order

So how does such a quantum network work?
In the \enquote{canonical quantization} picture we already adopted previously, we might simply replace classical, optical links by quantum versions thereof and similarly transmit \emph{flying} qubits instead of bits through those channels.
However, even optical fibers, the backbone of the modern internet, are lossy, and since the photon loss scales exponentially with distance, there would be little hope to build networks larger than a few to a few tens of kilometers.\sidenote{
    We can expect a survival probability of \qty{1}{\percent} over a distance of \qty{100}{\kilo\meter}~\cite{Azuma2023}.
    There is therefore arguably no feasible alternative to optical transmission over long distances.
}
In classical networks, this problem is remedied by repeater stations that simply produce copies of incoming photons and thus amplify the signal.
In quantum mechanics, however, this is forbidden by the no-cloning theorem, which states that one cannot achieve a perfect copy of a qubit prepared in an arbitrary and unknown quantum state~\cite{Wootters1982,Dieks1982}.
To the rescue comes, then, entanglement.
By letting two adjacent repeater stations share a bipartite maximally entangled state (often referred to as a \gls{epr} or Bell pair), a station, Charlie, positioned between two others, Alice and Bob, each of whom Charlie shares Bell pairs with can perform a Bell measurement on the two halves of the pairs in his possession und thereby project Alice and Bob's halves into a state that is maximally entangled between the two of them.
This technique of entangling two states that have never interacted with each other is known as entanglement swapping~\cite{Zukowski1993,Pan1998}.
\citet{Briegel1998,Dur1999} then proposed a quantum repeater protocol that uses entanglement swapping, enhanced by entanglement distillation,\sidenote{
    Also known as entanglement purification.
}
to successively entangle neighboring pairs of entangled states whose resource requirements scale logarithmically with the length of the quantum channel between whose ends entanglement needs to be established.
What is more, the protocol tolerates error and loss rates on the percent level and is thus much more benign than \gls{qec}.
Following the initial proposal, more improved schemes were developed that tolerate higher errors~\cite{Dur2007} or employ entirely different techniques~\cite{Bayrakci2022}, see \citer{Azuma2023} for a review, and recently also experimental realizations have been shown~\cite{Krutyanskiy2023}.

A crucial detail of the protocol is that it requires storing Bell states until the heralding of successful entanglement in a quantum memory.\sidenote{
    I note that there exist also protocols for memoryless, all-optical quantum repeaters~\cite{Li2019,Azuma2023}.
}
In practice, this means that quantum repeaters require a coherent light-matter interface between photonic flying qubits and stationary quantum memory since storing photons is not feasible.
Such an interface has been the subject of intense research and there exist a large number of competing approaches that differ in choice of material platform for the memory and choice of encoding for the photon~\cite{Awschalom2018,Beukers2024}.
Among the most advanced are atomic and defect-based systems.
Atomic and ionic systems are a natural choice as the energy scales of atomic transitions are compatible with photons in the telecom range ($\sim\qty{1550}{\nano\meter}$)~\cite{Sangouard2011,Krutyanskiy2023,Liu2024,Kucera2024}.
Nuclear spins coupled to defects in crystal lattices have long spin lifetimes at the same time as optical transitions~\cite{Togan2010,Nguyen2019,Bergeron2020,Stolk2024,Knaut2024}.
Optically interfacing semiconductor spin qubits or superconducting qubits, on the other hand, is more challenging because of a separation of energy scales.
While qubits in these systems have energy splittings in the \unit{\giga\hertz} regime, telecom photons have energies of hundreds of \unit{\tera\hertz}, and so bridging this gap requires some sort of intermediary.

\cite{Knaut2024,Liu2024,Kucera2024,Stolk2024} % network demonstration