% mainfile: ../main.tex
\chapter*{How to read this thesis}
\addcontentsline{toc}{chapter}{How to read this thesis}

\Thethesis has been typeset with Lua\LaTeX{} and is optimized for digital reading.
The document source code is available on \href{https://github.com/thangleiter/phd_thesis/}{Github}.

I make liberal use of cross-references and acronyms, the latter of which link to and are defined in \hyperref[glo]{the Glossary}.
As such, I recommend a PDF viewer that can preview the targets of internal hyperlinks for reading \thethesis in order to avoid having to jump back and forth within the document.

The various figures included in this document are either generated by \LaTeX{} itself or by a Python script included in the source repository.
Figure captions contain a hyperlink to the \hyperref[lof]{List of Figures} whose entries point to the source file that generates the figure.
For figures that contain experimental data, the entry in the \hyperref[lof]{List of Figures} furthermore typically contains values of external parameters that were fixed during the measurement.

Finally, I frequently employ an $\asinh$-scale for data that spans several orders of magnitude.
This scaling is asymptotically equivalent to a symmetric logarithmic scale for large $\abs{x}$, $\asinh(\pm x)\sim\pm\log\abs{x}$, but avoids the divergence close to zero, instead being linear, $\asinh(x)\sim x$.
The crossover from linear to logarithmic behavior is governed by a parameter $x_0$ such that $x\to x_0\asinh(\flatfrac{x}{x_0})$ and that can be freely chosen and influences the representation of the data.
The particular value of $x_0$ (corresponding to the \code{linear_width} parameter in \href{https://matplotlib.org/stable/gallery/scales/asinh_demo.html}{\matplotlib}) in a given figure can be looked up in the script file that generates it.
