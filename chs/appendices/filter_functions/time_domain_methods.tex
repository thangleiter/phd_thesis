% mainfile: ../../main.tex
\chapter{Monte Carlo and Lindblad master equation simulations}\label{ch:app:ff:time_domain_methods}
\section{Validation of QFT fidelities}\label{sec:app:ff:time_domain_methods:qft_validation}
In this section, we perform Lindblad master equation and \gls{mc} simulations to verify the fidelities predicted for the \gls{qft} circuit in the main text.
We focus on noise exclusively on the third qubit, entering through the noise operator $B_\alpha\equiv\sigma_y\gth{3}$.

\todo{this}
We assemble the \gls{qft} circuit discussed in the main text from a minimal gate set consisting of three atomic gates, $\mathbb{G} = \lbrace\mr{X}_{i}(\flatfrac{\pi}{2}),\mr{Y}_{i}(\flatfrac{\pi}{2}),\mr{CR}_{ij}(\flatfrac{\pi}{2^3})\rbrace$ on or between qubits $i$ and $j$.
We consider a simple model involving four single-spin qubits with in-phase (I) and quadrature (Q) single-qubit control and nearest neighbor exchange coupling so that the control Hamiltonian reads
\begin{equation}\label{eq:app:ff:control_hamiltonian:qft}
    H_\mr{c}(t) = \sum_{\langle i,j\rangle} I_i(t)\sx^{(i)} + Q_i(t)\sy^{(i)} + J_{ij}(t)\sz^{(i)}\otimes\sz^{(j)}
\end{equation}
where $\sigma_\alpha^{(i)}$ is the trivial extension of the Pauli matrix $\sigma_\alpha$ of qubit $i$ to the full tensor product Hilbert space.
For simplicity, we assume periodic boundary conditions so that qubits 1 and 4 are nearest neighbors as well.
Similarly, we define the noise Hamiltonian as
\begin{equation}\label{eq:app:ff:noise_hamiltonian:qft}
H_\mr{n}(t) = \sum_{\langle i,j\rangle} b_{I}(t)\sx^{(i)} + b_{Q}(t)\sy^{(i)} + b_{J}(t)\sz^{(i)}\otimes\sz^{(j)}
\end{equation}
with the noise fields $b_{\alpha}(t)$ for $\alpha\in\{I,Q,J\}$.

To validate the fidelity for white noise, we use a Lindblad master equation~\cite{Lindblad1976,Gorini1976} in superoperator form.
We represent linear maps $\mc{A}: \rho\rightarrow\mc{A}(\rho)$ by matrices in the Pauli transfer matrix representation as (see \citer{Hangleiter2021} for more details)
\begin{equation}
    \mc{A}_{ij} \coloneqq \tr(\sigma_i\mc{A}(\sigma_j))
\end{equation}
and operators as column vectors (\ie, generalized Bloch vectors) as
\begin{equation}\label{eq:app:ff:bloch_vector}
    \rho_i \coloneqq \tr(\sigma_i\rho),
\end{equation}
allowing us to write the Lindblad equation
\begin{equation}\label{eq:app:ff:lindblad:hilbert}
\dv{t}\rho(t) = -\i\comm{H(t)}{\rho(t)} + \sum_\alpha \gamma_\alpha\left(L_\alpha\rho(t) L_\alpha\adjoint - \frac{1}{2}\acomm{L_\alpha\adjoint L_\alpha}{\rho(t)}\right)
\end{equation}
as a linear differential equation in matrix form,
\begin{equation}\label{eq:app:ff:lindlbad:liouville}
\dv{t}\rho_i(t) = \sum_j\left(-\i\mc{H}_{ij}(t)+ \sum_\alpha \gamma_\alpha \mc{D}_{\alpha, ij}\right)\rho_j(t).
\end{equation}
Here, $\mc{H}_{ij}(t) = \tr(\sigma_i\comm{H(t)}{\sigma_j})$ and $\mc{D}_{\alpha, ij} = \tr(\sigma_i L_\alpha\sigma_j L_\alpha\adjoint - \frac{1}{2}\sigma_i \acomm{L_\alpha\adjoint L_\alpha}{\sigma_j})$.
By setting $L_\alpha\equiv\sigma_y\gth{3}$ as well as $\gamma_\alpha\equiv\flatfrac{S_0}{2}$ with $S_0$ the amplitude of the one-sided noise \gls{psd} so that $S(\omega) = S_0$, we can compare the fidelity obtained from the filter functions to that from the explicit simulation of \cref{eq:app:ff:lindlbad:liouville}.
The latter is computed as $\avgfid = d^{-2}\tr(\mc{Q}\adjoint\mc{U})$, where $\mc{Q}$ is the superpropagator due to the Hamiltonian evolution alone (\ie, the ideal evolution without noise).

For the Monte Carlo simulation, we explicitly generate time traces of $b_Q(t)$ (\cf \cref{eq:app:ff:noise_hamiltonian:qft}) by drawing pseudo-random numbers from a distribution whose \gls{psd} is $S(f) = \flatfrac{S_0}{f}$.
To do this, we draw complex, normally distributed samples in frequency space (\ie white noise), scale it with the square root of the \gls{psd}, and finally perform the inverse Fourier transform.
We choose an oversampling factor of 16 so that the time discretization of the simulation is $\Delta t_\mr{MC} = \flatfrac{\Delta t}{16} = \qty{62.5}{\pico\second}$ ($\Delta t = \qty{1}{\nano\second}$ is the time step of the pulses used in the FF simulation), leading to a highest resolvable frequency of $\fmax = \qty{16}{\giga\hertz}$.
Conversely, we increase the frequency resolution by sampling a time trace longer by a given factor, giving frequencies below \fmin (\qty{16}{\kilo\hertz} for pink, \qty{0}{\hertz} for white noise) weight zero, and truncating it to the number of time steps in the algorithm times the oversampling factor.
This yields a time trace with frequencies $f\in [\fmin, \fmax]$ and a given resolution (we choose $\df = \qty{160}{\hertz}$).

We then proceed to diagonalize the Hamiltonian $H(t) = H_c(t) + H_n(t)$ and compute the propagator for one noise realization as
\begin{equation}\label{eq:app:ff:mc_propagator}
U(t) = \prod_g V\gth{g}\exp\lbrace -\i\Omega\gth{g}\Delta t_\mr{MC}\rbrace V^{(g)\dagger},
\end{equation}
where $V\gth{g}$ is the unitary matrix of eigenvectors of $H(t)$ during time segment $g$ and $\Omega\gth{g}$ the diagonal matrix of eigenvalues.
Finally, we obtain an estimate for the average gate fidelity \avgfid from the entanglement fidelity \entfid as
\begin{equation}
    \ev{\avgfid} = \ev{\frac{d\entfid + 1}{d+1}} = \ev{\frac{\abs{\tr(Q\adjoint U(\tau))}^2 + d}{d(d+1)}}.
\end{equation}
Here, $Q\equiv\Uc(t=\tau)$ is the noise-free propagator at time $\tau$ of completion of the circuit and $\ev{\placeholder}$ denotes the ensemble average over $N$ Monte Carlo realizations of \cref{eq:app:ff:mc_propagator}, \ie, $\ev{A}=N\inverse\sum_{i=1}^N A_i$.
The standard error of the mean can be obtained as $\sigma_{\ev{\avgfid}} = \sigma_{\avgfid} / \sqrt{N}$ with $\sigma_{\avgfid}$ the standard deviation over the Monte Carlo traces.
