% mainfile: ../../main.tex
We begin by setting some notation.
% TODO
Wherever possible, we omit indices and thus imply matrix multiplication between objects.
We assume a single noise operator and drop the corresponding index; we can easily add them again later since none of the manipulations involve the noise indices.
We fully adopt the picture that control matrices can be concatenated by summing over individual time steps, which may either be single piecewise-constant segments or entire sequences, corresponding to \cref{eq:ff:control_matrix:pulse:freq:ff:calculation} or \cref{eq:ff:control_matrix:sequence:freq}, respectively.
To make clear that the control \enquote{matrices} are in fact vectors in Liouville space, we write them as bras wherever possible,\sidenote{
    Unfortunately, we are running out of symbols and sticking to our previous convention of using Roman font for Hilbert-space operators and calligraphic font for their Liouville-space duals would lead to naming clashes.
    We therefore use a calligraphic font for either.
}
\begin{equation}
    \ctrlmat(\omega) \doteq \dbra*{\ctrlmat(\omega)}
\end{equation}
and define as shorthands
\begin{align}
    \mc{G}\gth{g}(\omega) & \doteq \dbra*{\mc{G}\gth{g}(\omega)}
                            \coloneqq \e^{\i\omega t_{g-1}}\dbra*{\ctrlmat\gth{g}(\omega)}\liouvQ\gth{g-1} \label{eq:app:ff:ctrlmat_step} \\
    \mc{M}\gth{g}(\omega) & \doteq \dbra*{\mc{M}\gth{g}(\omega)}
                            \coloneqq \sum_{g^{\prime}=1}^{g}\dbra*{\mc{G}\gth{g^{\prime}}(\omega)} \label{eq:app:ff:ctrlmat_cumulative}
\end{align}
such that
\begin{equation}
    \dbra*{\ctrlmat(\omega)} = \dbra*{\mc{M}\gth{G}(\omega)} = \sum_{g^{\prime}=1}^G\dbra*{\mc{G}\gth{g^{\prime}}(\omega)}.
\end{equation}
\Ie, $\mc{M}\gth{g}(\omega)$ is the \emph{cumulative} control matrix of a sequence of $G$ total steps up to step $g$ and can thus be viewed as a function defined at discrete points in time, $\mc{M}\gth{g}(\omega)\simeq \mc{M}(\omega;t_0,t_g)$.
Finally, we drop the specifier $\FF\gth{2}$ distinguishing the second- from the first-order filter function for brevity; in this section, we always mean the former.

% TODO: sub- and superscripts
In the following, we will consider a sequence of piecewise-constant time steps split up into subsequences and will deal on the one hand with quantities that depend exclusively on the internal structure of a subsequence and those that do not depend on the internal structure on the other.
We will refer to the former as \emph{local} quantities, and denote their time step by a parenthesised superscript, \eg, $A\gth{i}$, which means the $i$th time step of $A$.
The subsequences themselves will be enumerated by a parenthesised subscript, so that if $A$ were a quantity of subsequence $g$, we would denote its $i$th time step as $A_{(g)}\gth{i}$.
Lastly, \emph{non-local} quantities that depend on preceding sequences will then be denoted in a \enquote{posterior} fashion, \eg, $A_{(g|g-1\to 1)}$ if $A$ depends on all sequences $g-1, \dotsc, 1$.

We start from \cref{eq:ff:frequency_shifts:freq}, which defines the second-order filter function by
\begin{equation}\label{eq:app:ff:filter_function:complete}
    \FF_{\alpha\beta,kl}(\omega) \coloneqq \sum_{g=1}^{G}\left[
        \mc{G}_{\alpha k}^{(g)\ast}(\omega) \mc{M}_{\beta l}\gth{g-1}(\omega) + \mc{J}_{\alpha\beta,kl}\gth{g}(\omega)
    \right]
\end{equation}
with
\begin{equation}\label{eq:app:ff:nested_time_step}
    \mc{J}_{\alpha\beta,kl}\gth{g}(\omega) \coloneqq s_\alpha\gth{g}\bar{B}_{\alpha,ij}\gth{g}\bar{C}_{k,ji}\gth{g} I_{ijmn}\gth{g}(\omega)\bar{C}_{l,nm}\gth{g}\bar{B}_{\beta,mn}\gth{g} s_\beta\gth{g}
\end{equation}
where repeated indices are contracted.
Comparing to the nested time integral, the first summand in the brackets contains all contributions from complete time segments up to the one containing the inner integration variable $t$, whereas the second captures the final, incomplete segment with $t_{g-1} < t \leq t_{g}$.
Now imagine the sequence of piecewise-constant time steps, $g\in\{1,\dotsc,G\}$, being split apart at some index $1<\gamma<G$ and thereby being divided into two subsequences $g\in\{1,\dotsc,\gamma\}$ and $h\in\{1,\dotsc,\eta\} \equiv g\in\{\gamma+1,\dotsc,G\}$ with $G = \gamma + \eta$.
Our goal is to obtain an expression for the second-order filter function $\FF(\omega)$ that is -- as much as possible -- a sum of local terms of these subsequences $g$ and $h$ and some term that cannot be expressed in terms of either $g$ or $h$ alone.

Up to $\gamma$, the filter function is simply that of the first sequence, $\FF_{(1)}(\omega)$, and we thus have for $g>\gamma$\sidenote{
    We drop indices for legibility as stated above; $\mc{J}\gth{g}(\omega)$ is a matrix on Liouville space, whereas $\mc{G}\gth{g}(\omega)$ and $\mc{M}\gth{g}(\omega)$ are Liouville-space row vectors and their product here is an outer product, $\dop*{\mc{G}\gth{g}(\omega)}{\mc{M}\gth{g-1}(\omega)}$, resulting in a matrix on Liouville space.
}
\begin{align}
    \FF_{(2|1)}(\omega) &= \FF(\omega) - \FF_{(1)}(\omega) \notag \\
                        &= \sum_{g=\gamma+1}^{G}\left[
                            \dop*{\mc{G}\gth{g}(\omega)}{\mc{M}\gth{g-1}(\omega)} + \mc{J}\gth{g}(\omega)
                        \right].\label{eq:app:ff:filter_function:12:2}
\end{align}
We must now express the quantities $\mc{G}\gth{g}(\omega)$, $\mc{M}\gth{g-1}(\omega)$, and $\mc{J}\gth{g}(\omega)$ locally in terms of the index $h$.
To this end, we first write down the step-wise control matrix $\mc{G}\gth{g}(\omega)$ in the second sequence and split off phases and propagators from the first sequence,
\begin{align}\label{eq:app:ff:ctrlmat_step:12}
    \dbra*{\mc{G}\gth{g}(\omega)} &= \e^{\i\omega t_{g}}\dbra*{\ctrlmat\gth{g}(\omega)}\liouvQ\gth{g-1} \notag \\
                                  &= \e^{\i\omega (t_\gamma + t_{h})}\dbra*{\ctrlmat_{(2)}\gth{h}(\omega)}\liouvQ_{(2)}\gth{h-1}\liouvQ_{(1)} \notag \\
                                  &= \e^{\i\omega\tau_{(1)}}\dbra*{\mc{G}_{(2)}\gth{h}(\omega)}\liouvQ_{(1)} \notag \\
                                  &= \dbra*{\mc{G}_{(2|1)}\gth{h}(\omega)}
\end{align}
where $\tau_{(1)}=t_{\gamma}$ and $\liouvQ_{(1)}$ is the control superpropagator of sequence (1).
Next, we consider the cumulative control matrix $\mc{M}\gth{g-1}(\omega)$.
Because in the total sequence it is given by the sum over all $\mc{G}\gth{g^{\prime}}(\omega)$ up to $g-1$, we can split off the complete control matrix of the first sequence and express the remainder by summing over $\mc{G}\gth{h}_{(2|1)}(\omega)$ from \cref{eq:app:ff:ctrlmat_step:12}:
\begin{align}
    \dbra*{\mc{M}\gth{g-1}(\omega)} &= \dbra*{\ctrlmat_{(1)}(\omega)} + \e^{\i\omega\tau_{(1)}}\sum_{h^{\prime}=1}^{g-1-\gamma}
                                        \dbra*{\mc{G}_{(2)}\gth{h^{\prime}}(\omega)}\liouvQ_{(1)} \notag \\
                                    &= \dbra*{\ctrlmat_{(1)}(\omega)} + \e^{\i\omega\tau_{(1)}}\dbra*{\mc{M}_{(2)}\gth{g-1-\gamma}(\omega)}\liouvQ_{(1)} \notag \\
                                    &= \dbra*{\mc{M}_{(2|1)}\gth{h-1}(\omega)}. \label{eq:app:ff:ctrlmat_cumulative:12:2}
\end{align}
Finally, we need to unravel $\mc{J}\gth{g}(\omega)$.
We start from \cref{eq:ff:control_matrix:pulse:freq:ff:calculation}, consider a time step $g\geq\gamma$ in the second sequence with $h = g-\gamma$, and rewrite
\begin{align}
        \mc{J}_{kl}\gth{g}(\omega) =& s\gth{g} \bar{B}_{ij}\gth{g} \bar{C}_{kji}\gth{g} I_{ijmn}\gth{g}(\omega)
                                       \bar{C}_{(2|1),lnm}\gth{g} \bar{B}_{mn}\gth{g} s\gth{g} \notag \\
                                   =& s_{(2)}\gth{h} \bar{B}_{(2),ij}\gth{h} \bar{C}_{(2|1),kji}\gth{h} I_{(2),ijmn}\gth{h}(\omega)
                                       \bar{C}_{lnm}\gth{h} \bar{B}_{(2),mn}\gth{h} s_{(2)}\gth{h} \notag \\
                                   =& \mc{J}_{(2|1),kl}\gth{h}(\omega) \label{eq:app:ff:filter_function:incomplete_timestep}
\end{align}
because all quantities except for $\bar{C}_{(2|1)}\gth{g}$ depend on their timestep $g$ alone, and where $i,j,m,n$ index the Hilbert space dimensions of the operators, while $k,l$ are the usual indices for the basis elements and therefore Liouville space dimensions.
On that term, we can factor out the propagators of the first complete sequence,\sidenote{
    Note that the $Q_{(i)}$ here are Hilbert space propagators, not their Liouville space counter parts $\liouvQ_{(i)}$, and that $Q_{(1)}\gth{h-1}\equiv Q_{h-1}$ in the notation of \cref{subsec:ff:theory:control_matrix:sequence}.
}
\begin{align}
    \bar{C}_{(2|1),kij}\gth{h} = \left[V_{(2)}^{(h)\dagger}Q_{(2)}\gth{h-1}Q_{(1)} C_k Q_{(1)}\adjoint Q_{(2)}^{(h-1)\dagger}V_{(2)}\gth{h}\right]_{ij}.
\end{align}

We can now finally put all pieces together and, starting from \cref{eq:app:ff:filter_function:12:2}, plug in \cref{eq:app:ff:ctrlmat_step:12,eq:app:ff:ctrlmat_cumulative:12:2,eq:app:ff:filter_function:incomplete_timestep}, so that we obtain
\begin{align}
    \FF_{(2|1)}(\omega) = \sum_{h=1}^{\eta}\Bigl[ & \dop*{\mc{G}_{(2|1)}\gth{h}(\omega)}{\mc{M}_{(2|1)}\gth{h-1}(\omega)}
                                                    + \mc{J}_{(2|1)}\gth{h}(\omega)\Bigr] \notag \\
                        = \sum_{h=1}^{\eta}\Bigl\lbrace
                               & \e^{-\i\omega\tau_{(1)}}\liouvQ_{(1)}\transpose\dket*{\mc{G}_{(2)}\gth{h}(\omega)} \\
                               & \times\Bigl[
                                     \dbra*{\ctrlmat_{(1)}(\omega)} + \e^{\i\omega\tau_{(1)}}\dbra*{\mc{M}_{(2)}\gth{g-1-\gamma}(\omega)}\liouvQ_{(1)}
                                 \Bigr] +\mc{J}_{(2|1)}\gth{h}(\omega)
                        \Bigr\rbrace. \notag
\end{align}
To simplify the unwieldy first summand in the curly braces further, we factor out the constant superpropagator $\liouvQ_{(1)}$,\sidenote{
    Recall that \liouvQ is the Liouville representation of the unitary operator $Q$ and as such -- and because our chosen basis \basis is Hermitian -- is an orthogonal matrix for which $\liouvQ\transpose\liouvQ = \eye$.
}
\begin{align}
    \MoveEqLeft \e^{-\i\omega\tau_{(1)}}\liouvQ_{(1)}\transpose\dket*{\mc{G}_{(2)}\gth{h}(\omega)}\left[\dbra*{\ctrlmat_{(1)}(\omega)}
            + \e^{\i\omega\tau_{(1)}}\dbra*{\mc{M}_{(2)}\gth{h-1}(\omega)}\liouvQ_{(1)}\right] \notag \\
        &= \liouvQ_{(1)}\transpose\dket*{\mc{G}_{(2)}\gth{h}(\omega)}\left[
               \e^{-\i\omega\tau_{(1)}}\dbra*{\ctrlmat_{(1)}(\omega)}\liouvQ_{(1)}\transpose + \dbra*{\mc{M}_{(2)}\gth{h-1}}
           \right]\liouvQ_{(1)} \\
        &\eqqcolon \liouvQ_{(1)}\transpose\dop*{\mc{G}_{(2)}\gth{h}(\omega)}{\mc{N}_{(2)}\gth{h-1}(\omega)}\liouvQ_{(1)},
\end{align}
where we defined the quantity
\begin{equation}
    \dbra*{\mc{N}_{(2)}\gth{h-1}(\omega)} = \e^{-\i\omega\tau_{(1)}}\dbra*{\ctrlmat_{(1)}(\omega)}\liouvQ_{(1)}\transpose + \dbra*{\mc{M}_{(2)}\gth{h-1}}
\end{equation}
that contains contributions from local terms of both sequences.
We have thus reduced the term to an outer product of two vectors on Liouville space whose only contributions are from terms from either sequence which can be reused in a calculation, and obtain at last
\begin{equation}\label{eq:app:ff:filter_function:12}
    \FF_{(2|1)}(\omega) = \sum_{h=1}^{\eta}\left[
        \liouvQ_{(1)}\transpose\dop*{\mc{G}_{(2)}\gth{h}(\omega)}{\mc{N}_{(2)}\gth{h-1}(\omega)}\liouvQ_{(1)} + \mc{J}_{(2|1)}\gth{h}(\omega)
    \right].
\end{equation}
In \cref{eq:app:ff:filter_function:12}, all terms except the last are properties of the individual sequences alone and can thus be computed ahead of time.
Indeed, we can now simply iterate this result to the general case of a sequence of $G$ subsequences with lengths $\lbrace\eta_g\rbrace_{g=1}^G$ by recursively splitting apart the later sequence $(2)$ and applying \cref{eq:app:ff:filter_function:12}.

Let us first briefly discuss the special case where $G=1$.
Then, $\liouvQ_{(0)}=\eye$, $\dbra*{\ctrlmat_{(0)}} = 0$, and $\dbra*{\mc{N}_{(1)}\gth{h-1}(\omega)} = \dbra*{\mc{M}_{(1)}\gth{h-1}(\omega)}$ so that \cref{eq:app:ff:filter_function:12} reduces to \cref{eq:app:ff:filter_function:complete} as it should.
We can therefore write the concatenation rule for second-order filter functions simply as
\begin{align}
    \FF(\omega) =& \sum_{g=1}^{G} \FF_{(g|g-1\to 1)}(\omega) \\
                =& \sum_{g=1}^{G}\sum_{h_{g}=1}^{\eta_g}\left[
                        \liouvQ_{(g-1)}\transpose\dop*{\mc{G}_{(g)}\gth{h_g}(\omega)}{\mc{N}_{(g)}\gth{h_g-1}(\omega)}\liouvQ_{(g-1)}
                        + \mc{J}_{(g|g-1\to 1)}\gth{h_g}(\omega)
                    \right] \label{eq:app:ff:filter_function:concatenated}
\end{align}
\Cref{eq:app:ff:filter_function:concatenated} is our final result.
It contains two terms; the first, the sum over $G$ transfer matrices, is the concatenated 
The terms in the first line of \cref{eq:app:ff:filter_function:concatenated} are all local functions of sequence $(g)$ and the sum in the second line needs to be computed for every single time step $h\in\{1,\dotsc,\sum_g\eta_g\}$.\sidenote{
    We note that only contractions are required (see \cref{eq:app:ff:filter_function:incomplete_timestep}); the individual terms, including the second-order integral (\cref{eq:ff:frequency_shifts:integral}), can all be retained from the computation of the individual sequence's second-order filter function.
}