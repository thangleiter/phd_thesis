% mainfile: ../../main.tex
We begin by setting some notation.
% TODO
Wherever possible, we omit indices and thus imply matrix multiplication between objects.
We assume a single noise operator and drop the corresponding index; we can easily add them again later since none of the manipulations involve the noise indices.
We fully adopt the picture that control matrices can be concatenated by summing over individual time steps, which may either be single piecewise-constant segments or entire sequences, corresponding to \cref{eq:ff:control_matrix:pulse:freq:ff:calculation} or \cref{eq:ff:control_matrix:sequence:freq}, respectively.
To make clear that the control \enquote{matrices} are in fact vectors in Liouville space, we write them as bras wherever possible,\sidenote{
    Unfortunately, we are running out of symbols and sticking to our previous convention of using Roman font for Hilbert-space operators and calligraphic font for their Liouville-space duals would lead to naming clashes.
    We therefore use a calligraphic font for either.
}
\begin{equation}
    \ctrlmat(\omega) \doteq \dbra*{\ctrlmat(\omega)}
\end{equation}
and define as shorthands for \cref{eq:ff:control_matrix:sequence:freq}
\begin{align}
    \mc{G}\gth{g}(\omega) & \doteq \dbra*{\mc{G}\gth{g}(\omega)}
                            \coloneqq \e^{\i\omega t_{g-1}}\dbra*{\ctrlmat\gth{g}(\omega)}\liouvQ\gth{g-1} \label{eq:app:ff:ctrlmat_step} \\
    \mc{M}\gth{g}(\omega) & \doteq \dbra*{\mc{M}\gth{g}(\omega)}
                            \coloneqq \sum_{g^{\prime}=1}^{g}\dbra*{\mc{G}\gth{g^{\prime}}(\omega)} \label{eq:app:ff:ctrlmat_cumulative}
\end{align}
such that
\begin{equation}
    \dbra*{\ctrlmat(\omega)} = \dbra*{\mc{M}\gth{G}(\omega)} = \sum_{g^{\prime}=1}^G\dbra*{\mc{G}\gth{g^{\prime}}(\omega)}.
\end{equation}
\Ie, $\mc{M}\gth{g}(\omega)$ is the \emph{cumulative} control matrix of a sequence of $G$ total steps up to step $g$ and can thus be viewed as a function defined at discrete points in time, $\mc{M}\gth{g}(\omega)\simeq \mc{M}(\omega;t_0,t_g)$.
Finally, we drop the specifier $\FF\gth{2}$ distinguishing the second- from the first-order filter function for brevity; in this section, we always mean the former.

In the following, we will consider a sequence of piecewise-constant time steps split up into subsequences and will deal on the one hand with quantities that depend exclusively on the internal structure of a subsequence and those that do not depend on the internal structure on the other.
For the former, we will denote their internal time step by a parenthesised superscript, \eg $A\gth{i}$, which means the $i$th time step of $A$.
The latter will have no superscript as they do not depend on the internal structure.
We will furthermore distinguish between \emph{local} quantities, which do not depend on the preceding dynamics (that is, are functions of the subsequence alone), and denote the subsequence index they belong to by a parenthesised subscript, \eg $A_{(i)}$ for some quantity $A$ of sequence $i$.
By contrast, quantities which are \emph{non-local} and thus depend on the preceding dynamics will then be denoted in a \enquote{posterior} fashion, \eg, $A_{(i|i-1\to 1)}$ if $A$ is a function of subsequence $i$ and depends on all previous subsequences $i-1, \dotsc, 1$.
A special case is the control propagator up to sequence $i$, $\liouvQ_{(i-1\to 1)}$, where we drop the index $i$ because the action of $\liouvQ_{(i-1\to 1)}$ during $i$ is the identity operation.

We start from \cref{eq:ff:frequency_shifts:freq}, from which for reasons that will become clear shortly we define the second-order filter function by
\begin{subequations}\label{eq:app:ff:filter_function:complete}
\begin{equation}\tag{\ref{eq:app:ff:filter_function:complete}}
    \FF_{\alpha\beta,kl}(\omega) \coloneqq \mc{N}_{\alpha\beta,kl}(\omega) + \sum_{g=1}^{G}\mc{J}_{\alpha\beta,kl}\gth{g}(\omega)
\end{equation}
with
\begin{align}
    \mc{N}_{\alpha\beta,kl}(\omega) &\coloneqq \sum_{g=1}^{G}\mc{G}_{\alpha k}^{(g)\ast}(\omega) \mc{M}_{\beta l}\gth{g-1}(\omega) \label{eq:app:ff:complete_time_step} \\
    \mc{J}_{\alpha\beta,kl}\gth{g}(\omega) &\coloneqq s_\alpha\gth{g}\bar{B}_{\alpha,ij}\gth{g}\bar{C}_{k,ji}\gth{g} I_{ijmn}\gth{g}(\omega)\bar{C}_{l,nm}\gth{g}\bar{B}_{\beta,mn}\gth{g} s_\beta\gth{g} \label{eq:app:ff:nested_time_step}
\end{align}
\end{subequations}
and where repeated indices are contracted.
Comparing to the nested time integral, the first summand in the brackets contains all contributions from complete time segments up to the one containing the inner integration variable $t$, whereas the second captures the final, incomplete segment with $t_{g-1} < t \leq t_{g}$.
Now imagine the sequence of piecewise-constant time steps, $g\in\{1,\dotsc,G\}$, being split apart at some index $1<\gamma<G$ and thereby being divided into two subsequences $g\in\{1,\dotsc,\gamma\}$ and $h\in\{1,\dotsc,\eta\} \equiv g\in\{\gamma+1,\dotsc,G\}$ with $G = \gamma + \eta$.
Our goal is to obtain an expression for the second-order filter function $\FF(\omega)$ that is -- as much as possible -- a sum of local terms of these subsequences $g$ and $h$.

Up to $\gamma$, the filter function is simply that of the first sequence, $\FF_{(1)}(\omega)$, and we thus have for $g>\gamma$\sidenote{
    We drop indices for legibility as stated above; $\mc{J}\gth{g}(\omega)$ is a matrix on Liouville space, whereas $\mc{G}\gth{g}(\omega)$ and $\mc{M}\gth{g}(\omega)$ are Liouville-space row vectors and their product here is an outer product, $\dop*{\mc{G}\gth{g}(\omega)}{\mc{M}\gth{g-1}(\omega)}$, resulting in a matrix on Liouville space.
}
\begin{align}
    \FF_{(2|1)}(\omega) &= \FF(\omega) - \FF_{(1)}(\omega) \notag \\
                        &= \sum_{g=\gamma+1}^{G}\left[
                            \dop*{\mc{G}\gth{g}(\omega)}{\mc{M}\gth{g-1}(\omega)} + \mc{J}\gth{g}(\omega)
                        \right],\label{eq:app:ff:filter_function:12:2}
\end{align}
where we already plugged in \cref{eq:app:ff:complete_time_step}.
We must now express the quantities $\mc{G}\gth{g}(\omega)$, $\mc{M}\gth{g-1}(\omega)$, and $\mc{J}\gth{g}(\omega)$ locally in terms of the index $h$.
To this end, we first write down the step-wise control matrix $\mc{G}\gth{g}(\omega)$ in the second sequence and split off phases and propagators from the first sequence,
\begin{align}\label{eq:app:ff:ctrlmat_step:12}
    \dbra*{\mc{G}\gth{g}(\omega)} &= \e^{\i\omega t_{g}}\dbra*{\ctrlmat\gth{g}(\omega)}\liouvQ\gth{g-1} \notag \\
                                  &= \e^{\i\omega (t_\gamma + t_{h})}\dbra*{\ctrlmat_{(2)}\gth{h}(\omega)}\liouvQ_{(2)}\gth{h-1}\liouvQ_{(1)} \notag \\
                                  &= \e^{\i\omega\tau_{(1)}}\dbra*{\mc{G}_{(2)}\gth{h}(\omega)}\liouvQ_{(1)} \notag \\
                                  &= \dbra*{\mc{G}_{(2|1)}\gth{h}(\omega)}
\end{align}
where $\tau_{(1)}=t_{\gamma}$ and $\liouvQ_{(1)}$ is the control superpropagator of sequence (1).
Next, we consider the cumulative control matrix $\mc{M}\gth{g-1}(\omega)$.
Because in the total sequence it is given by the sum over all $\mc{G}\gth{g^{\prime}}(\omega)$ up to $g-1$, we can split off the complete control matrix of the first sequence and express the remainder by summing over $\mc{G}\gth{h}_{(2|1)}(\omega)$ from \cref{eq:app:ff:ctrlmat_step:12}:
\begin{align}
    \dbra*{\mc{M}\gth{g-1}(\omega)} &= \dbra*{\ctrlmat_{(1)}(\omega)} + \e^{\i\omega\tau_{(1)}}\sum_{h^{\prime}=1}^{g-1-\gamma}
                                        \dbra*{\mc{G}_{(2)}\gth{h^{\prime}}(\omega)}\liouvQ_{(1)} \notag \\
                                    &= \dbra*{\ctrlmat_{(1)}(\omega)} + \e^{\i\omega\tau_{(1)}}\dbra*{\mc{M}_{(2)}\gth{g-1-\gamma}(\omega)}\liouvQ_{(1)} \notag \\
                                    &= \dbra*{\mc{M}_{(2|1)}\gth{h-1}(\omega)}. \label{eq:app:ff:ctrlmat_cumulative:12:2}
\end{align}
Finally, we need to unravel $\mc{J}\gth{g}(\omega)$.
We start from \cref{eq:ff:control_matrix:pulse:freq:ff:calculation}, consider a time step $g\geq\gamma$ in the second sequence with $h = g-\gamma$, and rewrite
\begin{align}
        \mc{J}_{kl}\gth{g}(\omega) =& s\gth{g} \bar{B}_{ij}\gth{g} \bar{C}_{kji}\gth{g} I_{ijmn}\gth{g}(\omega)
                                       \bar{C}_{(2|1),lnm}\gth{g} \bar{B}_{mn}\gth{g} s\gth{g} \notag \\
                                   =& s_{(2)}\gth{h} \bar{B}_{(2),ij}\gth{h} \bar{C}_{(2|1),kji}\gth{h} I_{(2),ijmn}\gth{h}(\omega)
                                       \bar{C}_{(2|1),lnm}\gth{h} \bar{B}_{(2),mn}\gth{h} s_{(2)}\gth{h} \notag \\
                                   =& \mc{J}_{(2|1),kl}\gth{h}(\omega) \label{eq:app:ff:filter_function:incomplete_timestep}
\end{align}
because all quantities except for $\bar{C}_{(2|1)}\gth{g}$ depend on their timestep $g$ alone, and where $i,j,m,n$ index the Hilbert space dimensions of the operators, while $k,l$ are the usual indices for the basis elements and therefore Liouville space dimensions.
On that term, we can factor out the propagators of the first complete sequence,\sidenote{
    Note that the $Q_{(i)}$ here are Hilbert space propagators, not their Liouville space counter parts $\liouvQ_{(i)}$, and that $Q_{(1)}\gth{h-1}\equiv Q_{h-1}$ in the notation of \cref{subsec:ff:theory:control_matrix:sequence}.
}
\begin{align}
    \bar{C}_{(2|1),kij}\gth{h} = \left[V_{(2)}^{(h)\dagger}Q_{(2)}\gth{h-1}Q_{(1)} C_k Q_{(1)}\adjoint Q_{(2)}^{(h-1)\dagger}V_{(2)}\gth{h}\right]_{ij}.
\end{align}

We can now finally put all pieces together and, starting from \cref{eq:app:ff:filter_function:12:2}, plug in \cref{eq:app:ff:ctrlmat_step:12,eq:app:ff:ctrlmat_cumulative:12:2,eq:app:ff:filter_function:incomplete_timestep}, so that we obtain\sidenote{
    Recall that \liouvQ is the Liouville representation of the unitary operator $Q$ and as such -- and because our chosen basis \basis is Hermitian -- is an orthogonal matrix for which $\liouvQ\transpose\liouvQ = \eye$.
}
\begin{align}
    \FF_{(2|1)}(\omega) = \sum_{h=1}^{\eta}\Bigl[ & \dop*{\mc{G}_{(2|1)}\gth{h}(\omega)}{\mc{M}_{(2|1)}\gth{h-1}(\omega)}
                                                    + \mc{J}_{(2|1)}\gth{h}(\omega)\Bigr] \notag \\
                        = \sum_{h=1}^{\eta}\Bigl\lbrace
                               & \e^{-\i\omega\tau_{(1)}}\liouvQ_{(1)}\transpose\dket*{\mc{G}_{(2)}\gth{h}(\omega)} \\
                               & \times\Bigl[
                                     \dbra*{\ctrlmat_{(1)}(\omega)} + \e^{\i\omega\tau_{(1)}}\dbra*{\mc{M}_{(2)}\gth{g-1-\gamma}(\omega)}\liouvQ_{(1)}
                                 \Bigr] +\mc{J}_{(2|1)}\gth{h}(\omega)
                        \Bigr\rbrace. \notag
\end{align}
To simplify the unwieldy first summand in the curly braces further, we expand the product,
\begin{align}
    \MoveEqLeft \e^{-\i\omega\tau_{(1)}}\liouvQ_{(1)}\transpose\dket*{\mc{G}_{(2)}\gth{h}(\omega)}\left[\dbra*{\ctrlmat_{(1)}(\omega)}
            + \e^{\i\omega\tau_{(1)}}\dbra*{\mc{M}_{(2)}\gth{h-1}(\omega)}\liouvQ_{(1)}\right] \\
        &= \e^{-\i\omega\tau_{(1)}}\liouvQ_{(1)}\transpose\dop*{\mc{G}_{(2)}\gth{h}(\omega)}{\ctrlmat_{(1)}(\omega)}
            + \liouvQ_{(1)}\transpose\dop*{\mc{G}_{(2)}\gth{h}(\omega)}{\mc{M}_{(2)}\gth{h-1}}\liouvQ_{(1)}. \notag
\end{align}
If we now pull in the sum over the time steps $h$, we can identify the control matrix in the first term and the contribution from complete segments to the second-order filter function ($\mc{N}_{(2)}(\omega)$, \cref{eq:app:ff:complete_time_step}) in the second,
\begin{align}
    \MoveEqLeft \sum_{h=1}^{\eta}\e^{-\i\omega\tau_{(1)}}\liouvQ_{(1)}\transpose\dop*{\mc{G}_{(2)}\gth{h}(\omega)}{\ctrlmat_{(1)}(\omega)}
            + \liouvQ_{(1)}\transpose\dop*{\mc{G}_{(2)}\gth{h}(\omega)}{\mc{M}_{(2)}\gth{h-1}}\liouvQ_{(1)} \notag \\
        &= \e^{-\i\omega\tau_{(1)}}\liouvQ_{(1)}\transpose\dop*{\ctrlmat_{(2)}(\omega)}{\ctrlmat_{(1)}(\omega)}
            + \liouvQ_{(1)}\transpose\mc{N}_{(2)}(\omega)\liouvQ_{(1)}.
\end{align}
As a last step, we recognize that the bra in the first term is nothing else but \cref{eq:app:ff:ctrlmat_step} so that we can write the filter function succinctly as
\begin{align}
    \FF_{(2|1)}(\omega) &= \dop*{\mc{G}_{(2)}(\omega)}{\ctrlmat_{(1)}(\omega)}
                            + \liouvQ_{(1)}\transpose\mc{N}_{(2)}(\omega)\liouvQ_{(1)}
                            + \sum_{h=1}^{\eta}\mc{J}_{(2|1)}\gth{h}(\omega) \notag \\
                        &= \mc{N}_{(2|1)}(\omega) + \sum_{h=1}^{\eta}\mc{J}_{(2|1)}\gth{h}(\omega). \label{eq:app:ff:filter_function:12}
\end{align}
% TODO: here
In \cref{eq:app:ff:filter_function:12}, all terms except the last are known ahead of time if the first- and second-order filter functions of the subsequences as well as the control matrix of the concatenated sequence have been computed.
We can extend this result to sequences consisting of an arbitrary number of $G$ subsequences with lengths $\lbrace\eta_g\rbrace_{g=1}^G$ by recursively shifting indices in \cref{eq:app:ff:filter_function:12}, $(2)\to (3), (1)\to (2)$, and adding $\FF_{(2|1)}(\omega)$, allowing us to write the concatenation rule for second-order filter functions as
\begin{subequations}\label{eq:app:ff:filter_function:concatenated}
\begin{align}
    \FF(\omega) =& \sum_{g=1}^{G} \FF_{(g|g-1\to 1)}(\omega) \notag \\
                =& \sum_{g=1}^{G}\left[
                        \mc{N}_{(g|g-1\to 1)}(\omega)
                        + \sum_{h_{g}=1}^{\eta_{g}}\mc{J}_{(g|g-1\to 1)}\gth{h_{g}}(\omega).
                    \right] \tag{\ref{eq:app:ff:filter_function:concatenated}}
\end{align}
with
\begin{multline}
    \mc{N}_{(g|g-1\to 1)}(\omega) \\
        = \dop*{\mc{G}_{(g)}(\omega)}{\ctrlmat_{(g-1)}(\omega)} + \liouvQ_{(g-1\to 1)}\transpose\mc{N}_{(g)}(\omega)\liouvQ_{(g-1\to 1)}
\end{multline}
and
\begin{multline}
    \mc{J}_{(g|g-1\to 1),kl}\gth{h}(\omega) \\
        = s_{(g)}\gth{h} \bar{B}_{(g),ij}\gth{h} \bar{C}_{(g|g-1\to 1),kji}\gth{h} I_{(g),ijmn}\gth{h}(\omega)
           \bar{C}_{(g|g-1\to 1),lnm}\gth{h} \bar{B}_{(g),mn}\gth{h} s_{(g)}\gth{h}
\end{multline}
\end{subequations}
\Cref{eq:app:ff:filter_function:concatenated} is our final result.
Before we analyze it in more detail, let us first briefly discuss the special case where $G=1$.
Then, $\liouvQ_{(0)}=\eye$, $\dbra*{\ctrlmat_{(0)}} = 0$, and hence $\dop*{\mc{G}_{(1)}(\omega)}{\ctrlmat_{(0)}(\omega)} = 0$ so that \cref{eq:app:ff:filter_function:12} reduces to \cref{eq:app:ff:filter_function:complete} as it should.
