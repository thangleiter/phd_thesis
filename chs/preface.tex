% mainfile: ../main.tex
\chapter*{Preface}
\addcontentsline{toc}{chapter}{Preface} % Add the preface to the table of contents as a chapter

\Thethesis consists of four parts.
Each part is a largely self-contained accounting of one aspect of my work during the last years.
Unfortunately, not everything always goes according to plan, and so there is not the \emph{one} overarching theme that connects all parts.
Yet it is in the nature of physics that there are always points of contact and connections.

One of these is \emph{noise}, which I have dealt with extensively from various different points of view.
In \cref{part:speck}, I present a software package to facilitate noise spectroscopy as an everyday tool in physics labs.
The experiments that we conduct as condensed-matter physicists suffer from many kinds of noise, and this tool is intended to help analyze it.
That is, given the \emph{system}, what is the noise and how can we quantify it?
In \cref{part:setup}, I then analyze, characterize, and improve upon a experimental setup designed to perform confocal optical spectroscopy of semiconductor nanostructures at Millikelvin temperatures.
There, the topic of noise enters as an undesired but unavoidable external influence, and I apply the tools developed in \cref{part:speck} to characterize and mitigate the noise in order to improve the experimental capabilities of the setup.
That is, given the \emph{noise in the current state of the system}, how can we modify the system in order to ameliorate it?
In \cref{part:ff}, I develop a theoretical formalism to describe noise and its influence on the coherent evolution of single quantum systems.
Also there, of course, the aim is to reduce the influence of noise, but in quite a different way.
There, I deal with modeling the noise and the way it interacts with the intended dynamics of a controlled quantum system so as to understand and come up with ways of avoiding perturbations induced by the noise.
That is, given the \emph{noise}, how can we predict how the system reacts to it, and how might we be able to control the system in a way that is robust to noise?

Another recurring theme of \thethesis is \emph{research software}.
As physicists, much of our work -- be it experimental or theoretical -- relies on computers doing some aspect of it for us.
Developing, and sharing, this software has thus become an integral part of physics research, and in \thethesis I contribute to the open-source research software ecosystem.
In \cref{part:speck}, I introduce the \pyspeck software package already mentioned above, which implements hardware-agnostic data acquisition, processing, and visualization for Fourier-transform noise spectroscopy.
In \cref{part:exp}, I present optical measurements of electrostatic exciton traps in the Millikelvin confocal microscope discussed in \cref{part:setup}.
These measurements are facilitated by the \mjolnir measurement framework, which abstracts the complex hardware stack required to perform the measurements, allowing for minimal coding overhead going from the concept of a measurement to its execution.
In \cref{part:ff}, I introduce the \filterfunctions software package, which implements the formalism developed in the same part, enabling easy adoption of the techniques for the computation of noisy quantum processes.

\Thethesis has been typeset with Lua\LaTeX{} and is optimized for digital reading.
The document source code is available on \href{https://github.com/thangleiter/phd_thesis/}{Github}.
Throughout the document, I make liberal use of cross-references and acronyms, the latter of which link to and are defined in the \hyperref[glo]{Glossary} on page~\pageref{glo}.
As such, I recommend a PDF viewer that can preview the targets of internal hyperlinks for reading \thethesis in order to avoid having to jump back and forth within the document.

The various figures included in this document are either generated by \LaTeX{}/Ti\emph{k}Z or by a Python script included in the source repository in the \code{img/py} directory.
Figure captions contain a hyperlink to the list of \hyperref[lof]{figure source files and parameters} on page~\pageref{lof} whose entries point to the source file that generates the figure.
For figures that contain experimental data, the entry in the list of \hyperref[lof]{figure source files and parameters} on page~\pageref{lof} furthermore typically contains values of external parameters that were fixed during the measurement.
In the spirit of open data, all data discussed in \thethesis are included in the \code{data} directory at the top level of the source repository.

Finally, I frequently employ an $\asinh$-scale for data that spans several orders of magnitude.
This scaling is asymptotically equivalent to a symmetric logarithmic scale for large $\abs{x}$, $\asinh(\pm x)\sim\pm\log\abs{x}$, but avoids the divergence close to zero, instead being linear, $\asinh(x)\sim x$.
The crossover from linear to logarithmic behavior is governed by a parameter $x_0$ such that $x\to x_0\asinh(\flatfrac{x}{x_0})$ and that can be freely chosen, thus influencing the representation of the data.
The particular value of $x_0$ (corresponding to the \code{linear_width} parameter in \href{https://matplotlib.org/stable/gallery/scales/asinh_demo.html}{\matplotlib}) in a given figure can be looked up in the script file that generates it.

\begin{flushright}
    \itshape
    Tobias Hangleiter\\
    Aachen, September 2025
\end{flushright}
