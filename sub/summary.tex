\documentclass[a4paper,10pt]{article} 
\usepackage{fontenc}
\usepackage{polyglossia}
\setmainlanguage{english}
\setotherlanguage{german}
\usepackage{chemformula}
\usepackage{siunitx} 
\usepackage{geometry} % To adjust margins

% Adjust margins to fit everything on one page
%\geometry{
%    left=1cm,
%    right=1cm,
%    top=1cm,
%    bottom=1cm,
%}

\begin{document}

\section*{Summary}
Quantum technology promises to be a paradigm shift in the precision, speed, and breadth of application of technology.
Recent advances in the fields of quantum computing, quantum networks, and quantum sensing have been rapid, but large-scale utilization is still out of reach.
This is largely due to the fragility of the individual quantum systems that this technology seeks to control and manipulate.
While small demonstrators have by now shown these novel machines to work in principle, the task in the coming years will be to make them reproducible at scale.
Leveraging the advanced semiconductor fabrication technology of today's classical computers, spin qubits in semiconductors such as \ch{Si/SiGe} or \ch{GaAs/AlGaAs} are hoped to meet this challenge.
In this thesis, I present contributions in four different areas towards this overarching goal.
I present an open-source software tool for Fourier-transform noise spectroscopy, improvements and characterization of a Millikelvin confocal microscope, optical measurements of electrostatic exciton traps, and a general filter-function formalism for the description of noisy quantum dynamics.
I review the theory of noise spectroscopy based on Welch's method of overlapping periodograms and lay out the design of the hardware-agnostic, open-source \texttt{python\_spectrometer} software package that facilitates interactive noise spectroscopy in everyday use.
In a didactic fashion, I explore the features and capabilities of the tool and guide the reader through its operation.
I then characterize a confocal microscope incorporated in a cryogen-free dilution refrigerator, applying the aforementioned noise spectroscopy tool to evaluate the displacement noise power spectral density of the system.
I develop an optical technique based on a knife-edge reflectance contrast to measure the displacement noise reaching a shot-noise floor of $\qty{1}{\nano\meter\per\sqrt{\hertz}}$.
Using this technique, I determine a displacement noise RMS of \qty{100}{\nano\meter}.
I measure the electron temperature of a \ch{GaAs/AlGaAs} quantum dot in the microscope, finding \qty{76}{\milli\kelvin}, and demonstrate at hand of a second-order correlation measurement of a self-assembled quantum dot the setup's optical capabilities.
In the experimental part of this thesis, I first provide an introduction to photoluminescence and the quantum-confined Stark effect in semiconductor quantum wells, appealing to simple analytical theory to develop an intuition for the relevant effects.
I then describe the \texttt{mjolnir} measurement framework developed to conduct optical measurements of electrostatic exciton traps in semiconductor membranes, and present extensive measurements in search of signatures of single-photon emitters.
Finally, I develop a comprehensive theoretical formalism for describing and analyzing the noisy quantum dynamics of quantum systems based on filter functions.
I combine the Magnus and cumulant expansions to derive a quantum-operations formulation of the dynamics under arbitrary classical noise that can be expressed in terms of filter functions capturing the susceptibility of a quantum system in response to noise at a given frequency.
I present the \texttt{filter\_functions} software package that implements the formalism and demonstrate its efficacy using several examples.

\clearpage
\section*{Zusammenfassung}
\addcontentsline{toc}{chapter}{Zusammenfassung}
\begin{german}
    Quantentechnologie verspricht einen Paradigmenwechsel in Bezug auf Präzision, Geschwindigkeit und Anwendungsvielfalt von Technologie.
    Die jüngsten Fortschritte in Quantencomputing, Quantennetzwerken und Quantensensorik waren rasant, doch eine großflächige Nutzung ist noch nicht in Sicht.
    Dies liegt hauptsächlich an der Fragilität der einzelnen Quantensysteme, die diese Technologie zu steuern versucht.
    Während kleine Demonstratoren gezeigt haben, dass diese Maschinen im Prinzip funktionieren, besteht die Aufgabe darin, sie in großem Maßstab reproduzierbar zu machen.
    Unter Nutzung fortgeschrittener Halbleiterfertigungstechnologie heutiger klassischer Computer soll diese Herausforderung mit Spin-Qubits in Halbleitern wie \ch{Si/SiGe} oder \ch{GaAs/AlGaAs} bewältigt werden.
    In dieser Arbeit stelle ich Beiträge in vier Bereichen in Richtung dieses Ziels vor.
    Ich präsentiere ein Open-Source-Softwaretool für die Fourier-Transform-Rauschspektroskopie, Verbesserungen eines Millikelvin-Konfokalmikroskops, optische Messungen elektrostatischer Exzitonenfallen und einen allgemeinen Filterfunktionsformalismus zur Beschreibung rauschbehafteter Quantendynamik.
    Ich gebe einen Überblick über die Theorie der Rauschspektroskopie auf der Grundlage von Welchs Methode und stelle das Design des hardwareunabhängigen Open-Source-Pakets \texttt{python\_spectrometer} vor, das interaktive Rauschspektroskopie im täglichen Gebrauch erleichtert.
    Auf didaktische Weise untersuche ich die Funktionen des Tools und führe den Leser durch dessen Bedienung.
    Anschließend charakterisiere ich ein konfokales Mikroskop, das in einen kryogenfreien Mischungskryostaten integriert ist, und wende das oben genannte Tool an, um die spektrale Leistungsdichte des Positionsrauschens zu messen.
    Ich entwickle eine Technik auf der Grundlage eines Messerkanten-Reflexionskontrasts, um das Positionsrauschen zu messen, das einen Schrotrauschuntergrund von $\qty{1}{\nano\meter\per\sqrt{\hertz}}$ erreicht.
    Mit dieser Technik bestimme ich ein Positionsrauschen (RMS) von \qty{100}{\nano\meter}.
    Ich messe die Elektronentemperatur eines \ch{GaAs/AlGaAs}-Quantenpunkts im Mikroskop und erhalte einen Wert von \qty{76}{\milli\kelvin}.
    Anhand einer Korrelationsmessung zweiter Ordnung eines selbstorganisierten Quantenpunkts demonstriere ich die optischen Fähigkeiten des Aufbaus.
    Im experimentellen Teil gebe ich zunächst eine Einführung in Photolumineszenz und den quantenbegrenzten Stark-Effekt in Halbleiter-Quantentöpfen und stütze mich dabei auf einfache analytische Theorie.
    Anschließend beschreibe ich den \texttt{mjolnir}-Messrahmen für optische Messungen von elektrostatischen Exzitonfallen in Halbleitermembranen und präsentiere umfangreiche Messungen auf der Suche nach Signaturen von Einzelphotonenemittern.
    Schließlich entwickle ich einen theoretischen Formalismus zur Beschreibung der verrauschten Quantendynamik von Quantensystemen auf der Grundlage von Filterfunktionen.
    Ich kombiniere Magnus- und Kumulanten-Entwicklungen zur Ableitung einer Quantenoperationsformulierung der Dynamik unter klassischem Rauschen, ausgedrückt durch Filterfunktionen.
\end{german}

\end{document}
