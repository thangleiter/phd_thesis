% mainfile: ../../main.tex
\setchapterpreamble[u]{\margintoc}
\chapter{Supplementary to Part \ref{part:exp}: Additional measurements and \texorpdfstring{\acrshort{tmm}}{TMM} simulations}\label{ch:app:exp}
In this appendix, I give additional information on the bandstructure simulations and show further measurements supporting the interpretation of data shown in \cref{ch:exp:observations}.
Moreover, I investiate the influence of the epoxy thickness on the simulations performed in \cref{sec:exp:tmm}.

\section{Self-consistent Poisson-Schrödinger simulation of the membrane band structure}\label{sec:app:exp:observations:ps}
\begin{margintable}
    \centering
    \footnotesize
    \caption{
        Heterostructure parameters used in \thethesis.
        The doping density $N_{\mr{d}}$ is nominal, whereas the charge carrier density in the \gls{qw}, $n$, is computed using the nominal doping values with parameters given in \cref{tab:app:exp:samples:ps}.
    }
    \label{tab:app:exp:samples}
    \begin{tabularx}{\marginparwidth}{@{} l @{} S[table-alignment=right, table-format=1.1e1] S[table-alignment=right, table-format=1.2e2] @{}}
        \toprule
        \textsc{Wafer}          & {$N_{\mr{d}}$ (\unit{\per\cubic\centi\meter})} & {$n$ (\unit{\per\square\centi\meter})} \\
        \midrule
        \textsc{M1\_05\_49}     & 6.5e17                                         & 1.95e11 \\
        \textsc{15460 (Honey)}  & 1.8e18                                         & 4.26e11 \\
        \textsc{15271 (Fig)}    & 8.0e17                                         & 3.91e11 \\
        \bottomrule
    \end{tabularx}
\end{margintable}
\begin{margintable}
    \centering
    \footnotesize
    \begin{threeparttable}
        \caption{
            Simulation parameters used to compute the charge carrier density $n$ in \cref{tab:app:exp:samples}.
        }
        \label{tab:app:exp:samples:ps}
        \begin{tabularx}{\marginparwidth}{c S l}
            \toprule
            \textsc{Parameter}              & {\textsc{Value}}  & \textsc{Unit} \\
            \midrule
            $E_{\mr{DX}}$\tnote{a}          & -71.5             & \unit{meV} \\
            $\Delta E_{\mr{c}}$\tnote{b}    & 240               & \unit{meV} \\
            $m_{\mr{c}}$                    & 0.067             & $m_e$ \\
            $\Delta z$                      & 0.5               & \unit{nm} \\
            $T$                             & 10                & \unit{mK} \\
            $V_{\mr{FP}}$\tnote{c}          & 0.76              & \unit{V} \\
            \bottomrule
        \end{tabularx}
        \begin{tablenotes}
            \scriptsize
            \item[a] Energy of the DX-center below the Fermi level.
            \item[b] Conduction band offset at the \ch{GaAs/AlGaAs} interface.
            \item[c] Fermi level pinning voltage.
        \end{tablenotes}
    \end{threeparttable}
\end{margintable}

The samples studied in \cref{ch:exp:observations} were all grown using \gls{mbe} and had similar nominal designs.
To compare the design charge carrier density with the measurements obtained from \gls{pl} of the Fermi edge, I simulated the band structure using a self-consistent Poisson-Schrödinger solver~\cite{PoissonSchroedinger}.
\Cref{tab:app:exp:samples} shows the nominal doping concentration $N_{\mr{d}}$ as well as the charge carrier density in the \gls{qw}, $n$, obtained from a simulation using the parameters given in \cref{tab:app:exp:samples:ps}.

\section{Additional data}\label{sec:app:exp:observations}
\subsection{Combined plot of \texorpdfstring{\acrshort{pl}}{PL} and \texorpdfstring{\acrshort{ple}}{PLE} data}\label{subsec:app:exp:observations:meas:pl_ple}
\begin{figure}
    \centering
    \includegraphics{img/pdf/experiment/doped_M1_05_49-2_ple_single}
    \caption[
        \sampleid{Doped M1_05_49-2}
        \thevoltage{-1.3}{CM}
        \thepower{1}{\micro}
        \protect\newline
        \imgsource{img/py/experiment/ple.py}
    ]{
        Combined plot of the \gls{pl} with excitation at \qty{795}{\nano\meter} (green) and \gls{ple} (magenta).
        The data overlap in the range of \qtyrange{801}{825}{\nano\meter}, where they are plotted with \qty{50}{\percent} transparency.
        The onset of the absorption edge in \gls{ple} coincides with the high-energy shoulder of the \gls{pl} emission.
    }
    \label{fig:app:exp:pl:doped_M1_05_49-2_ple}
\end{figure}

In \cref{sec:exp:observations:ple}, I showed \gls{ple} measurements of a large exciton trap.
When integrating the \gls{pl} spectrum over the detection energy, the data showed several distinct features including an onset of the absorption that depends on the electric field applied by the difference-mode voltage \VDM.
\Cref{fig:app:exp:pl:doped_M1_05_49-2_ple} shows the same data as panels (a) and (b) of \cref{fig:exp:pl:doped_M1_05_49-2_ple}, but drawn in the same plot.
It is clear that the onset of absorption occurs just as the \gls{pl} emission drops off, indicating a small Stokes shift.

\subsection{\texorpdfstring{\acrshort{2deg}}{2DEG} \texorpdfstring{\acrshort{pl}}{PL} as function of power}\label{subsec:app:exp:observations:meas:2deg}
\begin{figure}
    \centering
    \includegraphics{img/pdf/experiment/2deg_pl_power_dependence}
    \caption[
        \sampleid{Doped M1_05_49-2}
        \thewavelength{795}
        \protect\newline
        \imgsource{img/py/experiment/pl.py}
    ]{
        \Gls{2deg} \gls{pl} of an unbiased \gls{qw} as function of excitation power.
        The Fermi edge at \qty{1.5275}{\electronvolt} does not broaden over two orders of magnitude as the line cuts in the upper panels also show.
    }
    \label{fig:app:exp:observations:2deg_pl_power_dependence}
\end{figure}

In \cref{sec:exp:observations:pl}, it was determined that the electrons recombining from the Fermi edge of the \gls{2deg} were much hotter at $\sim\qty{1}{\kelvin}$ than the cryostat temperature at $\sim\qty{10}{\milli\kelvin}$.
The fit to the trion-exciton lineshape in \cref{sec:exp:observations:ple} yielded similar results.
\Cref{fig:app:exp:observations:2deg_pl_power_dependence} demonstrates that this is not a local heating of the lattice due to excess laser power.
Shown is the \gls{2deg} \gls{pl} as function of excitation power from \qtyrange{5}{500}{\nano\watt}, with the upper panel showing line cuts at highest and lowest power.
There is no discernible broadening in the Fermi edge on the high-energy side of the spectrum.

\section{Dependence of \texorpdfstring{\acrshort{tmm}}{TMM} simulations on epoxy thickness}\label{sec:app:exp:tmm}
\begin{marginfigure}
    \centering
    \includegraphics{img/pdf/experiment/reflectance_epoxy}
    \caption[\imgsource{img/py/experiment/tmm.py}]{
        Reflectance (top) and \gls{qw} absorptance (bottom) as function of epoxy thickness assuming coherent back-scattering.
        The period corresponds to the wavelength in epoxy, $\lambda_{\mr{epo}} = \lambda_0/n_{\mr{epo}}$.
    }
    \label{fig:app:exp:tmm:epoxy}
\end{marginfigure}

In \cref{sec:exp:tmm}, I carried out \gls{tmm} simulations of the membrane structure to elucidate the quenching of \gls{pl} intensity when focusing the laser on gates on the top or bottom side of the membrane.
There, I assumed incoherent scattering of photons at the epoxy/\ch{Si} interface below the membrane.
This assumption is based on the fact that \citet{Descamps2021} found the epoxy thickness, on the order of a few \unit{\micro\meter}, to vary significantly across the chip.
If the variation is fast enough, we can expect the phase of back-scattered photons to average out and hence destroy coherence.
Nonetheless, we should estimate the influence of coherent back-scattering.
\Cref{fig:app:exp:tmm:epoxy} shows the reflectance \reflectance and \gls{qw} absorptance \absorptance as function of the epoxy thickness.
The dashed line indicates the value resulting from the incoherent simulation.
Both quantities vary significantly with the epoxy thickness, but the incoherent values $\absorptance_\infty$ and $\reflectance_\infty$ are close to the average.
From the fact that neither the \gls{pl} nor the reflected laser intensity varies spatially by such large amounts in experiments, we can conclude that either the thickness variation is small enough to not play a significant role or that it varies fast enough to validate the assumption of incoherent back-scattering.
