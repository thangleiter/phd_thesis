% mainfile: main.tex
\chapter{Theory of spectral noise estimation}\label{ch:speck:theory}
There exists a multitude of methods for estimating noise properties.
\todo{lay out some others}

If the noisy process $x(t)$\sidenote{
    We discuss only classical noise here, meaning $x(t)$ commutes with itself at all times. For descriptions of and spectroscopy protocols for quantum noise refer to \citerr{Clerk2010}{Paz-Silva2017}, for example.
}
has Gaussian statistics, meaning that the value at a given point in time follows a normal distribution with some mean $\mu$ and variance $\sigma^2$ over multiple realizations of the process, it can be fully described by the \gls{psd} $S(\omega)$.\sidenote{
    The term \emph{power spectrum} is often used interchangably. I will do so as well, but emphasize at this point that in digital signal processing in particular, the \emph{spectrum} is a different quantity from the \emph{spectral density}.
}
\todo{maybe a classical signal processing ref?}
For the purpose of noise estimation, the assumption of Gaussianity is a rather weak one as the noise typically arises from a large ensemble of individual fluctuators and is therefore well approximated by a Gaussian distribution by the central limit theorem.\sidenote{
    As an example, consider electronic devices, where voltage noise arises from a large number of defects and other charge traps in oxides being populated and depopulated at certain rates $\gamma$. The ensemble average over these so-called \glspl{tlf} then yields the well-known \oneoverf-like noise spectra (at least for a large density~\cite{Mehmandoost2024}).
}
\todo{flesh out this sidenote?}
Even if the process $x(t)$ is not perfectly Gaussian, non-Gaussian contributions can be seen as higher-order contributions if viewed from the perspective of perturbation theory, and therefore the \gls{psd} still captures a significant part of the statistical properties.
For this reason, the \gls{psd} is the central quantity of interest in noise spectroscopy and I will discuss some of its properties in the following.

For real signals $x(t) \in\mathbb{R}$, $S(\omega)$ is an even function and one therefore distinguishes the \emph{two-sided} \gls{psd} $S^{(2)}(\omega)$ defined over $\mathbb{R}$ from the \emph{one-sided} \gls{psd} $S^{(1)}(\omega) = 2 S^{(2)}(\omega)$ defined only over $\mathbb{R}^+$.
Complex signals $x(t)\in\mathbb{C}$ such as those generated by Lock-in amplifiers after demodulation in turn have asymmetric, two-sided \glspl{psd}.
\todo{flesh out}

\section{Spectrum estimation from time series}\label{sec:speck:theory:time_series_estimation}
To see how the \gls{psd} may be estimated from time-series data, consider a continuous wide-sense stationary\sidenote{
    For a wide-sense stationary (also called weakly stationary) process $x(t)$, the mean is constant and the auto-correlation function $C(t, t') = \ev{x(t)^\ast x(t^\prime)}$ is given by $\ev{x(t)^\ast x(t + \tau)} = \ev{x(0)^\ast x(\tau)}$ with $\tau = t^\prime - t$.
    That is, it is a function of only the time lag $\tau$ and not the absolute point in time.
    For Gaussian processes as discussed here, this also implies stationarity~\cite{Koopmans1995}.
    The property further implies that $C(\tau)$ is an even function.
}
\todo{sketch of auto-correlation function?}
signal in the time domain $x(t)\in\mathbb{C}$ that is observed for some time $T$.
We define the windowed Fourier transform of $x(t)$ and its inverse by\sidenote{
    In this chapter we will always denote the Fourier transform of some quantity $\xi$ using the same symbol with a hat, $\hat{\xi}$.
}
\begin{align}
    \hat{x}_T(\omega) &= \int_{0}^{T}\dd{t} x(t)\e^{\i\omega t} \label{eq:windowed_ft}\\
       \qq*{and} x(t) &= \intinf\ddf{\omega}\hat{x}_T(\omega)\e^{-\i\omega t}, \label{eq:windowed_ft:inverse}
\end{align}
\ie, we assume that outside of the window of observation $x(t)$ is zero.
The auto-correlation function of $x(t)$ is given by
\begin{align}
    C(\tau) &= \expval{x(t)^\ast x(t + \tau)} \label{eq:autocorrelation}\\
            &= \lim_{T\to\infty} \frac{1}{T}\int_0^T\dd{t} x(t)^{\ast} x(t + \tau),
\end{align}
where $\expval{\placeholder}$ is the ensemble average over multiple realizations of the process and the last equality holds true for ergodic processes.
Expressing $x(t)$ in terms of its Fourier representation (\cref{eq:windowed_ft}) and reordering the integrals, we get\sidenote{
    Mathematicians might at this point argue the integrability of $x(t)$, but as we deal with physical processes with finite bandwidth (and have no shame), we do not.
}
\begin{align}
    C(\tau) &= \lim_{T\to\infty}\frac{1}{T}\int_0^T\dd{t}
                \intinf\ddf{\omega}\hat{x}_T(\omega)^{\ast}\e^{\i\omega t}
                \intinf\ddf{\omega^\prime}\hat{x}_T(\omega^\prime)\e^{-\i\omega^\prime (t + \tau)}  \\
            &= \lim_{T\to\infty}\frac{1}{T}\intinf\ddf{\omega}\intinf\ddf{\omega^\prime}
                \hat{x}_T(\omega)^{\ast}\hat{x}_T(\omega^\prime)\e^{-\i\omega^\prime\tau}
                \int_0^T\dd{t}\e^{\i t (\omega - \omega^\prime)} \label{eq:autocorrelation:fourier}
\end{align}
The innermost integral approaches a $\delta$-function for large $T$,\sidenote{
    Note that, because $x(t)$ is wide-sense stationary, we may shift the limits of integration $\int_{0}^{T}\to\int_{-\flatfrac{T}{2}}^{+\flatfrac{T}{2}}$.
}
allowing us to further simplify this under the limit as
\begin{align}
    C(\tau) &= \lim_{T\to\infty} \frac{1}{T}\intinf\ddf{\omega}\intinf\ddf{\omega^\prime}
                \hat{x}_T(\omega)^{\ast}\hat{x}_T(\omega^\prime)
                \e^{-\i\omega^\prime\tau}\delta(\omega - \omega^\prime)\\
            &= \lim_{T\to\infty}\frac{1}{T}
                \intinf\ddf{\omega}\abs{\hat{x}_T(\omega)}^2 \e^{-\i\omega\tau} \\
            &= \intinf\ddf{\omega} S(\omega) \e^{-\i\omega\tau} \label{eq:wiener_khinchin}
\end{align}
with the \gls{psd}
\begin{align}
    S(\omega) &= \lim_{T\to\infty}\frac{1}{T}\abs{\hat{x}_T(\omega)}^2 \label{eq:psd:definition}\\
              &= \intinf\dd{\tau} C(\tau)\e^{\i\omega\tau}
\end{align}
\Cref{eq:wiener_khinchin} is the Wiener-Khinchin theorem that states that the auto-correlation function $C(\tau)$ and the \gls{psd} $S(\omega)$ are Fourier-transform pairs~\cite{Koopmans1995}.
Furthermore, defining the latter through \cref{eq:psd:definition} gives us an intuitive picture of the \gls{psd} if we recall Parseval's theorem,
\begin{align}\label{eq:parseval}
    \intinf\ddf{\omega}\frac{1}{T}\abs{\hat{x}_T(\omega)}^2 = \frac{1}{T}\intinf\dd{t}\abs{x(t)}^2.
\end{align}
That is, the total power $P$ contained in the signal $x(t)$ is given by integrating over the \gls{psd}.
Similarly, the power contained in a band of frequencies $[\omega_1, \omega_2]$ is given by
\begin{align}
    P(\omega_1, \omega_2) &= \rms\left(\omega_1, \omega_2\right)^2 \\
                          &= \int_{\omega_1}^{\omega_2}\ddf{\omega} S(\omega) \label{eq:psd:bandpower}
\end{align}
where $\rms\left(\omega_1, \omega_2\right)$ is the root-mean-square within this frequency band.
These relations are helpful when analyzing noise \glspl{psd} to gauge the relative weight of contributions from different frequency bands to the total noise power.

\Cref{eq:psd:definition} represents the starting point for the experimental spectrum estimation procedure.
Instead of a continuous signal $x(t), t\in [0, T]$, consider its discretized version\sidenote{
    We only discuss the problem of equally spaced samples here. Variants for spectral estimation of time series with unequal spacing exist.
}
\todo{ref}
\begin{align}\label{eq:signal:discrete}
    x_n \qc n\in\lbrace 0, 1, \dotsc, N - 1\rbrace
\end{align}
defined at times $t_n = n\Delta t$ with $T = N\Delta t$ and where $\Delta t = \fs\inverse$ is the sampling interval (the inverse of the sampling frequency \fs).
Invoking the ergodic theorem, we can replace the long-term average in \cref{eq:psd:definition} by the ensemble average over $M$ realizations $i$ of the noisy signal $x_n\gth{\nu}$ and write
\begin{align}\label{eq:psd:bartlett}
    S_n &= \frac{1}{M} \sum_{i=0}^{M-1} \abs{\hat{x}_n\gth{\nu}}^2 \\
        &= \frac{1}{M} \sum_{i=0}^{M-1} S_n\gth{\nu}
\end{align}
where $\hat{x}_n\gth{\nu}$ is the discrete Fourier transform of $x_n\gth{\nu}$, we defined the \emph{periodogram} of $x_n\gth{\nu}$ by
\begin{align}\label{eq:periodogram}
    S_n\gth{\nu} = \abs{\hat{x}_n\gth{\nu}}^2,
\end{align}
and $S_n$ is an \emph{estimate} of the true \gls{psd} sampled at the discrete frequencies $\omega_n = \flatfrac{2\pi n}{T} \in 2\pi\times\lbrace\flatfrac{-\fs}{2}, \dotsc, \flatfrac{\fs}{2}\rbrace$.\sidenote{
    We blithely disregard integer algebra issues occuring here for conciseness and leave it as an exercise for the reader to figure out what the exact bounds of the set of $\omega_n$ are.
}
\Cref{eq:psd:bartlett} is known as Bartlett's method~\cite{Bartlett1948} for spectrum estimation.\sidenote{
    \label{sidenote:continuum_limit}
    By taking the limit $M\to\infty$ one recovers the true \gls{psd}, \[\lim_{M\to\infty}S_n = S(\omega_n).\]
    The continuum limit is as always obtained by sending $\Delta t\to 0, N\to\infty, N\Delta t=\text{const}$.
}

To better understand the properties of this estimate, let us take a look at the parameters $\Delta t$, $N$, and $M$.
The sampling interval $\Delta t$ defines the largest resolvable frequency by the Nyquist sampling theorem,
\begin{align}\label{eq:f_max}
    \fmax = \frac{\fs}{2} = \frac{1}{2\Delta t}.
\end{align}
In turn, the number of samples $N$ determines the frequency resolution $\Delta f$, or smallest resolvable frequency,
\begin{align}\label{eq:f_min}
    \fmin = \Delta f = \frac{1}{T} = \frac{1}{N\Delta t} = \frac{\fs}{N}.
\end{align}
Lastly, $M$ determines the variance of the set of periodograms $\bigl\lbrace S_n\gth{\nu}\bigr\rbrace_{i=0}^{M-1}$ and hence the accuracy of the estimate $S_n$.

In practice, the ensemble realizations $i$ are of course obtained sequentially, implying that one acquires a time series of data $x_n, n\in\lbrace0, 1, \dotsc, NM - 1\rbrace$ and partitions these data into $M$ sequences of length $N$.
It becomes clear, then, that the Bartlett average (\cref{eq:psd:bartlett}) trades spectral resolution (larger $N$) for estimation accuracy (larger $M$) given the finite acquisition time $T = NM\Delta t$.

An improvement in data efficiency can be obtained using Welch's method~\cite{Welch1967}.
To see how, we first need to discuss spectral windowing.

\section{Window functions}\label{sec:speck:theory:windows}
Partitioning a signal $x_n$ into $M$ sections $x_n\gth{\nu}$ of length $N$ is mathematically equivalent to multiplying the signal with the rectangular \emph{window function} given by\sidenote{
    This window is also known as the boxcar window.
}
\begin{align}\label{eq:window:boxcar}
    w_n\gth{\nu} =
    \begin{cases}
        1\qif (\nu - 1) N \leq n < \nu N\qand \\
        0\qelse
    \end{cases}
\end{align}
so that $x_n\gth{\nu} = x_n w_n\gth{\nu}$.
Now recall that multiplication and convolution are duals under the Fourier transform, implying that
\begin{align}\label{eq:window:ft_pairs}
    \hat{x}_n\gth{\nu} = \hat{x}_n \ast \hat{w}_n\gth{\nu}.
\end{align}
where the Fourier representation of the rectangular window
\begin{align}
    \hat{w}_n\gth{\nu} &= \e^{\i(\nu - \flatfrac{1}{2})\omega_n T}\hat{w}_n, \label{eq:window:boxcar:fourier}\\
             \hat{w}_n &= T\sinc\left(\frac{\omega_n T}{2}\right). \label{eq:window:boxcar:fourier:unshifted}
\end{align}
\Cref{fig:boxcar_fourier} shows the unshifted rectangular window $\hat{w}_n$ in Fourier space.
We can hence understand the Fourier spectrum of $x_n\gth{\nu}$ as sampling $\hat{x}_n$ with the probe $\hat{w}_n\gth{\nu}$.
However, while in the continuum limit (\cf \cref{sidenote:continuum_limit}) \cref{eq:window:boxcar:fourier:unshifted} tends towards $\delta(\omega_n)$ and thus will produce a faithful reconstruction of the true spectrum, the finite frequency spacing $\Delta f$ of discrete signals introduces a finite bandwidth of the probe as well as so-called \emph{side-lobes}.
These effects induce what is known as \emph{spectral leakage}~\cite{Koopmans1995} and lead to artifacts and deviations of the spectrum estimator $S_n$ from the true spectrum $S(\omega_n)$.

\begin{marginfigure}[-2cm]
    \begin{tikzpicture}
    \tikzset{font = \footnotesize}

    \begin{axis} [
        declare function={
            T = 1;
            dirichlet(\x) = 2 * sin(deg(\x * T / 2)) / \x;
        },
        axis lines=middle,
        %line width = 1 pt,
        %scale only axis, % Ensures the axis region is exactly \marginparwidth wide
        width=1.15\marginparwidth,
        height=5cm,
        domain=-11*pi:11*pi,
        samples=200,
        xmin=-12*pi, xmax=12*pi,
        ymin=-0.25, ymax=1.25,
        xtick={-10*pi, 0, 10*pi},
        xticklabels={$\flatfrac{-10\pi}{T}$, $0$, $\flatfrac{10\pi}{T}$},
        ytick={0, 1},
        yticklabels={$$, $T$},
        every tick/.style={
            black,
            thick
        },
        xlabel={$\omega_n$},
        ylabel={$\hat{w}_n$},
        x label style={at={(axis cs:12*pi,0)}, anchor=west, xshift=2pt},
        y label style={at={(current axis.north)}, anchor=south},
    ]

        \addplot [RWTHmagenta100, very thick] {dirichlet(x)};

        %\foreach \x in {-6,-5,-4,-3,-2,-1,1,2,3,4,5,6} {
        %    \draw[RWTHgreen50, thick] (axis cs:\x,0) -- (axis cs:\x,dirichlet(\y))
        %        node [draw=RWTHgreen100, circle, fill=RWTHgreen50, inner sep=2pt, pos=0]  {};
        %}

    \end{axis}
\end{tikzpicture}



    \caption{The Fourier representation of the rectangular window in continuous time.}
    \label{fig:boxcar_fourier}
\end{marginfigure}




