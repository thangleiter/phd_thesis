% mainfile: ../../main.tex
\chapter{The \texttt{python\_spectrometer} software package}\label{ch:speck:software}

\begin{margintable}
    \footnotesize
    \centering
    \setmintedinline[Python]{fontsize=\footnotesize}
    \caption[Overview of spectrum estimation parameters]{
        Variable names used in \cref{ch:speck:theory} and their corresponding parameter names as used in \pyspeck and \code{scipy.signal.welch()}~\cite{WelchScipy}.
    }
    \label{tab:software:parameters}
    \begin{tabular}{ c C }
        \toprule
        Variable & Parameter \\
        \midrule
        $L$ & n_pts \\
        \fs & fs \\
        $K$ & noverlap \\
        $N$ & nperseg \\
        $M$ & n_seg \\
        \fmin & f_min \\
        \fmax & f_max \\
        \bottomrule
    \end{tabular}
\end{margintable}

In this chapter, I will lay out the design and functionality of the \pyspeck \python package.

\section{Package design and implementation}\label{sec:speck:software:design}
The \pyspeck package provides a single user-facing class, \code{core.Spectrometer}, that manages the entire process from data acquisition to spectrum estimation and plotting.
Plotting functionality is factored out into a private submodule for maintainability.

\section{Feature overview}\label{sec:speck:software:features}

