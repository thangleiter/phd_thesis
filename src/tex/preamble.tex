%----------------------------------------------------------------------------------------
%    PACKAGES AND OTHER DOCUMENT CONFIGURATIONS
%----------------------------------------------------------------------------------------

% Choose the language
\ifxetexorluatex
    \usepackage{polyglossia}
    \setmainlanguage[variant=american]{english}
\else
    \usepackage[english]{babel} % Load characters and hyphenation
\fi
\usepackage[english=american]{csquotes}    % English quotes

% Load the bibliography package
\PassOptionsToPackage{
    natbib=true,
    datamodel=software,
}{biblatex}
\usepackage[addspace=false]{kaobiblio}

\usepackage{software-biblatex}
\ExecuteBibliographyOptions{
    halid=false,
    swhid=false,
    swlabels=false,
    vcs=true,
    license=true
}

%\addbibresource[label=ownpubs]{bib/00_own_publications.bib}
%\addbibresource[label=ownsoft]{bib/01_own_software.bib}
%\addbibresource[label=main]{bib/02_phd_thesis.bib}
\addbibresource[label=ownpubs,location=remote]{http://127.0.0.1:23119/better-bibtex/export/collection?/1/00_own_publications.biblatex&useJournalAbbreviation=true}
\addbibresource[label=ownsoft,location=remote]{http://127.0.0.1:23119/better-bibtex/export/collection?/1/01_own_software.biblatex&useJournalAbbreviation=true}
\addbibresource[label=main,location=remote]{http://127.0.0.1:23119/better-bibtex/export/collection?/1/02_phd_thesis.biblatex&useJournalAbbreviation=true}
%\addbibresource[label=filter_function_paper,location=remote]{http://127.0.0.1:23119/better-bibtex/export/collection?/1/03_filter_function_paper.biblatex&useJournalAbbreviation=true}

%----------------------------------------------------------------------------------------
% Custom packages
%----------------------------------------------------------------------------------------
\PassOptionsToPackage{
    chapter,%
    cache=true,%
}{minted}
    \usepackage{minted}

\setminted[python]{
    fontfamily=tt,% tt,courier,helvetica
    style=gruvbox-light,%gruvbox-light,github-light,default
    autogobble=true,%
    frame=leftline,% none | leftline | topline | bottomline | lines | single
    fontsize=\small,% normalsize, small, footnotesize
    linenos=false,
    firstnumber=last,% line number counts incrementally
}
\setmintedinline[python]{
    breaklines,%
    breakafter=._,%
    fontsize=\normalsize,%
}

\usepackage{lettrine}
%\usepackage{Zallman}
%\usepackage{Starburst}
%\usepackage{Rothdn}
% Use these like so:
% \Zallmanfamily
\renewcommand{\LettrineFontHook}{\fontspec{Libertinus Serif Initials}\color{RWTHblue100}}

\usepackage{tabularx} % better tables with given width
%\setlength{\extrarowheight}{3pt} % increase table row height
\usepackage{collcell} % for verb columns

\usepackage{fontawesome5}

%! begin preamble = tikz
\usepackage[edges]{forest}
\forestset{
  dir tree/.style={
    for tree={
      parent anchor=south west,
      child anchor=west,
      anchor=mid west,
      inner ysep=0pt,
      grow'=0,
      align=left,
      s sep=1ex,
      edge path={
        \noexpand\path [draw, \forestoption{edge}] (!u.parent anchor) ++(0.75em,0) |- (.child anchor)\forestoption{edge label};
      },
      font=\footnotesize\ttfamily,
      if n children=0{}{
        delay={
          prepend={[,phantom, calign with current]}
        }
      },
      fit=band,
      before computing xy={
        l=1.5em
      }
    },
  }
}

\usepackage{circuitikz}

%\usepackage{pgfplots}
%\pgfplotsset{compat=1.18}

%\usetikzlibrary{math}
\usetikzlibrary{quantikz2}

%! end preamble = tikz

%! begin preamble = math
%\usepackage{bm}
\usepackage{dsfont}
\usepackage{mathtools}
%\usepackage{mathrsfs}
\usepackage{siunitx}
\usepackage{physics}
\AtBeginDocument{\RenewCommandCopy\qty\SI}
%! end preamble = math
%----------------------------------------------------------------------------------------

% Load mathematical packages for theorems and related environments
\usepackage[framed=true]{kaotheorems}

% Load the package for hyperreferences
\usepackage{kaorefs}

\graphicspath{{img/}{./}} % Paths in which to look for images

% \makeindex[columns=3, title=Alphabetical Index, intoc] % Make LaTeX produce the files required to compile the index

\makeglossaries % Make LaTeX produce the files required to compile the glossary
\newglossaryentry{computer}{
	name=computer,
	description={is a programmable machine that receives input, stores and manipulates data, and provides output in a useful format}
}

% Glossary entries (used in text with e.g. \acrfull{fpsLabel} or \acrshort{fpsLabel})
\newacronym[longplural={Frames per Second}]{fpsLabel}{FPS}{Frame per Second}
\newacronym[longplural={Tables of Contents}]{tocLabel}{TOC}{Table of Contents}

\newacronym{rb}{RB}{randomized benchmarking}
\newacronym{srb}{SRB}{standard randomized benchmarking}
\newacronym{irb}{IRB}{interleaved randomized benchmarking}
\newacronym{gst}{GST}{gate set tomography}
\newacronym{qpt}{QPT}{quantum process tomography}
\newacronym{qec}{QEC}{quantum error correction}
\newacronym{se}{SE}{spin echo}
\newacronym{ptm}{PTM}{Pauli transfer matrix}
\newacronym{me}{ME}{Magnus expansion}
\newacronym{dd}{DD}{dynamical decoupling}
\newacronym{dcg}{DCG}{dynamically corrected gate}
\newacronym{ff}{FF}{filter function}
\newacronym{mc}{MC}{Monte Carlo}
\newacronym{cff}{CFF}{correlation filter function}
\newacronym{qft}{QFT}{quantum Fourier transform}
\newacronym{cp}{CP}{completely positive}
\newacronym{povm}{POVM}{positive operator-valued measure}
\newacronym{iid}{i.i.d.}{independent and identically distributed}
\newacronym{psd}{PSD}{power spectral density}
\newacronym{tlf}{TLF}{two-level fluctuator}
\newacronym{rms}{RMS}{root mean square}
\newacronym{daq}{DAQ}{data acquisition device}
\newacronym{dac}{DAC}{digital-to-analog converter}
\newacronym{adc}{ADC}{analog-to-digital converter}
\newacronym{tia}{TIA}{transimpedance amplifier}
\newacronym{fft}{FFT}{fast Fourier transform}
 % Include the glossary definitions

% \makenomenclature % Make LaTeX produce the files required to compile the nomenclature

%----------------------------------------------------------------------------------------
% DEBUG
%----------------------------------------------------------------------------------------
%\usepackage{blindtext} % Print text without any meaning for testing purposes
%\usepackage{showframe} % Uncomment to show boxes around the text area, margin, header and footer
%\usepackage{showlabels} % Uncomment to output the content of \label commands to the document where they are used

%\usepackage{printlen}
%\uselengthunit{in}
% Use:
% \printlength{\marginparsep}

%\makeatletter
%\newcommand\thefontsize[1]{{#1 The current font size is: \f@size pt\par}}
%\makeatother
% Use:
% \thefontsize\footnotesize
%----------------------------------------------------------------------------------------
\includeonly{
    chs/spectrometer/introduction,
    chs/spectrometer/theory,
    chs/spectrometer/software,
    chs/spectrometer/conclusion,
}

