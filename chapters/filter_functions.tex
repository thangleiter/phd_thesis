\chapter{Filter Functions}\label{ch:filter_functions}
For a given quantum operation \liouvUe resulting from the quantum system's evolution under the noise fully characterized by its one-sided \gls{psd} $S(\omega)$, we define the \gls{ff} \FF by
\begin{align}
    \label{eq:filter_function:definition}
    \liouvUe(\tau) = \exp\int\frac{\dd{\omega}}{2\pi}\FF S(\omega).
\end{align}
Now, suppose that
\begin{align}
    \label{eq:psd:monochromatic}
    S_{\omega_i}(\omega) = \sigma_i^2 \delta(\omega - \omega_i), %\todo{check normalization}
\end{align}
that is, the \gls{psd} of a monochromatic sinusoid of frequency $\omega_i$ and \gls{rms} $\sigma_i^2$\footnote{
    \Cref{eq:psd:monochromatic} discretizes $S(\omega)$ by sampling it at points $\omega_i$, \ie,
    \begin{align}
        S(\omega) = \lim_{n\to\infty}\sum_{i=1}^n S_{\omega_i}(\omega).
    \end{align}
}. %\todo{marginnote?}
Then \cref{eq:filter_function:definition} becomes
\begin{equation}
    \label{eq:filter_function:monochromatic}
    \begin{split}
        \liouvUe_{\omega_i}(\tau) =& \exp\left\lbrace\sigma_i^2\int\frac{\dd{\omega}}{2\pi}\FF\delta(\omega - \omega_i)\right\rbrace \\
                             =& \exp\left\lbrace\frac{\sigma_i^2}{2\pi}\mc{F}(\omega_i;\tau)\right\rbrace,
    \end{split}
\end{equation}
where $\liouvUe_{\omega_i}(\tau)$ is the noisy quantum operation generated by monochromatic noise with \gls{psd} $S_{\omega_i}(\omega)$ according to \cref{eq:psd:monochromatic}.
It is now easy to invert \cref{eq:filter_function:monochromatic}, and we obtain
\begin{align}
    \label{eq:filter_function:monte_carlo}
    \mc{F}(\omega_i; \tau) = \frac{2\pi}{\sigma_i^2}\log\liouvUe_{\omega_i}(\tau).
\end{align}
Because we represent quantum operations as matrices in Liouville space, \cref{eq:filter_function:monte_carlo} is easy to implement on a computer; we simply need to evaluate $\liouvUe_{\omega_i}(\tau)$ for a set of discrete frequencies $\lbrace\omega_i\rbrace_i$ using \gls{mc} and take the logarithm! \cite{Geck2021}.%\todo{citation}

Indeed, we can go a step further and split apart the coherent and incoherent contributions to the noisy evolution.
Since (in-)coherent quantum operations are represented by (anti-)symmetric matrices in Liouville space, we may define the incoherent and coherent \glspl{ff} by
\begin{equation}
    \label{eq:filter_function:incoherent}
    \begin{split}
        \mc{F}_\Gamma(\omega;\tau) =& \frac{1}{2}\left(\FF + \FF\transpose\right) \\
                                   =& \frac{\pi}{\sigma_i^2}\left(\log\liouvUe_\omega(\tau) + \log\liouvUe_\omega(\tau)\transpose\right),
    \end{split}
\end{equation}
and
\begin{equation}
    \label{eq:filter_function:coherent}
    \begin{split}
        \mc{F}_\Delta(\omega;\tau) =& \frac{1}{2}\left(\FF - \FF\transpose\right) \\
                                   =& \frac{\pi}{\sigma_i^2}\left(\log\liouvUe_\omega(\tau) - \log\liouvUe_\omega(\tau)\transpose\right),
    \end{split}
\end{equation}
respectively.



From a \gls{mc} point of view, \liouvUe is given by
\begin{align}
    \expval{\liouvUe}(\tau) = \liouvQ\transpose \expval{\liouvU(\tau)},
\end{align}
where $\liouvU(\tau)$ is the solution of the Schrödinger equation for a single realization of the noise.
Solving \cref{eq:filter_function:definition} for \FF, we find
\begin{align}
    \FF = \int\frac{\dd{\omega'}}{2\pi} \delta(\omega-\omega') \log\expval{\liouvUe(\tau)}
\end{align}
