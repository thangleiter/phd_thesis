\begin{tikzpicture}[
    >=stealth,
    auto,
    box/.style={
        draw,
        rectangle,
        rounded corners,
        thick,
        align=center,
        inner sep=2mm,
%        drop shadow,
    },
    float1/.style={box, fill=RWTHmagenta25, text=RWTHmagenta100},
    float2/.style={box, fill=RWTHmagenta10, text=RWTHmagenta75},
    int1/.style={box, fill=RWTHgreen25, text=RWTHgreen100},
    int2/.style={box, fill=RWTHgreen10, text=RWTHgreen75},
    int3/.style={box, fill=RWTHblack10, text=RWTHblack100},
]

    % Central node: nperseg
    \node[int1] (nperseg) {$N$};

    % Frequency branch: relative to nperseg, all nodes placed at the same x-coordinate.
    \node[float1, above left=1cm of nperseg, anchor=center] (fs) {\fs};
    \node[float1, below left=1cm of nperseg, anchor=center] (df) {\df};
    \node[float2, left=of fs] (fmax) {\fmax};
    \node[float2, left=of df] (fmin) {\fmin};

    % Segmentation branch: relative to nperseg
    \node[int2, right=of nperseg] (npts) {$L$};
    \node[int2, above right=of npts, anchor=center] (noverlap) {$K$};
    \node[int2, below right=of npts, anchor=center] (nseg) {$M$};

    % Stand-alone node for n_avg
    \node[int3, above=1cm of $(npts)!0.5!(nperseg)$] (navg) {$O$};

    % Draw frequency branch arrows
    \draw[->] (nperseg) -- (fs);
    \draw[->] (nperseg) -- (df);
    \draw[<->] (df) to[bend left=45] (fs);
    \draw[->] (fs) to[bend right=45] (fmax);
    \draw[->] (df) to[bend left=45] (fmin);
    \draw[<->] (fs) -- (nperseg);
    \draw[<->] (df) -- (nperseg);

    % Draw segmentation branch arrows (nperseg, noverlap, and n_seg together inform n_pts)
    \draw[<->] (nperseg) -- (npts);
    \draw[->] (noverlap) -- (npts);
    \draw[->] (nseg) -- (npts);

    % n_avg remains independent (no arrows drawn)

\end{tikzpicture}
