%%%%%%%%%%%%%%%%%%%%%%%%%%%%%%%%%%%%%%%%%
% kaobook
% LaTeX Class
% Version 0.9.8 (2021/08/23)
%
% This template originates from:
% https://www.LaTeXTemplates.com
%
% For the latest template development version and to make contributions:
% https://github.com/fmarotta/kaobook
%
% Authors:
% Federico Marotta (federicomarotta@mail.com)
% Based on the doctoral thesis of Ken Arroyo Ohori (https://3d.bk.tudelft.nl/ken/en)
% and on the Tufte-LaTeX class.
% Modified for LaTeX Templates by Vel (vel@latextemplates.com)
%
% License:
% LPPL (see included MANIFEST.md file)
%
%%%%%%%%%%%%%%%%%%%%%%%%%%%%%%%%%%%%%%%%%

%----------------------------------------------------------------------------------------
%	EXAMPLE AND DOCUMENTATION OF THE KAOBOOK CLASS
%----------------------------------------------------------------------------------------

\documentclass[
    a4paper, % Page size
    fontsize=10pt, % Base font size
    twoside=, % Use different layouts for even and odd pages (in particular, if twoside=true, the margin column will be always on the outside)
	%open=any, % If twoside=true, uncomment this to force new chapters to start on any page, not only on right (odd) pages
	%chapterentrydots=true, % Uncomment to output dots from the chapter name to the page number in the table of contents
	numbers=noenddot, % Comment to output dots after chapter numbers; the most common values for this option are: enddot, noenddot and auto (see the KOMAScript documentation for an in-depth explanation)
	fontmethod=modern, % Can also use "modern" with XeLaTeX or LuaTex; "tex" is the default for PdfLaTex, and "modern" is the default for those two.
	listing=minted
]{kaobook}

%----------------------------------------------------------------------------------------
%	PACKAGES AND OTHER DOCUMENT CONFIGURATIONS
%----------------------------------------------------------------------------------------

% Choose the language
\ifxetexorluatex
	\usepackage{polyglossia}
	\setmainlanguage[variant=american]{english}
\else
	\usepackage[english]{babel} % Load characters and hyphenation
\fi
\usepackage[english=american]{csquotes}	% English quotes

% Load packages for testing
%\usepackage{blindtext} % Print text without any meaning for testing purposes
%\usepackage{showframe} % Uncomment to show boxes around the text area, margin, header and footer
%\usepackage{showlabels} % Uncomment to output the content of \label commands to the document where they are used

% Load the bibliography package
\PassOptionsToPackage{
	natbib=true,
	datamodel=software,
}{biblatex}
\usepackage[addspace=false]{kaobiblio}

\usepackage{software-biblatex}
\ExecuteBibliographyOptions{
	halid=false,
	swhid=false,
	swlabels=false,
	vcs=true,
	license=true
}

%\addbibresource[label=ownpubs]{bib/00_own_publications.bib}
%\addbibresource[label=ownsoft]{bib/01_own_software.bib}
%\addbibresource[label=main]{bib/02_phd_thesis.bib}
\addbibresource[label=ownpubs,location=remote]{http://127.0.0.1:23119/better-bibtex/export/collection?/1/00_own_publications.biblatex&useJournalAbbreviation=true}
\addbibresource[label=ownsoft,location=remote]{http://127.0.0.1:23119/better-bibtex/export/collection?/1/01_own_software.biblatex&useJournalAbbreviation=true}
\addbibresource[label=main,location=remote]{http://127.0.0.1:23119/better-bibtex/export/collection?/1/02_phd_thesis.biblatex&useJournalAbbreviation=true}
%\addbibresource[label=filter_function_paper,location=remote]{http://127.0.0.1:23119/better-bibtex/export/collection?/1/03_filter_function_paper.biblatex&useJournalAbbreviation=true}

%----------------------------------------------------------------------------------------
% Custom packages
%----------------------------------------------------------------------------------------
\PassOptionsToPackage{
    chapter,%
    cache=true,%
}{minted}
    \usepackage{minted}

\setminted[Python]{
    autogobble=true,%
    linenos=true,%
    firstnumber=last,% line number counts incrementally
    fontfamily=tt,% tt,courier,helvetica
    fontsize=\small,% normalsize, small, footnotesize
    style=gruvbox-light,%
    frame=leftline,% none | leftline | topline | bottomline | lines | single
}
\setmintedinline[Python]{
    fontsize=\normalsize,% normalsize, small, footnotesize
}

\usepackage{tabularx} % better tables with given width
%\setlength{\extrarowheight}{3pt} % increase table row height
\usepackage{collcell} % for verb columns

%! begin preamble = tikz
%\usepackage{pgfplots}
%\pgfplotsset{compat=1.18}

%\usetikzlibrary{math}
\usetikzlibrary{quantikz2}

%! end preamble = tikz

%! begin preamble = math
%\usepackage{bm}
\usepackage{dsfont}
\usepackage{mathtools}
%\usepackage{mathrsfs}
\usepackage{siunitx}
\usepackage{physics}
\AtBeginDocument{\RenewCommandCopy\qty\SI}
%! end preamble = math
%----------------------------------------------------------------------------------------

% Load mathematical packages for theorems and related environments
\usepackage[framed=true]{kaotheorems}

% Load the package for hyperreferences
\usepackage{kaorefs}

\graphicspath{{img/}} % Paths in which to look for images

% \makeindex[columns=3, title=Alphabetical Index, intoc] % Make LaTeX produce the files required to compile the index

\makeglossaries % Make LaTeX produce the files required to compile the glossary
\usepackage{glossaries}\newglossaryentry{computer}{
	name=computer,
	description={is a programmable machine that receives input, stores and manipulates data, and provides output in a useful format}
}

% Glossary entries (used in text with e.g. \acrfull{fpsLabel} or \acrshort{fpsLabel})
\newacronym[longplural={Frames per Second}]{fpsLabel}{FPS}{Frame per Second}
\newacronym[longplural={Tables of Contents}]{tocLabel}{TOC}{Table of Contents}

\newacronym{rb}{RB}{randomized benchmarking}
\newacronym{srb}{SRB}{standard randomized benchmarking}
\newacronym{irb}{IRB}{interleaved randomized benchmarking}
\newacronym{gst}{GST}{gate set tomography}
\newacronym{qpt}{QPT}{quantum process tomography}
\newacronym{qec}{QEC}{quantum error correction}
\newacronym{se}{SE}{spin echo}
\newacronym{ptm}{PTM}{Pauli transfer matrix}
\newacronym{me}{ME}{Magnus expansion}
\newacronym{dd}{DD}{dynamical decoupling}
\newacronym{dcg}{DCG}{dynamically corrected gate}
\newacronym{ff}{FF}{filter function}
\newacronym{mc}{MC}{Monte Carlo}
\newacronym{cff}{CFF}{correlation filter function}
\newacronym{qft}{QFT}{quantum Fourier transform}
\newacronym{cp}{CP}{completely positive}
\newacronym{povm}{POVM}{positive operator-valued measure}
\newacronym{iid}{i.i.d.}{independent and identically distributed}
\newacronym[longplural={power spectral densities}]{psd}{PSD}{power spectral density}
\newacronym[longplural={cross power spectral densities}]{csd}{CSD}{cross power spectral density}
\newacronym[longplural={amplitude spectral densities}]{asd}{ASD}{amplitude spectral density}
\newacronym{enbw}{ENBW}{equivalent noise bandwidth}
\newacronym{tlf}{TLF}{two-level fluctuator}
\newacronym{rms}{RMS}{root mean square}
\newacronym{daq}{DAQ}{data acquisition}
\newacronym{dac}{DAC}{digital-to-analog converter}
\newacronym{adc}{ADC}{analog-to-digital converter}
\newacronym{tia}{TIA}{transimpedance amplifier}
\newacronym{lia}{LIA}{lock-in amplifier}
\newacronym{fft}{FFT}{fast Fourier transform}
\newacronym{dut}{DUT}{device under test}
\newacronym{api}{API}{Application Programming Interface}
\newacronym{gui}{GUI}{graphical user interface}
\newacronym{dr}{DR}{dilution refrigerator}
\newacronym{ptr}{PTR}{pulse tube refrigerator}
\newacronym{sem}{SEM}{Scanning electron microscope}
\newacronym{pt1}{PT1}{first pulse tube stage}
\newacronym{pt2}{PT2}{second pulse tube stage}
\newacronym{mxc}{MXC}{mixing chamber}
\newacronym{sma}{SMA}{SubMiniature version A}
\newacronym{bnc}{BNC}{Bayonet Neill–Concelman}
\newacronym{dmm}{DMM}{digital multimeter}
\newacronym{smf}{SMF}{single-mode fiber}
\newacronym{mmf}{MMF}{multi-mode fiber}
\newacronym{ebl}{EBL}{electron-beam lithography}
\newacronym{tem}{TEM}{transverse electromagnetic}
\newacronym{te}{TE}{transverse electric}
\newacronym{qw}{QW}{quantum well}
\newacronym{apd}{APD}{avalanche photodiode}
\newacronym{spcm}{SPCM}{single-photon counting module}
\newacronym{pde}{PDE}{photon detection efficiency}
\newacronym{hbt}{HBT}{Hanbury Brown-Twiss}
\newacronym{cmos}{CMOS}{complementary metal-oxide-semiconductor}
\newacronym[longplural={vibration criteria}]{vc}{VC}{vibration criterion}
\newacronym{iso}{ISO}{International Organization for Standardization}
\newacronym{snr}{SNR}{signal-to-noise ratio}
\newacronym{ar}{AR}{anti-reflection}
\newacronym{bbar}{BBAR}{broadband anti-reflection}
\newacronym{nir}{NIR}{near-infrared}
\newacronym{na}{NA}{numerical aperture}
\newacronym{ca}{CA}{clear aperture}
\newacronym{wd}{WD}{working distance}
\newacronym{mfd}{MFD}{mode field diameter}
\newacronym{bs}{BS}{beam splitter}
\newacronym{ccd}{CCD}{charge-coupled device}
\newacronym{pcc}{PCC}{photonic crystal cavity}
\newacronym{los}{LOS}{line-of-sight}
\newacronym{qd}{QD}{quantum dot}
\newacronym{gdqd}{GDQD}{gate-defined quantum dot}
\newacronym{oaqd}{OAQD}{optically active quantum dot}
\newacronym{saqd}{SAQD}{self-assembled quantum dot}
\newacronym[longplural={two-dimensional electron gases}]{2deg}{2DEG}{two-dimensional electron gas}
\newacronym[longplural={two-dimensional hole gases}]{2dhg}{2DHG}{two-dimensional hole gas}
\newacronym{bob}{BOB}{break-out box}
\newacronym{pl}{PL}{photoluminescence}
\newacronym{ple}{PLE}{photoluminescence excitation}
\newacronym{cw}{cw}{continuous-wave}
\newacronym{tmm}{TMM}{transfer-matrix method}
\newacronym{blg}{BLG}{bilayer graphene}
\newacronym{tmd}{TMD}{transition-metal dichalcogenide}
\newacronym{od}{OD}{optical density}
\newacronym{qe}{QE}{quantum efficiency}
\newacronym{nd}{ND}{neutral-density}
\newacronym{pcb}{PCB}{printed circuit board}
\newacronym{soi}{SOI}{spin-orbit interaction}
\newacronym{bec}{BEC}{Bose-Einstein condensate}
\newacronym{qcse}{QCSE}{quantum-confined Stark effect}
\newacronym{fes}{FES}{Fermi-edge singularity}
 % Include the glossary definitions

% \makenomenclature % Make LaTeX produce the files required to compile the nomenclature

% Reset sidenote counter at chapters
%\counterwithin*{sidenote}{chapter}

%----------------------------------------------------------------------------------------
% DEBUG
%----------------------------------------------------------------------------------------
%\usepackage{printlen}
%\uselengthunit{in}
% Use:
% \printlength{\marginparsep}

%\makeatletter
%\newcommand\thefontsize[1]{{#1 The current font size is: \f@size pt\par}}
%\makeatother
% Use:
% \thefontsize\footnotesize
%----------------------------------------------------------------------------------------
\includeonly{
    chs/appendices/filter_functions,
    chs/filter_functions/time_domain_methods,
    chs/filter_functions/prr
}

